% Persian Book Template
% Complete LaTeX template for Persian books with proper RTL support
% Compile with XeLaTeX or LuaLaTeX

\documentclass[12pt,a4paper,oneside]{book}

% Essential packages for Persian
\usepackage{fontspec}
\usepackage{xepersian}
\usepackage{bidi}

% Additional useful packages
\usepackage{geometry}
\usepackage{graphicx}
\usepackage{hyperref}
\usepackage{fancyhdr}
\usepackage{titlesec}
\usepackage{tocloft}
\usepackage{enumitem}
\usepackage{amsmath}
\usepackage{amsfonts}
\usepackage{amssymb}
\usepackage{xcolor}
\usepackage{booktabs}
\usepackage{longtable}
\usepackage{float}

% Page geometry
\geometry{
    top=3cm,
    bottom=3cm,
    left=3cm,
    right=3cm,
    headheight=15pt
}

% Persian fonts (adjust based on your system)
\settextfont[Scale=1.2]{Vazir}  % Main text font
\setdigitfont{Vazir}           % Digits font
\setlatintextfont{Times New Roman} % For Latin text within Persian

% Alternative fonts (uncomment if you prefer):
% \settextfont{B Nazanin}
% \settextfont{IRLotus}
% \settextfont{XB Niloofar}

% Hyperref setup
\hypersetup{
    colorlinks=true,
    linkcolor=blue,
    filecolor=magenta,
    urlcolor=cyan,
    citecolor=red,
    bookmarks=true,
    bookmarksopen=true,
    bookmarksnumbered=true,
    pdftitle={عنوان کتاب},
    pdfauthor={نام نویسنده},
    pdfsubject={موضوع کتاب},
    pdfkeywords={کلیدواژه‌ها}
}

% Header and footer
\pagestyle{fancy}
\fancyhf{}
\fancyhead[LE,RO]{\thepage}
\fancyhead[LO]{\rightmark}
\fancyhead[RE]{\leftmark}
\renewcommand{\headrulewidth}{0.4pt}

% Chapter and section formatting
\titleformat{\chapter}[display]
{\normalfont\huge\bfseries\centering}
{\chaptertitlename\ \thechapter}{20pt}{\Huge}

\titleformat{\section}
{\normalfont\Large\bfseries}
{\thesection}{1em}{}

\titleformat{\subsection}
{\normalfont\large\bfseries}
{\thesubsection}{1em}{}

% Table of contents formatting
\renewcommand{\contentsname}{فهرست مطالب}
\renewcommand{\listfigurename}{فهرست تصاویر}
\renewcommand{\listtablename}{فهرست جداول}
\renewcommand{\bibname}{منابع}
\renewcommand{\indexname}{نمایه}
\renewcommand{\figurename}{شکل}
\renewcommand{\tablename}{جدول}
\renewcommand{\chaptername}{فصل}
\renewcommand{\appendixname}{پیوست}

% Custom commands for better Persian typography
\newcommand{\persianquote}[1]{«#1»}
\newcommand{\englishtext}[1]{\lr{#1}}
\newcommand{\persiannum}[1]{\lr{#1}}

% Document information
\title{عنوان کتاب}
\author{نام نویسنده}
\date{\today}

\begin{document}

% Front matter
\frontmatter

% Title page
\begin{titlepage}
\centering
\vspace*{2cm}

{\Huge\bfseries عنوان کتاب}

\vspace{1.5cm}

{\Large نویسنده: نام نویسنده}

\vspace{1cm}

{\large مترجم: نام مترجم}

\vspace{2cm}

% Add publisher logo or image here if needed
% \includegraphics[width=0.3\textwidth]{logo.png}

\vspace{2cm}

{\large انتشارات نام ناشر}

\vspace{1cm}

{\large سال انتشار}

\vfill

\end{titlepage}

% Copyright page
\newpage
\thispagestyle{empty}
\vspace*{10cm}

\noindent
عنوان اصلی: Original English Title\\
نویسنده: Original Author Name\\
مترجم: Translator Name\\
ناشر: Publisher Name\\
سال انتشار: Publication Year\\
شابک: ISBN Number\\

\vspace{1cm}

\noindent
تمامی حقوق این کتاب محفوظ است و هرگونه تکثیر، انتشار یا استفاده از مطالب آن بدون اجازه کتبی ناشر ممنوع می‌باشد.

\newpage

% Dedication (optional)
\thispagestyle{empty}
\vspace*{8cm}
\begin{center}
{\large\itshape تقدیم به...}
\end{center}
\newpage

% Preface
\chapter*{پیشگفتار}
\addcontentsline{toc}{chapter}{پیشگفتار}

در اینجا پیشگفتار کتاب نوشته می‌شود. این بخش معمولاً شامل توضیحاتی درباره هدف کتاب، روش ترجمه، و تشکر از افرادی است که در تهیه کتاب مشارکت داشته‌اند.

% Table of contents
\tableofcontents

% List of figures (if needed)
% \listoffigures

% List of tables (if needed)
% \listoftables

% Main matter
\mainmatter

% Chapter 1
\chapter{عنوان فصل اول}
\label{ch:chapter1}

این فصل اول کتاب است. در اینجا محتوای اصلی کتاب شروع می‌شود.

\section{عنوان بخش اول}
\label{sec:section1}

محتوای بخش اول در اینجا نوشته می‌شود. برای نوشتن متن فارسی، کافی است به صورت عادی تایپ کنید.

\subsection{عنوان زیربخش}

محتوای زیربخش در اینجا قرار می‌گیرد.

% Example of including an image
\begin{figure}[h]
\centering
% \includegraphics[width=0.8\textwidth]{image.png}
\caption{عنوان تصویر}
\label{fig:example}
\end{figure}

% Example of a table
\begin{table}[h]
\centering
\caption{نمونه جدول}
\label{tab:example}
\begin{tabular}{|c|c|c|}
\hline
ستون اول & ستون دوم & ستون سوم \\
\hline
داده ۱ & داده ۲ & داده ۳ \\
داده ۴ & داده ۵ & داده ۶ \\
\hline
\end{tabular}
\end{table}

% Example of lists
\section{نمونه فهرست‌ها}

فهرست نقطه‌ای:
\begin{itemize}
\item مورد اول
\item مورد دوم
\item مورد سوم
\end{itemize}

فهرست شماره‌دار:
\begin{enumerate}
\item مورد اول
\item مورد دوم
\item مورد سوم
\end{enumerate}

% Example of mathematical formulas
\section{فرمول‌های ریاضی}

فرمول درون متن: $E = mc^2$

فرمول جداگانه:
\begin{equation}
\int_{a}^{b} f(x) dx = F(b) - F(a)
\label{eq:fundamental}
\end{equation}

% Example of cross-references
همان‌طور که در شکل~\ref{fig:example} مشاهده می‌کنید و در جدول~\ref{tab:example} آمده است، و طبق معادله~\ref{eq:fundamental}...

% Chapter 2
\chapter{عنوان فصل دوم}
\label{ch:chapter2}

محتوای فصل دوم در اینجا قرار می‌گیرد.

% Add more chapters as needed...

% Back matter
\backmatter

% Bibliography
\begin{thebibliography}{99}
\addcontentsline{toc}{chapter}{منابع}

\bibitem{ref1}
نام نویسنده، «عنوان مقاله»، نام مجله، شماره، سال، صفحات.

\bibitem{ref2}
نام نویسنده، \textit{عنوان کتاب}، نام ناشر، شهر انتشار، سال انتشار.

\end{thebibliography}

% Index (optional)
% \printindex

\end{document}