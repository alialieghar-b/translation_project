\documentclass{article}
\usepackage{fontspec}
\usepackage{xcolor}
\usepackage{geometry}
\usepackage{listings}
\usepackage{qrcode}
\usepackage{tcolorbox}
\usepackage{hyperref}
\usepackage{graphicx}

% Page setup
\geometry{a4paper, margin=1in}

% Configure listings for visual display
\lstset{
    basicstyle=\ttfamily\small,
    breaklines=true,
    breakatwhitespace=true,
    breakindent=20pt,
    frame=single,
    backgroundcolor=\color{gray!10},
    postbreak=\mbox{\textcolor{red}{$\hookrightarrow$}\space}
}

% Custom command for code with QR
\newtcolorbox{codeqr}[2]{
    enhanced,
    colback=gray!5,
    colframe=blue!50,
    title=#1,
    fonttitle=\bfseries,
    attach boxed title to top left={yshift=-2mm, xshift=2mm},
    boxed title style={colback=blue!20},
    sidebyside,
    sidebyside align=top,
    sidebyside gap=10pt,
    righthand width=2.5cm,
    lower separated=false
}

% Smaller QR code command
\newcommand{\qrsmall}[1]{%
    \qrcode[height=2cm, tight, level=M]{#1}%
}

\title{Oh My Zsh Installation Guide\\
\large PDF Version with QR Code Copy}
\author{Interactive PDF Solution}
\date{\today}

\begin{document}

\maketitle

\section{How to Use This Guide}

\begin{tcolorbox}[colback=blue!5, colframe=blue!40, title=📱 QR Code Instructions]
\textbf{How to copy commands:}
\begin{enumerate}
    \item Use your phone's camera or QR scanner app
    \item Scan the QR code next to any command
    \item The full command will appear on your phone
    \item Copy from your phone and paste into terminal
\end{enumerate}

\textbf{Why QR codes?}
\begin{itemize}
    \item Works with any PDF viewer
    \item No JavaScript required
    \item Mobile-friendly
    \item Always copies the correct single-line command
\end{itemize}
\end{tcolorbox}

\section{Step 1: Install Zsh}

\subsection{Ubuntu/Debian Systems}

\begin{codeqr}{Update Package List}{}
\begin{lstlisting}
sudo apt update
\end{lstlisting}
\tcblower
\qrsmall{sudo apt update}
\end{codeqr}

\begin{codeqr}{Install Zsh}{}
\begin{lstlisting}
sudo apt install zsh
\end{lstlisting}
\tcblower
\qrsmall{sudo apt install zsh}
\end{codeqr}

\subsection{CentOS/RHEL Systems}

\begin{codeqr}{Install Zsh (CentOS/RHEL)}{}
\begin{lstlisting}
sudo yum install zsh
\end{lstlisting}
\tcblower
\qrsmall{sudo yum install zsh}
\end{codeqr}

\subsection{Fedora Systems}

\begin{codeqr}{Install Zsh (Fedora)}{}
\begin{lstlisting}
sudo dnf install zsh
\end{lstlisting}
\tcblower
\qrsmall{sudo dnf install zsh}
\end{codeqr}

\subsection{macOS Systems}

\begin{codeqr}{Install Zsh (Homebrew)}{}
\begin{lstlisting}
brew install zsh
\end{lstlisting}
\tcblower
\qrsmall{brew install zsh}
\end{codeqr}

\subsection{Verify Installation}

\begin{codeqr}{Check Zsh Version}{}
\begin{lstlisting}
zsh --version
\end{lstlisting}
\tcblower
\qrsmall{zsh --version}
\end{codeqr}

\begin{codeqr}{Find Zsh Location}{}
\begin{lstlisting}
which zsh
\end{lstlisting}
\tcblower
\qrsmall{which zsh}
\end{codeqr}

\section{Step 2: Install Oh My Zsh}

\subsection{Method 1: Using curl}

\begin{codeqr}{Install Oh My Zsh (curl)}{}
\begin{lstlisting}
sh -c "$(curl -fsSL 
https://raw.githubusercontent.com/ohmyzsh/ohmyzsh/master/tools/install.sh)"
\end{lstlisting}
\tcblower
\qrsmall{sh -c "$(curl -fsSL https://raw.githubusercontent.com/ohmyzsh/ohmyzsh/master/tools/install.sh)"}
\end{codeqr}

\subsection{Method 2: Using wget}

\begin{codeqr}{Install Oh My Zsh (wget)}{}
\begin{lstlisting}
sh -c "$(wget 
https://raw.githubusercontent.com/ohmyzsh/ohmyzsh/master/tools/install.sh -O -)"
\end{lstlisting}
\tcblower
\qrsmall{sh -c "$(wget https://raw.githubusercontent.com/ohmyzsh/ohmyzsh/master/tools/install.sh -O -)"}
\end{codeqr}

\section{Step 3: Change Default Shell}

\subsection{Standard Systems}

\begin{codeqr}{Change Shell (Standard)}{}
\begin{lstlisting}
chsh -s $(which zsh)
\end{lstlisting}
\tcblower
\qrsmall{chsh -s $(which zsh)}
\end{codeqr}

\subsection{GitHub Codespaces}

\begin{codeqr}{Change Shell (Codespaces Method 1)}{}
\begin{lstlisting}
sudo chsh -s $(which zsh) $(whoami)
\end{lstlisting}
\tcblower
\qrsmall{sudo chsh -s $(which zsh) $(whoami)}
\end{codeqr}

\begin{codeqr}{Change Shell (Codespaces Method 2)}{}
\begin{lstlisting}
sudo chsh "$(id -un)" --shell "$(which zsh)"
\end{lstlisting}
\tcblower
\qrsmall{sudo chsh "$(id -un)" --shell "$(which zsh)"}
\end{codeqr}

\section{Step 4: Install Popular Plugins}

\subsection{Zsh Autosuggestions}

\begin{codeqr}{Install Autosuggestions Plugin}{}
\begin{lstlisting}
git clone https://github.com/zsh-users/zsh-autosuggestions 
${ZSH_CUSTOM:-~/.oh-my-zsh/custom}/plugins/zsh-autosuggestions
\end{lstlisting}
\tcblower
\qrsmall{git clone https://github.com/zsh-users/zsh-autosuggestions ${ZSH_CUSTOM:-~/.oh-my-zsh/custom}/plugins/zsh-autosuggestions}
\end{codeqr}

\subsection{Zsh Syntax Highlighting}

\begin{codeqr}{Install Syntax Highlighting Plugin}{}
\begin{lstlisting}
git clone https://github.com/zsh-users/zsh-syntax-highlighting.git 
${ZSH_CUSTOM:-~/.oh-my-zsh/custom}/plugins/zsh-syntax-highlighting
\end{lstlisting}
\tcblower
\qrsmall{git clone https://github.com/zsh-users/zsh-syntax-highlighting.git ${ZSH_CUSTOM:-~/.oh-my-zsh/custom}/plugins/zsh-syntax-highlighting}
\end{codeqr}

\subsection{Powerlevel10k Theme}

\begin{codeqr}{Install Powerlevel10k Theme}{}
\begin{lstlisting}
git clone --depth=1 https://github.com/romkatv/powerlevel10k.git 
${ZSH_CUSTOM:-$HOME/.oh-my-zsh/custom}/themes/powerlevel10k
\end{lstlisting}
\tcblower
\qrsmall{git clone --depth=1 https://github.com/romkatv/powerlevel10k.git ${ZSH_CUSTOM:-$HOME/.oh-my-zsh/custom}/themes/powerlevel10k}
\end{codeqr}

\section{Step 5: Apply Changes}

\begin{codeqr}{Reload Configuration}{}
\begin{lstlisting}
source ~/.zshrc
\end{lstlisting}
\tcblower
\qrsmall{source ~/.zshrc}
\end{codeqr}

\section{Verification Commands}

\begin{codeqr}{Check Current Shell}{}
\begin{lstlisting}
echo $SHELL
\end{lstlisting}
\tcblower
\qrsmall{echo $SHELL}
\end{codeqr}

\begin{codeqr}{Check User Shell Setting}{}
\begin{lstlisting}
grep $(whoami) /etc/passwd
\end{lstlisting}
\tcblower
\qrsmall{grep $(whoami) /etc/passwd}
\end{codeqr}

\section{Troubleshooting}

\begin{codeqr}{Fix Permissions (if needed)}{}
\begin{lstlisting}
sudo chsh -s $(which zsh) $USER
\end{lstlisting}
\tcblower
\qrsmall{sudo chsh -s $(which zsh) $USER}
\end{codeqr}

\begin{codeqr}{Check Plugin Directory}{}
\begin{lstlisting}
ls ~/.oh-my-zsh/custom/plugins/
\end{lstlisting}
\tcblower
\qrsmall{ls ~/.oh-my-zsh/custom/plugins/}
\end{codeqr}

\section{Success!}

\begin{tcolorbox}[colback=green!5, colframe=green!40, title=🎉 Congratulations!]
You have successfully:
\begin{itemize}
    \item ✅ Installed Zsh shell
    \item ✅ Installed Oh My Zsh framework  
    \item ✅ Changed default shell (including Codespaces methods)
    \item ✅ Configured themes and plugins
    \item ✅ Applied the configuration
\end{itemize}

\textbf{Next Steps:}
\begin{itemize}
    \item Explore themes: \url{https://github.com/ohmyzsh/ohmyzsh/wiki/Themes}
    \item Discover plugins: \url{https://github.com/ohmyzsh/ohmyzsh/wiki/Plugins}
    \item Customize your prompt further
\end{itemize}
\end{tcolorbox}

\section{QR Code Benefits}

\begin{tcolorbox}[colback=blue!5, colframe=blue!40, title=📱 Why QR Codes Work Best for PDF]
\textbf{Advantages:}
\begin{itemize}
    \item ✅ Works with ANY PDF viewer (Adobe, Chrome, mobile apps)
    \item ✅ No JavaScript or security restrictions
    \item ✅ Mobile-friendly - use phone camera
    \item ✅ Always copies the correct single-line command
    \item ✅ Works offline once PDF is downloaded
    \item ✅ Cross-platform compatibility
\end{itemize}

\textbf{User Experience:}
\begin{enumerate}
    \item Read the guide on computer/tablet
    \item Scan QR code with phone
    \item Copy command from phone
    \item Paste into terminal on computer
\end{enumerate}

This bridges the gap between visual display and functional copying in PDF format!
\end{tcolorbox}

\end{document}