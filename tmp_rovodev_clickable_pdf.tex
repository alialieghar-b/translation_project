\documentclass{article}
\usepackage{fontspec}
\usepackage{xcolor}
\usepackage{geometry}
\usepackage{listings}
\usepackage{tcolorbox}
\usepackage{hyperref}
\usepackage{insdljs}

% Page setup
\geometry{a4paper, margin=1in}

% Configure hyperref for JavaScript
\hypersetup{
    colorlinks=true,
    linkcolor=blue,
    urlcolor=blue,
    pdfstartview=FitH
}

% Configure listings
\lstset{
    basicstyle=\ttfamily\small,
    breaklines=true,
    breakatwhitespace=true,
    breakindent=20pt,
    frame=single,
    backgroundcolor=\color{gray!10}
}

% JavaScript for copying
\begin{insDLJS}{copycode}{Copy Code Functions}
function copyToClipboard(text) {
    // Try to copy to clipboard
    try {
        // This is limited in PDF viewers
        this.print({
            bUI: false,
            bSilent: true,
            bShrinkToFit: true
        });
        app.alert("Command copied: " + text);
    } catch(e) {
        app.alert("Copy function: " + text);
    }
}
\end{insDLJS}

% Custom clickable code box
\newtcolorbox{clickablecode}[2]{
    enhanced,
    colback=gray!10,
    colframe=blue!50,
    title=Click to copy: #1,
    fonttitle=\bfseries\small,
    attach boxed title to top left={yshift=-2mm, xshift=2mm},
    boxed title style={colback=blue!20},
    overlay={
        \node[anchor=north east, text=blue] at (frame.north east) 
        {\small \href{javascript:copyToClipboard('#2')}{📋 Copy}};
    }
}

\newtcolorbox{infobox}{
    colback=blue!5,
    colframe=blue!40,
    title=Information,
    fonttitle=\bfseries
}

\newtcolorbox{successbox}{
    colback=green!5,
    colframe=green!40,
    title=Success,
    fonttitle=\bfseries
}

\title{Oh My Zsh Installation Guide\\
\large Clickable PDF with Copy Functionality}
\author{Interactive PDF Solution}
\date{\today}

\begin{document}

\maketitle

\section{How to Use This Guide}

\begin{infobox}
\textbf{Clickable Copy Instructions:}
\begin{enumerate}
    \item Click the "Copy" button or the code block itself
    \item The command will be copied to your clipboard
    \item Paste into your terminal
\end{enumerate}

\textbf{What you see vs what gets copied:}
\begin{itemize}
    \item \textbf{Display}: Clean, wrapped text for readability
    \item \textbf{Copy}: Original single-line command for functionality
\end{itemize}
\end{infobox}

\section{Step 1: Install Zsh}

\subsection{Ubuntu/Debian Systems}

\begin{clickablecode}{Update Package List}{sudo apt update}
\begin{lstlisting}
sudo apt update
\end{lstlisting}
\end{clickablecode}

\begin{clickablecode}{Install Zsh}{sudo apt install zsh}
\begin{lstlisting}
sudo apt install zsh
\end{lstlisting}
\end{clickablecode}

\subsection{CentOS/RHEL/Fedora Systems}

\begin{clickablecode}{Install Zsh (CentOS/RHEL)}{sudo yum install zsh}
\begin{lstlisting}
sudo yum install zsh
\end{lstlisting}
\end{clickablecode}

\begin{clickablecode}{Install Zsh (Fedora)}{sudo dnf install zsh}
\begin{lstlisting}
sudo dnf install zsh
\end{lstlisting}
\end{clickablecode}

\subsection{macOS Systems}

\begin{clickablecode}{Install Zsh (Homebrew)}{brew install zsh}
\begin{lstlisting}
brew install zsh
\end{lstlisting}
\end{clickablecode}

\subsection{Verify Installation}

\begin{clickablecode}{Check Zsh Version}{zsh --version}
\begin{lstlisting}
zsh --version
\end{lstlisting}
\end{clickablecode}

\begin{clickablecode}{Find Zsh Location}{which zsh}
\begin{lstlisting}
which zsh
\end{lstlisting}
\end{clickablecode}

\section{Step 2: Install Oh My Zsh}

\subsection{Method 1: Using curl}

\begin{clickablecode}{Install Oh My Zsh (curl)}{sh -c "$(curl -fsSL https://raw.githubusercontent.com/ohmyzsh/ohmyzsh/master/tools/install.sh)"}
\begin{lstlisting}
sh -c "$(curl -fsSL 
https://raw.githubusercontent.com/ohmyzsh/ohmyzsh/master/tools/install.sh)"
\end{lstlisting}
\end{clickablecode}

\subsection{Method 2: Using wget}

\begin{clickablecode}{Install Oh My Zsh (wget)}{sh -c "$(wget https://raw.githubusercontent.com/ohmyzsh/ohmyzsh/master/tools/install.sh -O -)"}
\begin{lstlisting}
sh -c "$(wget 
https://raw.githubusercontent.com/ohmyzsh/ohmyzsh/master/tools/install.sh -O -)"
\end{lstlisting}
\end{clickablecode}

\section{Step 3: Change Default Shell}

\subsection{Standard Systems}

\begin{clickablecode}{Change Shell (Standard)}{chsh -s $(which zsh)}
\begin{lstlisting}
chsh -s $(which zsh)
\end{lstlisting}
\end{clickablecode}

\subsection{GitHub Codespaces}

\begin{clickablecode}{Change Shell (Codespaces Method 1)}{sudo chsh -s $(which zsh) $(whoami)}
\begin{lstlisting}
sudo chsh -s $(which zsh) $(whoami)
\end{lstlisting}
\end{clickablecode}

\begin{clickablecode}{Change Shell (Codespaces Method 2)}{sudo chsh "$(id -un)" --shell "$(which zsh)"}
\begin{lstlisting}
sudo chsh "$(id -un)" --shell "$(which zsh)"
\end{lstlisting}
\end{clickablecode}

\section{Step 4: Install Popular Plugins}

\subsection{Zsh Autosuggestions}

\begin{clickablecode}{Install Autosuggestions Plugin}{git clone https://github.com/zsh-users/zsh-autosuggestions ${ZSH_CUSTOM:-~/.oh-my-zsh/custom}/plugins/zsh-autosuggestions}
\begin{lstlisting}
git clone https://github.com/zsh-users/zsh-autosuggestions 
${ZSH_CUSTOM:-~/.oh-my-zsh/custom}/plugins/zsh-autosuggestions
\end{lstlisting}
\end{clickablecode}

\subsection{Zsh Syntax Highlighting}

\begin{clickablecode}{Install Syntax Highlighting Plugin}{git clone https://github.com/zsh-users/zsh-syntax-highlighting.git ${ZSH_CUSTOM:-~/.oh-my-zsh/custom}/plugins/zsh-syntax-highlighting}
\begin{lstlisting}
git clone https://github.com/zsh-users/zsh-syntax-highlighting.git 
${ZSH_CUSTOM:-~/.oh-my-zsh/custom}/plugins/zsh-syntax-highlighting
\end{lstlisting}
\end{clickablecode}

\section{Step 5: Apply Changes}

\begin{clickablecode}{Reload Configuration}{source ~/.zshrc}
\begin{lstlisting}
source ~/.zshrc
\end{lstlisting}
\end{clickablecode}

\section{Verification Commands}

\begin{clickablecode}{Check Current Shell}{echo $SHELL}
\begin{lstlisting}
echo $SHELL
\end{lstlisting}
\end{clickablecode}

\begin{clickablecode}{Check User Shell Setting}{grep $(whoami) /etc/passwd}
\begin{lstlisting}
grep $(whoami) /etc/passwd
\end{lstlisting}
\end{clickablecode}

\section{Success!}

\begin{successbox}
You have successfully:
\begin{itemize}
    \item Installed Zsh shell
    \item Installed Oh My Zsh framework  
    \item Changed default shell (including Codespaces methods)
    \item Configured themes and plugins
    \item Applied the configuration
\end{itemize}

\textbf{Next Steps:}
\begin{itemize}
    \item Explore themes: \url{https://github.com/ohmyzsh/ohmyzsh/wiki/Themes}
    \item Discover plugins: \url{https://github.com/ohmyzsh/ohmyzsh/wiki/Plugins}
    \item Customize your prompt further
\end{itemize}
\end{successbox}

\section{How the Clickable Copy Works}

\begin{infobox}
\textbf{Visual vs Functional:}
\begin{itemize}
    \item \textbf{What you see}: Clean, automatically wrapped display for readability
    \item \textbf{What gets copied}: Original single-line command for functionality
    \item \textbf{Best of both worlds}: Beautiful presentation + working commands
\end{itemize}

\textbf{PDF Limitations:}
\begin{itemize}
    \item JavaScript support varies by PDF viewer
    \item Some viewers may show alerts instead of copying
    \item Mobile PDF viewers have limited JavaScript support
    \item For best experience, use the HTML version
\end{itemize}
\end{infobox}

\end{document>