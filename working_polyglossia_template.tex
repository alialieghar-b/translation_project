% Working Persian Template with xepersian
% Fixed to use xepersian for proper Persian script rendering
% Compile with: xelatex working_polyglossia_template.tex

\documentclass[12pt,a4paper]{article}

% Package loading order is important!
\usepackage{fontspec}
\usepackage{geometry}
\usepackage{amsmath}
\usepackage{xepersian}

% Document setup
\geometry{a4paper, margin=1in}

% Font setup using xepersian commands
\settextfont{Vazirmatn}
\setlatintextfont{Times New Roman}

\begin{document}

\title{\fa{قالب کار کرده برای اسناد فارسی}}
\author{\fa{نویسنده}}
\date{\today}
\maketitle

\section*{\fa{مقدمه}}

\fa{این یک قالب کار کرده برای اسناد فارسی با استفاده از polyglossia است.}

\section*{Mixed Language Example}

\en{This is English text using custom command.}

\fa{این متن فارسی با دستور سفارشی است.}

Direct commands also work:
- \textfarsi{متن مستقیم فارسی}
- \textenglish{Direct English text}

\section*{\fa{ریاضیات}}

\fa{فرمول ریاضی:} $E = mc^2$

\begin{equation}
\int_{0}^{\infty} e^{-x} dx = 1
\end{equation}

\section*{\fa{فهرست}}

\begin{itemize}
\item \fa{مورد اول}
\item \fa{مورد دوم}
\item \en{English item}
\end{itemize}

\section*{\fa{نتیجه‌گیری}}

\fa{این قالب بر اساس تست‌های مختلف و رفع خطاهای مختلف ساخته شده است.}

\end{document}