\documentclass{article}
\usepackage{fontspec}
\usepackage{xcolor}
\usepackage{geometry}
\usepackage{hyperref}
\usepackage{tcolorbox}
\usepackage{enumitem}

% Page setup
\geometry{a4paper, margin=1in}

% Configure hyperref
\hypersetup{
    colorlinks=true,
    linkcolor=blue,
    urlcolor=blue,
    pdfstartview=FitH
}

% Simple info boxes (no enhanced features)
\newtcolorbox{infobox}{
    colback=blue!5,
    colframe=blue!40,
    title=Information,
    fonttitle=\bfseries
}

\newtcolorbox{warningbox}{
    colback=orange!5,
    colframe=orange!40,
    title=Warning,
    fonttitle=\bfseries
}

\newtcolorbox{successbox}{
    colback=green!5,
    colframe=green!40,
    title=Success,
    fonttitle=\bfseries
}

% Simple command display (lessons learned: keep it simple)
\newcommand{\showcmd}[2]{%
    \begin{center}
    \fbox{\begin{minipage}{0.95\textwidth}
        \textbf{#1}\\[0.3em]
        \texttt{#2}
    \end{minipage}}
    \end{center}
    \vspace{0.3em}
}

\title{Complete Guide: From Bash to Oh My Zsh\\
\large Including GitHub Codespaces Configuration}
\author{Terminal Enhancement Guide}
\date{\today}

\begin{document}

\maketitle

\tableofcontents
\newpage

\section{Introduction}

This guide provides step-by-step instructions for switching from the default bash shell to Zsh with the Oh My Zsh framework. We'll cover installation, configuration, and specific instructions for GitHub Codespaces environments.

\begin{infobox}
\textbf{What is Oh My Zsh?}\\
Oh My Zsh is an open-source framework for managing Zsh configuration that provides:
\begin{itemize}
    \item Beautiful themes and prompts
    \item Hundreds of helpful plugins
    \item Auto-completion enhancements
    \item Git integration
    \item Customizable aliases and functions
\end{itemize}
\end{infobox}

\section{Prerequisites}

Before starting, ensure you have:

\begin{enumerate}
    \item Terminal access (bash, zsh, or similar)
    \item Internet connection for downloading packages
    \item Basic command-line knowledge
    \item Administrative privileges (sudo access)
\end{enumerate}

\begin{warningbox}
\textbf{Backup Your Current Configuration}\\
Before making changes, backup your existing shell configuration:
\end{warningbox}

\showcmd{Backup .bashrc}{cp ~/.bashrc ~/.bashrc.backup}

\showcmd{Backup .bash\_profile}{cp \textasciitilde/.bash\_profile \textasciitilde/.bash\_profile.backup}

\section{Step 1: Install Zsh}

\subsection{Ubuntu/Debian Systems}

\showcmd{Update package list}{sudo apt update}

\showcmd{Install Zsh}{sudo apt install zsh}

\subsection{CentOS/RHEL/Fedora Systems}

\showcmd{Install Zsh (CentOS/RHEL)}{sudo yum install zsh}

\showcmd{Install Zsh (Fedora)}{sudo dnf install zsh}

\subsection{macOS Systems}

\showcmd{Install Zsh with Homebrew}{brew install zsh}

\showcmd{Install Zsh with MacPorts}{sudo port install zsh}

\subsection{Verify Zsh Installation}

\showcmd{Check Zsh version}{zsh --version}

\showcmd{Find Zsh location}{which zsh}

Expected output should show Zsh version 5.0 or higher.

\section{Step 2: Install Oh My Zsh}

\subsection{Method 1: Using curl}

\begin{center}
\fbox{\begin{minipage}{0.95\textwidth}
\textbf{Install Oh My Zsh with curl}\\[0.3em]
\texttt{sh -c "\$(curl -fsSL \textbackslash\\
\phantom{sh -c "} https://raw.githubusercontent.com/ohmyzsh/ohmyzsh/master/tools/install.sh)"}
\end{minipage}}
\end{center}

\subsection{Method 2: Using wget}

\begin{center}
\fbox{\begin{minipage}{0.95\textwidth}
\textbf{Install Oh My Zsh with wget}\\[0.3em]
\texttt{sh -c "\$(wget \textbackslash\\
\phantom{sh -c "} https://raw.githubusercontent.com/ohmyzsh/ohmyzsh/master/tools/install.sh -O -)"}
\end{minipage}}
\end{center}

\subsection{Method 3: Manual Installation}

\showcmd{Clone Oh My Zsh repository}{git clone https://github.com/ohmyzsh/ohmyzsh.git ~/.oh-my-zsh}

\showcmd{Copy template configuration}{cp ~/.oh-my-zsh/templates/zshrc.zsh-template ~/.zshrc}

\section{Step 3: Change Default Shell to Zsh}

\subsection{Standard Linux/Unix Systems}

\showcmd{Change shell (standard method)}{chsh -s \$(which zsh)}

\subsection{GitHub Codespaces - Method 1}

For GitHub Codespaces environments, use this command:

\showcmd{Change shell (Codespaces method 1)}{sudo chsh -s \$(which zsh) \$(whoami)}

\subsection{GitHub Codespaces - Method 2 (Alternative)}

If the first method doesn't work, try:

\showcmd{Change shell (Codespaces method 2)}{sudo chsh "\$(id -un)" --shell "\$(which zsh)"}

\begin{infobox}
\textbf{Why Different Commands for Codespaces?}\\
GitHub Codespaces runs in a containerized environment where:
\begin{itemize}
    \item Standard \texttt{chsh} might not have sufficient permissions
    \item \texttt{sudo} is required for shell changes
    \item User identification needs to be explicit
    \item Container restrictions may require alternative syntax
\end{itemize}
\end{infobox}

\subsection{Verify Shell Change}

\showcmd{Check current shell}{echo \$SHELL}

\showcmd{Check user shell setting}{grep \$(whoami) /etc/passwd}

The output should show the path to zsh (e.g., \texttt{/usr/bin/zsh}).

\section{Step 4: Restart Terminal}

After changing the shell, you need to restart your terminal session:

\begin{enumerate}
    \item Close your current terminal
    \item Open a new terminal session
    \item Or logout and login again
    \item For Codespaces: Restart the container or reload the window
\end{enumerate}

\begin{successbox}
If successful, you should see the Oh My Zsh welcome message and a new colorful prompt!
\end{successbox}

\section{Step 5: Basic Configuration}

\subsection{Edit Zsh Configuration}

\showcmd{Edit .zshrc with nano}{nano ~/.zshrc}

\showcmd{Edit .zshrc with vim}{vim ~/.zshrc}

\subsection{Key Configuration Options}

Here's a basic \texttt{.zshrc} configuration:

\begin{verbatim}
# Oh My Zsh installation path
export ZSH="$HOME/.oh-my-zsh"

# Theme selection
ZSH_THEME="robbyrussell"  # Default theme
# ZSH_THEME="agnoster"    # Popular alternative
# ZSH_THEME="powerlevel10k/powerlevel10k"  # Advanced theme

# Plugins
plugins=(
    git
    zsh-autosuggestions
    zsh-syntax-highlighting
    docker
    kubectl
    node
    npm
)

# Source Oh My Zsh
source $ZSH/oh-my-zsh.sh

# Custom aliases
alias ll="ls -la"
alias la="ls -A"
alias l="ls -CF"
alias ..="cd .."
alias ...="cd ../.."

# Custom functions
mkcd() {
    mkdir -p "$1" && cd "$1"
}
\end{verbatim}

\section{Step 6: Install Popular Plugins}

\subsection{Zsh Autosuggestions}

\begin{verbatim}
git clone https://github.com/zsh-users/zsh-autosuggestions \
  ${ZSH_CUSTOM:-~/.oh-my-zsh/custom}/plugins/zsh-autosuggestions
\end{verbatim}

\subsection{Zsh Syntax Highlighting}

\begin{verbatim}
git clone https://github.com/zsh-users/zsh-syntax-highlighting.git \
  ${ZSH_CUSTOM:-~/.oh-my-zsh/custom}/plugins/zsh-syntax-highlighting
\end{verbatim}

\subsection{Powerlevel10k Theme (Optional)}

\begin{verbatim}
git clone --depth=1 https://github.com/romkatv/powerlevel10k.git \
  ${ZSH_CUSTOM:-$HOME/.oh-my-zsh/custom}/themes/powerlevel10k
\end{verbatim}

Then set \texttt{ZSH\_THEME="powerlevel10k/powerlevel10k"} in your \texttt{.zshrc}.

\section{Step 7: Apply Changes}

After making configuration changes:

\showcmd{Reload configuration}{source ~/.zshrc}

\section{Troubleshooting}

\subsection{Common Issues and Solutions}

\begin{enumerate}
    \item \textbf{Permission Denied}
    
    \showcmd{Fix permissions}{sudo chsh -s \$(which zsh) \$USER}
    
    \item \textbf{Zsh Not Found}
    
    \showcmd{Check Zsh installation}{which zsh}
    
    If empty, reinstall zsh:
    
    \showcmd{Reinstall Zsh (Ubuntu/Debian)}{sudo apt install zsh}
    
    \item \textbf{Oh My Zsh Installation Failed}
    
    Check internet connection and try manual installation:
    
    \showcmd{Test connection}{curl -fsSL https://raw.githubusercontent.com/ohmyzsh/ohmyzsh/master/tools/install.sh}
    
    \item \textbf{Plugins Not Working}
    
    \showcmd{Check plugin directory}{ls ~/.oh-my-zsh/custom/plugins/}
    
    Restart terminal after adding plugins.
\end{enumerate}

\subsection{Codespaces-Specific Issues}

\begin{warningbox}
\textbf{Container Persistence}\\
In GitHub Codespaces, remember that:
\begin{itemize}
    \item Shell changes persist across container restarts
    \item Custom configurations in \texttt{~/.zshrc} are preserved
    \item Installed plugins remain available
    \item Theme preferences are maintained
\end{itemize}
\end{warningbox}

\section{Advanced Configuration}

\subsection{Custom Prompt Configuration}

\begin{verbatim}
# Custom prompt with git status
PROMPT='%F{cyan}%n@%m%f:%F{yellow}%~%f$(git_prompt_info) %# '
ZSH_THEME_GIT_PROMPT_PREFIX=" %F{red}("
ZSH_THEME_GIT_PROMPT_SUFFIX=")%f"
ZSH_THEME_GIT_PROMPT_DIRTY="%F{yellow}*%f"
ZSH_THEME_GIT_PROMPT_CLEAN=""
\end{verbatim}

\subsection{Environment Variables}

\begin{verbatim}
# Development environment
export EDITOR="code"
export BROWSER="google-chrome"
export TERM="xterm-256color"

# Path additions
export PATH="$HOME/.local/bin:$PATH"
export PATH="$HOME/bin:$PATH"

# Node.js
export NVM_DIR="$HOME/.nvm"
[ -s "$NVM_DIR/nvm.sh" ] && \. "$NVM_DIR/nvm.sh"
\end{verbatim}

\section{Useful Aliases and Functions}

\begin{verbatim}
# Git aliases
alias gs="git status"
alias ga="git add"
alias gc="git commit"
alias gp="git push"
alias gl="git log --oneline"

# Docker aliases
alias dps="docker ps"
alias dpa="docker ps -a"
alias di="docker images"
alias drm="docker rm"
alias drmi="docker rmi"

# System aliases
alias h="history"
alias j="jobs"
alias c="clear"
alias e="exit"

# Directory navigation
alias home="cd ~"
alias root="cd /"
alias dtop="cd ~/Desktop"
alias docs="cd ~/Documents"

# File operations
alias cp="cp -i"
alias mv="mv -i"
alias rm="rm -i"
alias mkdir="mkdir -p"

# Network
alias ping="ping -c 5"
alias ports="netstat -tulanp"
\end{verbatim}

\section{Verification and Testing}

\subsection{Test Your Setup}

\begin{enumerate}
    \item \textbf{Check Shell}
    
    \showcmd{Verify current shell}{echo \$SHELL}
    
    \showcmd{Check process}{ps -p \$\$}
    
    \item \textbf{Test Oh My Zsh}
    
    \showcmd{Check Oh My Zsh version}{omz version}
    
    \showcmd{List plugins}{omz plugin list}
    
    \item \textbf{Test Autocompletion}
    
    Type \texttt{git <TAB><TAB>} or \texttt{docker <TAB><TAB>} to test autocompletion.
    
    \item \textbf{Test Plugins}
    
    Type a command and see suggestions, use arrow keys to navigate history.
\end{enumerate}

\section{Conclusion}

You have successfully:

\begin{successbox}
\begin{itemize}
    \item Installed Zsh shell
    \item Installed Oh My Zsh framework
    \item Changed default shell (including Codespaces methods)
    \item Configured themes and plugins
    \item Set up useful aliases and functions
    \item Verified the installation
\end{itemize}
\end{successbox}

\subsection{Next Steps}

\begin{enumerate}
    \item Explore more themes: \url{https://github.com/ohmyzsh/ohmyzsh/wiki/Themes}
    \item Discover plugins: \url{https://github.com/ohmyzsh/ohmyzsh/wiki/Plugins}
    \item Customize your prompt further
    \item Share your configuration with team members
    \item Consider using dotfiles for configuration management
\end{enumerate}

\subsection{Resources}

\begin{itemize}
    \item Oh My Zsh GitHub: \url{https://github.com/ohmyzsh/ohmyzsh}
    \item Zsh Documentation: \url{https://zsh.sourceforge.io/Doc/}
    \item Awesome Zsh Plugins: \url{https://github.com/unixorn/awesome-zsh-plugins}
    \item Powerlevel10k: \url{https://github.com/romkatv/powerlevel10k}
\end{itemize}

\begin{infobox}
\textbf{Remember}\\
Your terminal is now more powerful and user-friendly. Take time to explore the features and customize it to your workflow. Happy coding!
\end{infobox}

\end{document}