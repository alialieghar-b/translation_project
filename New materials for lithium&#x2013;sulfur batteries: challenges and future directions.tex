% --- SECTION 1: Page 1 of Translation ---
% --- Translator: Transatlantic Linguistic Consultant
% --- Document: New materials for lithium-sulfur batteries
% --- Source Language: English
% --- Target Language: Persian
% --- Compression: None
% --- Notes: Added footnotes for key technical terms and acronyms for academic clarity.
% --------------------------------------------------------------------------

\documentclass[12pt,a4paper,twocolumn]{article} % Using two-column format to match original layout

% Essential packages for multilingual Persian-English documents
\usepackage[utf8]{inputenc}
\usepackage{polyglossia}
\usepackage{fontspec}
\usepackage{bidi}
\usepackage[margin=2cm]{geometry} % Adjusted margins for two-column layout
\usepackage{fancyhdr}
\usepackage{hyperref}
\usepackage{xcolor}
\usepackage{graphicx}
\usepackage{caption}
\usepackage{amsmath} % For math environments

% Language configuration
\setdefaultlanguage{farsi}
\setotherlanguage{english}

% Font configuration for optimal Persian-English rendering
\setmainfont{Times New Roman}
\newfontfamily\farsifont[Script=Arabic,Scale=1.1]{XB Niloofar}
\newfontfamily\farsifontbold[Script=Arabic,Scale=1.1,BoldFont={XB Niloofar Bold}]{XB Niloofar}

% Page layout and headers
\pagestyle{fancy}
\fancyhf{}
\fancyhead[L]{\persian{مقاله مروری ویژه}}
\fancyhead[R]{\persian{ChemComm}}
\fancyfoot[C]{\thepage}
\renewcommand{\headrulewidth}{0.4pt}
\renewcommand{\footrulewidth}{0.4pt}

% Hyperlink styling
\hypersetup{
    colorlinks=true,
    linkcolor=blue,
    filecolor=magenta,      
    urlcolor=cyan,
    citecolor=red,
    pdftitle={مواد نوین برای باتری‌های لیتیوم-گوگرد},
    pdfauthor={مترجم: مشاور زبان‌شناسی ترنس-آتلانتیک}
}

% Custom commands for bilingual content and styling
\newcommand{\persian}[1]{\textfarsi{#1}}
\newcommand{\english}[1]{\textenglish{#1}}
\newcommand{\transliteration}[1]{\textit{#1}}
\newcommand{\farsibold}[1]{{\farsifontbold{#1}}}

% Document metadata
\title{\farsibold{مواد نوین برای باتری‌های لیتیوم-گوگرد: چالش‌ها و چشم‌اندازهای آینده}}
\author{\persian{مونتری ساوانگپروک (Montree Sawangphruk)}\\
\small{\persian{ترجمه و تدوین: مشاور زبان‌شناسی و متخصص لاتک ترنس-آتلانتیک}}}
\date{\persian{برگردان: ۳۱ جولای ۲۰۲۵}}

\begin{document}

\maketitle
\thispagestyle{fancy} % Apply fancy style to the first page as well

\begin{abstract}
\farsibold{چکیده:} \persian{این مقاله مروری، پیشرفت‌های اخیر در زمینه باتری‌های لیتیوم-گوگرد (Li-S)\footnote{\english{Lithium-Sulfur (Li-S)}} را بررسی می‌کند؛ این باتری‌ها به عنوان نسل آینده دستگاه‌های ذخیره‌سازی انرژی، به دلیل چگالی انرژی نظری فوق‌العاده بالا (حدود ۲۵۰۰ وات‌ساعت بر کیلوگرم)، صرفه اقتصادی و مزایای زیست‌محیطی، بسیار امیدبخش هستند. با وجود پتانسیل بالا، تجاری‌سازی آن‌ها به دلیل چالش‌های کلیدی مانند اثر شاتل پلی‌سولفید، ماهیت عایق گوگرد، ناپایداری آند فلز لیتیوم و نگرانی‌های مربوط به ایمنی حرارتی، محدود باقی مانده است. این مقاله، دیدگاهی جامع و آینده‌نگر در مورد استراتژی‌های مواد نوظهور—با تمرکز بر مهندسی کاتد، الکترولیت و آند—برای غلبه بر این موانع ارائه می‌دهد. تأکید ویژه‌ای بر کامپوزیت‌های پیشرفته گوگرد-کربن، از جمله چارچوب‌های گرافنی سه‌بعدی، چارچوب‌های فلزی-آلی (MOFs)\footnote{\english{Metal-Organic Frameworks (MOFs)}}، چارچوب‌های آلی کووالانسی (COFs) و مواد مبتنی بر MXene شده است که بهبودهای چشمگیری در بهره‌برداری از گوگرد، سینتیک ردوکس و پایداری چرخه‌ای نشان داده‌اند. نوآوری‌ها در الکترولیت‌ها—به‌ویژه سیستم‌های حالت جامد و پلیمری ژلی—به دلیل نقششان در سرکوب انحلال پلی‌سولفید و افزایش ایمنی مورد بحث قرار می‌گیرند. این مقاله همچنین استراتژی‌های محافظت از آند فلز لیتیوم، مانند استفاده از لایه‌های SEI مصنوعی و داربست‌های لیتیومی سه‌بعدی و آلیاژسازی لیتیوم را بررسی می‌کند. در نهایت، مسائل حیاتی مرتبط با تولید در مقیاس بزرگ، ایمنی و مقیاس‌پذیری تجاری را مورد بحث قرار می‌دهد. با نوآوری مداوم در مواد چندکاره و طراحی الکترود، باتری‌های Li-S در موقعیت مناسبی برای تحول در ذخیره‌سازی انرژی برای وسایل نقلیه الکتریکی، دستگاه‌های الکترونیکی قابل حمل و سیستم‌های در مقیاس شبکه قرار دارند.}
\end{abstract}

\vspace{1cm}

\persian{
\farsibold{وابستگی سازمانی نویسنده:} \\
مرکز تعالی فناوری ذخیره‌سازی انرژی، گروه مهندسی شیمی و بیومولکولی، دانشکده علوم و مهندسی انرژی، مؤسسه علم و فناوری ویدیاسیریمدی (VISTEC)، رایونگ ۲۱۲۱۰، تایلند. \\
ایمیل: \english{montree.s@vistec.ac.th}
}
\vspace{1em}
\rule{\columnwidth}{0.4pt}
\vspace{1em}

\begin{figure}[h]
    \centering
    \includegraphics[width=0.4\columnwidth]{example-image-a} % Placeholder for author's photo
    \caption*{\persian{مونتری ساوانگپروک}}
\end{figure}

\persian{
\farsibold{زندگی‌نامه نویسنده:} \\
دانشیار پروفسور دکتر \farsibold{مونتری ساوانگپروک} مدیر مرکز تعالی فناوری ذخیره‌سازی انرژی (CEST) در مؤسسه VISTEC تایلند است. ایشان دکترای تخصصی (\english{DPhil}) خود را در رشته شیمی فیزیک و نظری در سال ۲۰۱۰ از دانشگاه آکسفورد، انگلستان، دریافت کردند. پژوهش‌های ایشان بر مواد پیشرفته برای سیستم‌های ذخیره‌سازی انرژی—به‌ویژه باتری‌ها و ابرخازن‌ها—با تأکید بر فناوری‌های پایدار و نوآورانه متمرکز است. دکتر ساوانگپروک بیش از ۱۸۰ مقاله در مجلات معتبر منتشر کرده و بیش از ۶۰ پرونده ثبت اختراع دارد. مشارکت‌های ایشان با جوایز معتبر متعددی از جمله جایزه ستاره نوظهور آسیا (۲۰۱۹)، جایزه دانشمند برجسته ملی (۲۰۱۹) و جایزه پژوهشگر برجسته ملی (۲۰۲۵) به رسمیت شناخته شده است.
}
\vspace{1cm}

\section*{\persian{۱. مقدمه}}
\subsection*{\persian{۱.۱. مروری بر باتری‌های \english{Li-S}}}

\persian{
تقاضای جهانی فزاینده برای ذخیره‌سازی انرژی با کارایی بالا، که ناشی از گذار به سوی انرژی‌های تجدیدپذیر و پذیرش گسترده وسایل نقلیه الکتریکی (EVs)\footnote{\english{Electric Vehicles (EVs)}} با فروش بیش از ۱۷ میلیون دستگاه در سال ۲۰۲۴ است، جستجو برای فناوری‌های باتری نسل بعد را تشدید کرده است. در میان نامزدهای نوظهور مختلف، باتری‌های لیتیوم-گوگرد (\english{Li-S}) به دلیل چگالی انرژی نظری فوق‌العاده بالا (حدود ۲۵۰۰ وات‌ساعت بر کیلوگرم)، صرفه اقتصادی و فراوانی طبیعی گوگرد، توجه قابل توجهی را به خود جلب کرده‌اند.$^{۱-۵}$ برخلاف باتری‌های لیتیوم-یون (\english{Li-ion}) متداول که به مواد کاتدی مبتنی بر بین‌نشینی\footnote{\english{Intercalation}} مانند اکسیدهای فلزات واسطه متکی هستند، باتری‌های \english{Li-S} از گوگرد عنصری به عنوان کاتد و فلز لیتیوم به عنوان آند استفاده می‌کنند. این پیکربندی مزایای قابل توجهی از نظر پتانسیل ذخیره‌سازی انرژی ارائه می‌دهد و نویدبخش یک تغییر تحول‌آفرین در فناوری باتری است.
}
\persian{
گوگرد ظرفیت ویژه نظری ۱۶۷۵ میلی‌آمپر ساعت بر گرم (\english{mAh g⁻¹}) را از خود نشان می‌دهد که بسیار فراتر از ظرفیت ۱۴۰ تا ۲۰۰ میلی‌آمپر ساعت بر گرم مواد کاتدی متداول مانند لیتیوم آهن فسفات (\english{LiFePO₄} یا \english{LFP}) و لیتیوم نیکل منگنز کبالت اکسید (\english{NMC}) است که در کاتدهای باتری‌های لیتیوم-یون استفاده می‌شوند. این ظرفیت قابل توجه، همراه با سبکی و قیمت مناسب گوگرد، باتری‌های \english{Li-S} را برای کاربردهایی از حمل‌ونقل الکتریکی گرفته تا ذخیره‌سازی انرژی در مقیاس شبکه، بسیار جذاب می‌سازد. علاوه بر این، لیتیوم...
}

% --- End of content for page 1 ---

% --- SECTION 2: Page 2 of Translation ---
% --- Notes: Figure 1 and its caption are included. Technical terms are footnoted.
% --------------------------------------------------------------------------

\persian{
فلز، با ظرفیت نظری ۳۸۶۰ میلی‌آمپر ساعت بر گرم، چگالی انرژی سلول‌های \english{Li-S} را بیشتر تقویت می‌کند. در نتیجه، فناوری \english{Li-S} پتانسیل عظیمی برای کاربردهایی که نیازمند راه‌حل‌های ذخیره‌سازی انرژی سبک و با ظرفیت بالا هستند، در خود دارد.
}
\persian{
با این حال، علی‌رغم پتانسیل بالایشان، تجاری‌سازی باتری‌های \english{Li-S} به دلیل چالش‌های بنیادین مرتبط با شیمی ردوکس پیچیده گوگرد، با موانعی روبرو بوده است. انحلال و مهاجرت پلی‌سولفیدهای لیتیوم میانی (\english{Li₂Sₙ} که در آن \english{4 ≤ n ≤ 8}) منجر به «اثر شاتل پلی‌سولفید»\footnote{\english{Polysulfide Shuttle Effect}} معروف می‌شود که باعث افت شدید ظرفیت، بازده کولمبی پایین و پایداری چرخه‌ای ضعیف (کمتر از ۳۰۰ چرخه برای پیکربندی سلول در سطح تجاری) می‌گردد. علاوه بر این، ماهیت عایق گوگرد حالت جامد (\english{S₈}) با رسانایی \english{5 × 10⁻³⁰ S cm⁻¹} در دمای ۲۵ درجه سانتی‌گراد و \english{Li₂S} با رسانایی حدود \english{10⁻¹³ S cm⁻¹} در ۲۵ درجه سانتی‌گراد، موانع سینتیکی قابل توجهی را ایجاد می‌کند و نیازمند رویکردهای نوآورانه برای افزایش رسانایی الکترونی و سینتیک واکنش است.
}
\persian{
شکل ۱ مکانیزم‌های بنیادین، مسیرهای واکنش و چالش‌های مرتبط با واکنش کاهش گوگرد (SRR)\footnote{\english{Sulfur Reduction Reaction (SRR)}} در باتری‌های \english{Li-S} را نشان می‌دهد. شکل ۱(الف) اثر شاتل پلی‌سولفید، یک مسئله عمده در باتری‌های \english{Li-S}، را به تصویر می‌کشد. در حین دشارژ، \english{S₈} به پلی‌سولفیدهای با زنجیره بلند (\english{Li₂Sₙ}) کاهش می‌یابد که در الکترولیت حل شده و به سمت آند لیتیوم مهاجرت می‌کنند. این امر منجر به واکنش‌های جانبی و اتلاف ظرفیت می‌شود. کنترل این اثر شاتل برای بهبود طول عمر باتری حیاتی است.$^۷$
}
\persian{
شکل ۱(ب) پروفایل ولتاژ \english{SRR} را نشان می‌دهد که در چهار مرحله رخ می‌دهد: (I) گوگرد به پلی‌سولفیدهای محلول کاهش می‌یابد؛ (II) این ترکیبات میانی کاهش بیشتری یافته و به پلی‌سولفیدهای با زنجیره کوتاه‌تر تبدیل می‌شوند؛ (III) پلی‌سولفیدهای فاز مایع به \english{Li₂S} نامحلول تبدیل می‌شوند؛ و (IV) رسوب نهایی \english{Li₂S} رخ می‌دهد. سینتیک کند مرحله نهایی اغلب عملکرد باتری را محدود می‌کند.$^۸$ شکل ۱(ج) مشخصه‌یابی پراش اشعه ایکس درجا (Operando XRD)\footnote{\english{Operando X-ray Diffraction (XRD)}} را ارائه می‌دهد که گذار فازها را در حین شارژ و دشارژ آشکار می‌سازد. ردیابی این تغییرات به درک تحول پلی‌سولفیدها و تشکیل \english{Li₂S} در فاز جامد کمک می‌کند که برای طراحی کاتالیزورها جهت بهبود سینتیک واکنش حیاتی است.$^{۹,۱۰}$ شکل ۱(د) مسیر واکنش و چشم‌انداز انرژی آزاد \english{SRR} را نشان می‌دهد.$^{۱۱}$ تبدیل گام به گام \english{S₈} به \english{Li₂S} از طریق پلی‌سولفیدهای میانی به همراه بهبود کاتالیزوری توسط موادی مانند \english{MoS₂} و گرافن نشان داده شده است. این کاتالیزورها موانع انرژی فعال‌سازی را کاهش داده، کاهش گوگرد را تسریع و بازده را بهبود می‌بخشند. در مجموع، این شکل‌ها جنبه‌های کلیدی \english{SRR} در باتری‌های \english{Li-S} را برجسته کرده و بینش‌هایی در مورد مکانیزم‌های واکنش، چالش‌ها و استراتژی‌ها برای بهبود عملکرد باتری از طریق طراحی کاتالیزور و تکنیک‌های سرکوب شاتل ارائه می‌دهند.
}

\begin{figure*}[t]
    \centering
    \includegraphics[width=0.9\textwidth]{example-image-b} % Placeholder for the composite Figure 1
    \caption{\persian{
    \farsibold{شکل ۱.} (الف) تصویر شماتیک از اثر شاتل پلی‌سولفید در باتری‌های \english{Li-S}. مهاجرت پلی‌سولفیدهای محلول با زنجیره بلند (\english{Li₂Sₙ}) از طریق جداکننده به آند لیتیوم منجر به واکنش‌های جانبی ناخواسته، افت ظرفیت و بازده کولمبی پایین می‌شود. بازتولید شده از مرجع ۸ با مجوز از \english{Wiley-VCH}، کپی‌رایت ۲۰۲۲. (ب) پروفایل ولتاژ \english{SRR} حین دشارژ در یک باتری \english{Li-S}. فرآیند به چهار مرحله تقسیم می‌شود: (I) تبدیل جامد-مایع گوگرد عنصری (\english{S₈}) به پلی‌سولفیدهای با زنجیره بلند (\english{Li₂S₄})؛ (II) کاهش بیشتر به پلی‌سولفیدهای کوتاه‌تر؛ (III) گذار مایع-جامد و تشکیل \english{Li₂S}؛ و (IV) رسوب نهایی \english{Li₂S}. بازتولید شده از مرجع ۸ با مجوز از \english{Wiley-VCH}، کپی‌رایت ۲۰۲۲. (ج) مشخصه‌یابی \english{in situ XRD} و \english{operando} از گونه‌های گوگردی در طی چرخه‌های شارژ و دشارژ. تحول فازی گونه‌های مختلف پلی‌سولفید، از جمله \english{Li₂S}، مشاهده می‌شود که بینشی در مورد تغییرات ساختاری کاتد ارائه می‌دهد. بازتولید شده از مراجع ۹ و ۱۰ با مجوز از انجمن شیمی آمریکا، کپی‌رایت ۲۰۲۲ و ۲۰۲۰. (د) نمودار انرژی آزاد و مسیر واکنش \english{SRR} در یک باتری \english{Li-S} که تبدیل \english{S₈} به \english{Li₂S} را از طریق پلی‌سولفیدهای میانی (\english{Li₂S₆, Li₂S₄, Li₂S₂}) برجسته می‌کند. اثر کاتالیزوری گرافن و \english{MoS₂} نشان داده شده و نقش آن‌ها در بهبود سینتیک \english{SRR} را نمایش می‌دهد. بازتولید شده از مرجع ۱۱ با مجوز از \english{Wiley-VCH}، کپی‌رایت ۲۰۲۲.
    }}
    \label{fig:1}
\end{figure*}


\subsection*{\persian{۱.۲. پیشرفت‌ها در طراحی سلول \english{Li-S}}}
\persian{
در دهه‌های گذشته، پژوهشگران افزودنی‌ها و ساختارهای کامپوزیتی مختلفی را برای بهبود عملکرد الکتروشیمیایی باتری‌های \english{Li-S} بررسی کرده‌اند. مواد عملکردی—شامل پلیمرها،$^{۱۲-۱۴}$ سرامیک‌ها و فلزات—به طور گسترده مورد تحقیق قرار گرفته‌اند. علی‌رغم این پیشرفت‌ها، اکثر کاتدهای \english{Li-S} همچنان به شدت به مواد کربنی رسانا برای جبران ماهیت عایق گوگرد و محصولات دشارژ آن متکی هستند. مطالعات پیشین نشان داده‌اند که واکنش‌های تبدیل گوگرد و همچنین گذارهای فازی جامد-مایع-جامد درگیر، به شدت تحت تأثیر پارامترهای ساخت سلول قرار دارند. این عوامل فراتر از ملاحظات مواد بنیادین بوده و مستقیماً بر مقیاس‌پذیری، هزینه‌های تولید و عملکرد الکتروشیمیایی تأثیر می‌گذارند. بنابراین، بهینه‌سازی همزمان خواص ذاتی مواد و پارامترهای ساخت خارجی برای دستیابی به باتری‌های \english{Li-S} کاربردی، حیاتی است.
}
\persian{
یک محدودیت کلیدی در کاتدهای \english{Li-S} ناشی از لزوم استفاده از کربن‌های متخلخل با سطح ویژه بالا برای بهبود بهره‌برداری از گوگرد است. اگرچه کاتدهای کامپوزیتی معمولاً حاوی ۶۰ تا ۸۰ درصد وزنی گوگرد هستند، اما محتوای واقعی ماده فعال پس از مخلوط شدن با بایندرها و افزودنی‌های رسانا به زیر این آستانه کاهش می‌یابد. علاوه بر این، هنگام پوشش‌دهی روی یک جمع‌کننده جریان استاندارد آلومینیومی، بارگذاری مؤثر گوگرد اغلب به سطوح لازم برای عملکرد با چگالی انرژی بالا (≥۶ میلی‌گرم بر سانتی‌متر مربع) نمی‌رسد.$^{۱۵}$ برای کاهش انحلال پلی‌سولفید، از کربن‌های متخلخل و افزودنی‌های عملکردی استفاده می‌شود، اما این مواد همچنین الکترولیت را جذب می‌کنند که منجر به نسبت بیش از حد الکترولیت به گوگرد (بیش از ۲۰ میکرولیتر بر میلی‌گرم) می‌شود. این محتوای بالای الکترولیت، چگالی انرژی را کاهش داده و تحلیل پلاریزاسیون و سینتیک واکنش را پیچیده می‌کند.$^{۱۵}$
}
\persian{
علاوه بر این، بسیاری از سلول‌های \english{Li-S} گزارش‌شده از بارگذاری‌های پایین گوگرد (۱-۲ میلی‌گرم بر سانتی‌متر مربع) استفاده می‌کنند که محدودیت‌های الکتروشیمیایی ناشی از ماهیت عایق گوگرد را به درستی منعکس نمی‌کند.
}

% --- End of content for page 2 ---

% --- SECTION 3: Page 3 of Translation ---
% --- Notes: Table 1 and Figure 2 with its caption are included.
% --------------------------------------------------------------------------

\persian{
طبیعت آن است.$^{۱۵}$ علاوه بر این، بیشتر پیکربندی‌های آزمایشگاهی بر فلز لیتیوم اضافی و حجم بالای الکترولیت متکی هستند، که این امر مصرف غیرقابل برگشت لیتیوم ناشی از واکنش‌های جانبی پلی‌سولفید و تشکیل دندریت لیتیوم را پنهان می‌کند. برای حل این مسائل، تلاش‌های پژوهشی کنونی بر بهینه‌سازی تکنیک‌های ساخت برای بهبود قابلیت اطمینان الکتروشیمیایی و امکان‌سنجی عملی متمرکز شده است. هدف اصلی، به حداقل رساندن استفاده از مواد غیرفعال، کاهش حجم اضافی الکترولیت و دستیابی به نسبت متعادل‌تری بین آند و کاتد است تا در نهایت چگالی انرژی کلی سیستم افزایش یابد. مطالعات اخیر نشان می‌دهد که یک باتری \english{Li-S} برای دستیابی به چگالی انرژی \english{400 Wh kg⁻¹} نیازمند بارگذاری گوگرد تقریباً \english{5 mg cm⁻²}، محتوای گوگرد ۶۰ تا ۸۰ درصد وزنی و نسبت الکترولیت به گوگرد \english{4.0 µL mg⁻¹} است. برای رساندن چگالی انرژی به فراتر از \english{500 Wh kg⁻¹}، بارگذاری گوگرد باید از \english{6 mg cm⁻²} فراتر رود و نسبت الکترولیت به گوگرد نیز به \english{≤3:1 µL mg⁻¹} کاهش یابد (جدول ۱).$^{۱۵}$
}

\begin{table*}[t]
    \centering
    \caption{\persian{\farsibold{جدول ۱.} خلاصه‌ای از وضعیت سلول‌های لیتیوم-گوگرد: شکاف بین وضعیت فعلی و الزامات برای سلول‌های باتری با انرژی بالا.$^{۱۵}$}}
    \label{tab:summary}
    \begin{tabular}{rllll}
        \toprule
        \farsibold{الزامات مورد نیاز} & \farsibold{چالش‌ها} & \farsibold{وضعیت فعلی} & \farsibold{جنبه} \\
        \midrule
        \persian{ثقلی: \english{400-600 Wh kg⁻¹}} & \persian{چگالی انرژی حجمی کمتر از باتری‌های لیتیوم-یون} & \persian{ثقلی: حدود \english{350 Wh kg⁻¹} (نمونه اولیه)} & \farsibold{چگالی انرژی} \\
        \persian{حجمی: \english{>600 Wh L⁻¹}} & \persian{(حدود \english{700 Wh L⁻¹})} & \persian{حجمی: حدود \english{325 Wh L⁻¹}} & \\
        \hline
        \persian{بهبود پایداری آند لیتیوم، کاهش} & \persian{تعداد چرخه‌های بالا تنها در شرایط ایده‌آل آزمایشگاهی} & \persian{اعداد بالا تنها در شرایط آزمایشگاهی} & \farsibold{عمر چرخه‌ای} \\
        \persian{تخلیه الکترولیت} & \persian{نیازمند لیتیوم اضافی، نسبت \english{E/S} بالا و بارگذاری کم گوگرد} & \persian{بیشتر مطالعات: \english{>7:1}} & \\
        \hline
        \persian{هدف \english{E/S ≤ 3:1}} & \persian{نسبت \english{E/S} بالا چگالی انرژی عملی را محدود می‌کند} & \persian{بیشتر مطالعات: \english{>7:1}} & \farsibold{نسبت الکترولیت به گوگرد (E/S)} \\
        \hline
        \persian{\english{≥6 mg cm⁻²}} & \persian{بسیار کم برای چگالی انرژی بالا} & \persian{\english{<2 mg cm⁻²} در ~۶۰٪ مطالعات} & \farsibold{بارگذاری گوگرد} \\
        \hline
        \persian{≥۸۰٪} & \persian{محدود شده توسط انحلال پلی‌سولفید و سینتیک واکنش} & \persian{حدود ۵۰-۷۰٪} & \farsibold{بهره‌برداری از گوگرد} \\
        \hline
        \persian{≥۷۰٪} & \persian{کمتر از باتری‌های لیتیوم-یون (حدود ۹۰٪)} & \persian{۳۰-۷۰٪ گوگرد در کاتد} & \farsibold{کسر ماده فعال} \\
        \hline
        \persian{استفاده از الکترولیت‌های حالت جامد،} & \persian{ریسک‌های ایمنی، عمر چرخه‌ای کوتاه} & \persian{مستعد رشد دندریت، تخلیه الکترولیت} & \farsibold{آند فلز لیتیوم} \\
        \persian{آندهای هیبریدی لیتیوم-سیلیکون} & & & \\
        \hline
        \persian{هزینه بالقوه \english{Li-S}: حدود \english{70-130} دلار بر کیلووات‌ساعت} & \persian{هزینه‌های بالای فرآیند برای \english{Li-S}} & \persian{حدود ۲۰۰ دلار بر کیلووات‌ساعت (مشابه لیتیوم-یون فعلی)} & \farsibold{تخمین هزینه} \\
        \hline
        \persian{بهبود پایداری برای خودروهای الکتریکی و بازار انبوه} & \persian{عمر چرخه‌ای کوتاه پذیرش در خودروهای الکتریکی را محدود می‌کند} & \persian{کاربردهای خاص (پهپادها، نظامی)} & \farsibold{کاربردها} \\
        \hline
        \persian{نیاز به پیشرفت‌های بزرگ در مدیریت الکترولیت} & \persian{شکاف بزرگ بین شرایط آزمایشگاهی و تجاری} & \persian{محدود} & \farsibold{قابلیت تجاری‌سازی} \\
        \persian{و بهینه‌سازی الکترود} & & & \\
        \bottomrule
    \end{tabular}
\end{table*}

\subsection*{\persian{۱.۳. چالش‌های مقیاس‌بندی باتری‌های \english{Li-S} به سلول‌های با فرمت بزرگ}}
\persian{
علی‌رغم چگالی انرژی نظری امیدوارکننده و صرفه اقتصادی، باتری‌های \english{Li-S} هنگام گذار از سلول‌های سکه‌ای آزمایشگاهی به سلول‌های استوانه‌ای و کیسه‌ای (پاوچ) با فرمت بزرگ با چالش‌های قابل توجهی روبرو هستند. این چالش‌ها عمدتاً از محدودیت‌های بنیادین مواد، فرآیندهای الکتروشیمیایی پیچیده و نگرانی‌های ایمنی ناشی می‌شوند که همگی بر قابلیت تجاری‌سازی آن‌ها تأثیر می‌گذارند. این بخش موانع کلیدی در مقیاس‌بندی باتری‌های \english{Li-S}، از جمله مکانیزم‌های تخریب ظرفیت، اثرات شاتل پلی‌سولفید، ناپایداری الکترود و خطرات فرار حرارتی\footnote{\english{Thermal Runaway}} را بررسی می‌کند.
}
\subsubsection*{\persian{۱.۳.۱. تخریب ظرفیت و پایداری الکتروشیمیایی}}
\persian{
یک مانع بزرگ برای پذیرش گسترده باتری‌های \english{Li-S}، افت سریع ظرفیت آن‌هاست که عمدتاً به دلیل اثر شاتل پلی‌سولفید، خوردگی الکترود و تجزیه الکترولیت رخ می‌دهد. واکنش‌های تبدیلی در سلول‌های \english{Li-S} شامل انحلال پلی‌سولفیدهای لیتیوم میانی (\english{Li₂S₈ → Li₂S₆ → ... → Li₂S}) است که در طول چرخه‌ها بین الکترودها مهاجرت می‌کنند. این امر منجر به اتلاف مداوم ماده فعال، خود-دشارژی و بازده کولمبی ضعیف می‌شود.
}
\persian{
تحقیقات اخیر بر روی سلول‌های استوانه‌ای \english{18650 Li-S} با بارگذاری بالای گوگرد (\english{5 mg cm⁻²})، همانطور که در شکل ۲ نشان داده شده، چالش‌های بنیادین پیش روی تجاری‌سازی باتری‌های \english{Li-S} با فرمت بزرگ را برجسته می‌کند.$^{۱۶}$ علی‌رغم بهینه‌سازی ساختار کاتد با استفاده از سیستم‌های بایندر مختلف مانند کربوکسی‌متیل سلولز (\english{CMC}) و لاستیک استایرن-بوتادین...
}

\begin{figure*}[t]
    \centering
    \includegraphics[width=0.85\textwidth]{example-image-c} % Placeholder for Figure 2
    \caption{\persian{
    \farsibold{شکل ۲.} باتری‌های استوانه‌ای \english{Li-S} در مقیاس بزرگ. بازتولید شده از مرجع ۱۶ با مجوز از انجمن سلطنتی شیمی، کپی‌رایت ۲۰۲۳. (الف) تصویر شماتیک از یک باتری استوانه‌ای \english{18650 Li-S} با بارگذاری بالای گوگرد \english{5 mg cm⁻²} که مسیرهای واکنش ردوکس گوگرد را در طول چرخه‌ها نشان می‌دهد. مکانیسم اصلی خرابی در این باتری‌ها شامل افت شدید ظرفیت به دلیل خوردگی الکترود، تجزیه الکترولیت و اثر شاتل پلی‌سولفید است. (ب) شماتیکی از فرآیندهای ساخت الکترود و سلول، شامل اختلاط در خلاء، پوشش‌دهی رول-به-رول، کلندرینگ و مراحل مونتاژ سلول مانند پیچیدن، شیارزنی، تزریق الکترولیت و کریمپینگ که در یک گلاوباکس پر از آرگون انجام می‌شود. (ج) پروفایل‌های گالوانواستاتیک شارژ-دشارژ (\english{GCD}) سلول \english{18650 Li-S} در \english{0.05C} که نوسانات ولتاژ و تخریب تدریجی ظرفیت را در طی چرخه‌های متعدد نشان می‌دهد. (د) عکس‌هایی از باتری ژله‌ای-رول در حالت استراحت و پس از چرخه‌های اول، شصتم و نودم شارژ-دشارژ که تجمع پلی‌سولفید در الکترولیت را برجسته می‌کند. (ه) شواهد بصری از انحلال پلی‌سولفید و تخریب الکترود پس از چرخه‌های طولانی، از جمله تغییر رنگ الکترولیت. (و) مقایسه تصویری فلز لیتیوم قبل و بعد از چرخه‌زنی. (ز) و (ح) تصاویر میکروسکوپ نوری از فلز لیتیوم (ز) قبل از استفاده و (ح) پس از چرخه‌زنی که نشان‌دهنده تخریب سطح و تغییرات مورفولوژیکی ناشی از چرخه‌زنی مداوم است.
    }}
    \label{fig:2}
\end{figure*}


% --- End of content for page 3 ---

% --- SECTION 4: Page 4 of Translation ---
% --- Notes: Continues discussion of large-format cell challenges and introduces cathode materials.
% --------------------------------------------------------------------------

\persian{
(\english{SBR})، افت سریع ظرفیت تنها پس از چند چرخه شارژ-دشارژ مشاهده شد. طیف‌سنجی فرابنفش-مرئی (\english{UV-vis}) تجمع قابل توجه پلی‌سولفید در الکترولیت را تأیید کرد، که نشان‌دهنده وقوع شاتل کنترل‌نشده پلی‌سولفید است. تحلیل‌های پس از آزمون (\english{Post-mortem}) خوردگی شدید کاتد و اتلاف موضعی ماده فعال را آشکار ساخت، به ویژه در نواحی مرکزی ژله‌ای-رول که نفوذ الکترولیت در آنجا بارزتر بود. همزمان، آند فلز لیتیوم تخریب سطحی قابل توجهی، از جمله رشد دندریتی و خوردگی، نشان داد که به ناپایداری و خرابی سلول کمک می‌کرد. علاوه بر این، تجزیه اجزای الکترولیت منجر به تولید محصولات جانبی گازی (مانند \english{CO}, \english{NO} و \english{CO₂}) شد،$^{۱۶}$ که باعث افزایش فشار داخلی و تسریع بیشتر مکانیزم‌های تخریب گردید.
}
\persian{
شکل ۲ یک نمای کلی جامع از توسعه، ساخت و مسیرهای خرابی باتری‌های استوانه‌ای \english{Li-S} در مقیاس بزرگ ارائه می‌دهد. شکل ۲(الف) واکنش‌های ردوکس بنیادین گوگرد را درون یک باتری \english{18650 Li-S} به تصویر می‌کشد. شکل ۲(ب) فرآیندهای آماده‌سازی الکترود و مونتاژ سلول، از جمله اختلاط، پوشش‌دهی، کلندرینگ و ساخت ژله‌ای-رول تحت شرایط بی‌اثر را تشریح می‌کند. عملکرد الکتروشیمیایی، همانطور که در شکل ۲(ج) نشان داده شده است، پلاتوهای دشارژ متمایزی مرتبط با گونه‌های گوگرد را نشان می‌دهد، اما همچنین پلاریزاسیون فزاینده و افت ظرفیت را نیز نمایش می‌دهد. شکل‌های ۲(د) و (ه) انحلال پلی‌سولفید و تخریب الکترولیت را در طول چرخه‌زنی طولانی‌مدت برجسته می‌کنند. در نهایت، شکل‌های ۲(و-ح) شواهد تصویری و میکروسکوپی از تخریب فلز لیتیوم را ارائه می‌دهند و بر موانع کلیدی مانند تشکیل دندریت و واکنش‌های پارازیتی تأکید می‌کنند. این یافته‌ها در مجموع بر نیاز به استراتژی‌های مواد پیشرفته، طراحی‌های سلولی مستحکم و رویکردهای مؤثر برای به دام انداختن پلی‌سولفیدها برای دستیابی به باتری‌های \english{Li-S} کاربردی و بادوام تأکید می‌کنند.
}
\subsubsection*{\persian{۱.۳.۲. چالش‌های ساخت الکترود و سلول}}
\persian{
گذار از سلول‌های سکه‌ای \english{Li-S} در مقیاس کوچک به سلول‌های استوانه‌ای و کیسه‌ای بزرگتر، پیچیدگی‌های تولیدی بیشتری را به همراه دارد. یک سلول \english{18650 Li-S} معمولی نیازمند کنترل دقیق بر ساخت الکترود و تزریق الکترولیت برای به حداقل رساندن ناهمگونی‌ها و ناهماهنگی‌های عملکردی است.$^{۱۶}$ فرآیند ساخت شامل اختلاط در خلاء دوغاب کاتد، پوشش‌دهی رول-به-رول و کلندرینگ برای تشکیل فیلم‌های الکترود یکنواخت است. سپس الکترودها از طریق پیچیدن، شیارزنی و تزریق الکترولیت در یک محیط پر از آرگون بی‌اثر به سلول‌های استوانه‌ای مونتاژ شده و سپس با کریمپینگ مهر و موم می‌شوند. آماده‌سازی نامناسب الکترود می‌تواند منجر به توزیع ناهموار گوگرد، لایه‌لایه شدن و انحلال بیش از حد پلی‌سولفید شود که همگی به طور قابل توجهی بر عمر چرخه‌ای تأثیر می‌گذارند.$^{۱۶}$ علاوه بر این، بارگذاری بالای گوگرد در سلول‌های استوانه‌ای ماهیت عایق \english{S₈} و \english{Li₂S} را تشدید می‌کند و منجر به اورپتانسیل‌های بالا و بهره‌برداری ضعیف از ماده فعال می‌شود. آزمون \english{GCD} سلول‌های \english{Li-S} با فرمت بزرگ در \english{0.05C} ناپایداری ولتاژ پیشرونده را نشان می‌دهد که تخلیه الکترولیت و پلاریزاسیون شدید واکنش را برجسته می‌کند. پرداختن به این مسائل نیازمند بهینه‌سازی تخلخل کاتد، الاستیسیته بایندر و ترکیب الکترولیت برای افزایش رسانایی یونی و الکترونی و در عین حال به حداقل رساندن واکنش‌های جانبی است.
}

\subsubsection*{\persian{۱.۳.۳. فرار حرارتی و نگرانی‌های ایمنی}}
\persian{
پایداری حرارتی یک چالش حیاتی برای باتری‌های \english{Li-S} با فرمت بزرگ باقی می‌ماند.$^{۱۷}$ برخلاف باتری‌های لیتیوم-یون متداول که در آن‌ها فرار حرارتی معمولاً در آند به دلیل تشکیل دندریت لیتیوم آغاز می‌شود، در سلول‌های \english{Li-S}، خرابی فاجعه‌بار اغلب از کاتد شروع می‌شود. این امر ناشی از نقاط ذوب پایین گوگرد (۱۱۵ درجه سانتی‌گراد) و \english{Li₂S} (حدود ۲۰۰ درجه سانتی‌گراد) است که واکنش‌های متقاطع ناخواسته الکترود را در دماهای بالا ترویج می‌دهد. مطالعات اخیر فرار حرارتی با استفاده از سلول‌های کیسه‌ای \english{1.5 Ah Li-S} با الکترولیت‌های مبتنی بر \english{LiTFSI} نشان داد که دمای شروع خودگرمایشی\footnote{\english{Self-heating onset temperature}} در حالت شارژ ۱۰۰٪ (\english{SOC}) به پایینی ۸۲ درجه سانتی‌گراد می‌رسد و فرار حرارتی کامل در ۱۸۱ درجه سانتی‌گراد رخ می‌دهد.$^{۱۷}$ در \english{SOC} صفر درصد، پایداری حرارتی کمی بهبود یافت، اما تورم شدید و آزادسازی گاز همچنان مشاهده شد. آزمون‌های کالری‌سنجی نرخ شتاب (ARC)\footnote{\english{Accelerating Rate Calorimetry}} تأیید کردند که فرار حرارتی عمدتاً ناشی از تبخیر حلال (مانند دی‌متوکسی‌اتان (\english{DME}) و ۱,۳-دی‌اکسولان (\english{DOL})) است که گازهای قابل اشتعالی مانند \english{CH₄}, \english{C₂H₄} و \english{H₂S} تولید می‌کند. این یافته‌ها نیاز فوری به فرمولاسیون‌های پیشرفته الکترولیت، پایدارکننده‌های الکترود و استراتژی‌های مدیریت حرارتی برای کاهش خطرات ایمنی در باتری‌های تجاری \english{Li-S} را برجسته می‌کنند.$^{۱۷}$
}
\persian{
برای بهبود پایداری حرارتی، حلال‌های الکترولیت جایگزین مانند تترااتیلن گلیکول دی‌متیل اتر (\english{TEGDME}) مورد بررسی قرار گرفته‌اند. در حالی که الکترولیت‌های مبتنی بر \english{TEGDME} دماهای شروع خودگرمایشی بالاتری (حدود ۲۲۴ درجه سانتی‌گراد برای نمونه بکر، حدود ۱۳۵ درجه سانتی‌گراد برای نمونه چرخه‌زده) از خود نشان می‌دهند، اما فرار حرارتی را به طور کامل از بین نمی‌برند. علاوه بر این، استفاده از الکترولیت‌های حالت جامد (مانند آرژیرودیت \english{Li₆PS₅Cl}) مقاومت حرارتی را افزایش می‌دهد، اما ناپایداری فصل مشترک با فلز لیتیوم یک محدودیت بزرگ باقی می‌ماند. فرار حرارتی در سلول‌های \english{Li-S} تمام-جامد در ۱۹۰ درجه سانتی‌گراد مشاهده شده است که بر نیاز به بهبود فصل مشترک الکترود-الکترولیت و مواد مقاوم در برابر حرارت تأکید می‌کند.
}

\section*{\persian{۲. مواد پیشرفته برای بهبود کاتد}}
\persian{
در باتری‌های \english{Li-S}، یکی از حیاتی‌ترین اجزا برای بهبود عملکرد، کاتد است. کاتد گوگرد با چالش‌های قابل توجهی روبرو است، عمدتاً به دلیل رسانایی الکتریکی پایین ذاتی گوگرد و انحلال گونه‌های پلی‌سولفید میانی در الکترولیت. در نتیجه، بخش بزرگی از تحقیقات در زمینه باتری‌های \english{Li-S} بر روی بهبود کاتد گوگرد از طریق توسعه مواد پیشرفته‌ای که این مسائل را حل می‌کنند، متمرکز شده است. این بخش دو استراتژی اصلی برای بهبود کاتد گوگرد را بررسی می‌کند: استفاده از کامپوزیت‌های گوگرد-کربن و استراتژی‌های تثبیت پلی‌سولفید.
}

\subsection*{\persian{۲.۱. گرافن سه‌بعدی و کامپوزیت‌های آن برای باتری‌های لیتیوم-گوگرد با کارایی بالا}}
\persian{
پرداختن به مسائل یا محدودیت‌های \english{Li-S} نیازمند توسعه مواد میزبان پیشرفته گوگرد است که عملکرد الکتروشیمیایی را افزایش داده و طول عمر باتری را بهبود می‌بخشند. مواد کربنی متخلخل، به‌ویژه کربن‌های متخلخل سبک و رسانا، به طور گسترده به عنوان میزبان گوگرد مورد مطالعه قرار گرفته‌اند.$^۶$ این مواد نقش دوگانه‌ای ایفا می‌کنند: تسهیل انتقال الکترون...
}

% --- End of content for page 4 ---
% --- SECTION 5: Page 5 of Translation ---
% --- Notes: Includes formula for sulfur content and introduces Figure 3.
% --------------------------------------------------------------------------

\persian{
و محبوس کردن پلی‌سولفیدها در نانوحفره‌ها برای سرکوب مهاجرت و افزایش پایداری چرخه‌ای. برای جای دادن محتوای بالای گوگرد در کامپوزیت‌های گوگرد/کربن، حجم حفرات بزرگ ضروری است. به عنوان مثال، حجم حفرات نظری \english{3.48 cm³ g⁻¹} برای پشتیبانی از ۸۰٪ وزنی گوگرد مورد نیاز است (جدول ۲)،$^{۱۸}$ با در نظر گرفتن انبساط حجمی در طول لیتیاسیون.
}
\persian{
معادله محاسبه محتوای گوگرد به شرح زیر است:
}
\begin{equation*}
S(\text{wt\%}) = \frac{\frac{V}{1.8} \times d}{\frac{V}{1.8} \times d + 1} \times 100\%
\end{equation*}
\persian{
که در آن: \english{V} = حجم حفرات ویژه کربن متخلخل (\english{cm³ g⁻¹})، \english{1.8} = ضریب انبساط حجمی از گوگرد به \english{Li₂S}، و \english{d} = چگالی گوگرد (\english{2.07 g cm⁻³}) است.
از آنجا که گوگرد هنگام تبدیل به \english{Li₂S} حدود ۸۰٪ انبساط حجمی پیدا می‌کند، حجم حفرات موجود برای جای دادن گوگرد با ضریب \english{1/1.8} تنظیم می‌شود. بنابراین، محتوای کل گوگرد با ضرب حجم حفرات ویژه تنظیم شده در چگالی گوگرد تعیین می‌شود.$^{۱۸}$
}

\begin{table}[h]
    \centering
    \caption{\persian{\farsibold{جدول ۲.} رابطه بین حجم حفرات ویژه مواد میزبان متخلخل و محتوای نظری \english{S₈}.$^{۱۸}$}}
    \label{tab:pore_volume}
    \begin{tabular}{rccccccc}
        \toprule
        \persian{۹۰} & \persian{۸۵} & \persian{۸۰} & \persian{۷۵} & \persian{۷۰} & \persian{۶۵} & \persian{۶۰} & \farsibold{محتوای گوگرد (درصد وزنی)} \\
        \hline
        \english{7.83} & \english{4.93} & \english{3.48} & \english{2.61} & \english{2.03} & \english{1.62} & \english{1.3} & \farsibold{حجم حفرات ویژه (\english{cm³ g⁻¹})} \\
        \bottomrule
    \end{tabular}
\end{table}

\persian{
مطالعات اخیر کارایی مواد کربنی متخلخل پیشرفته را در دستیابی به بارگذاری بالای گوگرد نشان داده‌اند. ورقه‌های گرافن دوپ‌شده با نیتروژن با حجم حفرات بسیار بالا، سلول‌های \english{Li-S} را قادر می‌سازد تا به ظرفیت حدود \english{1000 mAh g⁻¹} دست یابند. به طور مشابه، گرافن بسیار متخلخل با حجم حفرات \english{3.51 cm³ g⁻¹} یک کاتد گوگرد با ۸۰٪ وزنی گوگرد و ظرفیت ویژه اولیه \english{1500 mAh g⁻¹} تولید می‌کند.$^{۱۸}$
}
\persian{
فراتر از حجم حفرات، توزیع اندازه حفرات به طور حیاتی بر عملکرد باتری \english{Li-S} تأثیر می‌گذارد. ساختارهای متخلخل به میکروحفره‌ها (\english{<2 nm})، مزوحفره‌ها (\english{2-50 nm}) و ماکروحفره‌ها (\english{>50 nm}) طبقه‌بندی می‌شوند. در حالی که میکروحفره‌ها محبوس‌سازی قوی گوگرد و تماس الکتریکی را ترویج می‌دهند، حجم حفرات و انتقال یون لیتیوم را محدود می‌کنند. قابل ذکر است که میکروحفره‌های کوچکتر از \english{0.5 nm} تشکیل مولکول‌های گوگرد کوچک را امکان‌پذیر می‌سازند و به طور مؤثر اثر شاتل پلی‌سولفید را کاهش داده و پایداری چرخه‌ای را افزایش می‌دهند. در مقابل، مزوحفره‌ها و ماکروحفره‌ها نفوذ بهتر الکترولیت و سینتیک یون لیتیوم را فراهم می‌کنند و از محتوای بالای گوگرد پشتیبانی می‌کنند اما به طور بالقوه پایداری چرخه‌ای بلندمدت را به خطر می‌اندازند. اگرچه اندازه حفرات بهینه همچنان تحت بررسی است، مزوحفره‌ها و ماکروحفره‌ها به طور کلی برای دستیابی به محتوای بالای گوگرد ترجیح داده می‌شوند، در حالی که میکروحفره‌های کوچک (\english{<0.5 nm}) برای پایداری سودمند هستند. میکروحفره‌های بزرگ (\english{0.5-2 nm}) به دلیل سینتیک واکنش کند و نگهداری ضعیف گوگرد، کمتر مطلوب هستند.$^{۱۹,۲۰}$
}
\persian{
اخیراً، گرافن سه‌بعدی (\english{3D}) و کامپوزیت‌های آن به دلیل رسانایی الکتریکی استثنایی، یکپارچگی ساختاری و توانایی در کاهش شاتل پلی‌سولفید، به عنوان مواد امیدوارکننده‌ای برای باتری‌های \english{Li-S} ظهور کرده‌اند.$^{۲۱-۵۶}$ برخلاف مواد کربنی متداول، گرافن سه‌بعدی دارای یک شبکه متخلخل به‌هم‌پیوسته است که انتقال الکترون را افزایش می‌دهد، نفوذ یون لیتیوم را تسهیل می‌کند و انبساط حجمی گوگرد را در طی چرخه‌های شارژ-دشارژ در خود جای می‌دهد. به طور خاص، گرافن سه‌بعدی دوپ‌شده با نیتروژن (\english{3D-NG}) از طریق برهمکنش‌های شیمیایی قوی، نگهداری برتر گوگرد را نشان می‌دهد و به طور مؤثر مهاجرت پلی‌سولفید را سرکوب کرده و عمر چرخه‌ای را طولانی می‌کند. ادغام گوگرد در چارچوب‌های گرافن سه‌بعدی، بارگذاری بالای گوگرد را در حالی که پایداری الکتروشیمیایی عالی حفظ می‌شود، امکان‌پذیر می‌سازد. یافته‌های پژوهشی نشان می‌دهد که کاتدهای گوگرد مبتنی بر گرافن سه‌بعدی قابلیت نرخ بالا، عمر چرخه‌ای طولانی و بازده کولمبی قابل توجهی را حتی در دانسیته‌های جریان بالا از خود نشان می‌دهند. علاوه بر این، انعطاف‌پذیری مکانیکی گرافن سه‌بعدی دوام الکترود را افزایش داده و از تخریب ساختاری در طول چرخه‌زنی طولانی‌مدت جلوگیری می‌کند.
}
\persian{
برای افزایش بارگذاری گوگرد و در عین حال حفظ یکپارچگی ساختاری، یک کامپوزیت با محتوای بالای گوگرد که در یک چارچوب \english{3D-NG} پیچیده شده، پیشنهاد شده است.$^{۵۶}$ ساختار به‌هم‌پیوسته \english{3D-NG} نه تنها انتقال الکترون را تسهیل کرده و انبساط گوگرد را در خود جای می‌دهد، بلکه از طریق برهمکنش‌های شیمیایی، تثبیت پلی‌سولفید را نیز تقویت می‌کند. یک روش سنتز محلولی تک‌مرحله‌ای، ساخت این کامپوزیت را با محتوای گوگرد استثنایی \english{87.6 wt\%} امکان‌پذیر ساخت. ارزیابی‌های الکتروشیمیایی ظرفیت دشارژ پایدار \english{792 mAh g⁻¹} در \english{600 mA g⁻¹} پس از ۱۴۵ چرخه، با نرخ افت حداقل \english{0.05\%} در هر چرخه را نشان داد. علاوه بر این، حتی در نرخ بالای \english{1500 mA g⁻¹}، الکترود ظرفیت \english{671 mAh g⁻¹} را پس از ۲۰۰ چرخه حفظ کرد که نشان‌دهنده عملکرد نرخ فوق‌العاده آن است.$^{۵۶}$
}
\persian{
یک پیشرفت قابل توجه دیگر، کاتد آئروژل گرافن دوپ‌شده با \english{Li₂S} سه‌بعدی خودپشتیبان است که توسط ژو و همکاران معرفی شد.$^{۵۰}$ برخلاف کاتدهای مبتنی بر گوگرد سنتی، این طراحی از \english{Li₂S} به عنوان ماده کاتدی پیش-لیتیومی شده استفاده می‌کند، نیاز به آندهای فلز لیتیوم را دور زده و ایمنی را بهبود می‌بخشد. فرآیند ساخت از یک تکنیک نفوذ-تبخیر مایع برای پوشش یکنواخت \english{Li₂S} بر روی آئروژل‌های گرافن دوپ‌شده با هترواتم استفاده کرد و یک کاتد رسانا و پایدار مکانیکی تشکیل داد. شبکه گرافن سه‌بعدی انتقال سریع یون و الکترون را امکان‌پذیر ساخت، در حالی که دوپینگ نیتروژن یا بور جذب پلی‌سولفید را افزایش داده و اثر شاتل را به حداقل رساند. آزمون‌های الکتروشیمیایی ظرفیت ویژه اولیه \english{801 mAh g⁻¹} در \english{0.3C} را با حفظ قابل توجه \english{403 mAh g⁻¹} پس از ۳۰۰ چرخه در \english{0.5C} نشان داد. محاسبات نظریه تابعی چگالی (\english{DFT}) پیوند قوی \english{Li₂S} بر روی گرافن دوپ‌شده را بیشتر تأیید کرد و به طور مؤثر مانع انرژی فعال‌سازی را کاهش داده و پایداری ساختاری کاتد را افزایش داد. در مقایسه با کاتدهای \english{Li₂S} متداول، این طراحی نوین مقاومت انتقال بار به طور قابل توجهی کاهش یافته و قابلیت نرخ برتری را نشان داد و ظرفیت \english{487 mAh g⁻¹} در \english{2C} ارائه داد.$^{۵۰}$
}
\persian{
علاوه بر این، لی و همکاران (۲۰۲۰) یک لایه یکنواخت پلی‌پیرول (\english{PPy}) پوشش داده شده بر روی آئروژل گوگرد/گرافن (\english{PPy@S/GA-VD}) را از طریق یک تکنیک رسوب‌دهی فاز بخار برای بهبود عملکرد باتری \english{Li-S} معرفی کردند که در شکل ۳ در زیر نشان داده شده است.$^{۵۷}$
}
\persian{
فرآیند سنتز \english{PPy@S/GA-VD} شامل مراحل کلیدی زیر است:$^{۵۷}$ (۱) تشکیل هیدروژل گوگرد-گرافن (\english{S@graphene hydrogel}): یک هیدروژل گرافن متخلخل سه‌بعدی با ذرات گوگرد جاسازی شده در آن سنتز می‌شود. این ساختار سطح ویژه بالا و رسانایی را فراهم می‌کند.
}

\begin{figure}[h]
    \centering
    \includegraphics[width=0.9\columnwidth]{example-image-b} % Placeholder for Figure 3
    \caption*{} % Placeholder for caption - will be added on next page
    \label{fig:3}
\end{figure}

% --- End of content for page 5 ---
% --- SECTION 6: Page 6 of Translation ---
% --- Notes: Figure 3 caption completed, Figure 4 introduced.
% --------------------------------------------------------------------------

\begin{figure*}[t]
    \centering
    \includegraphics[width=0.9\textwidth]{example-image-b} % Placeholder for Figure 3 from previous page
    \caption{\persian{
    \farsibold{شکل ۳.} سنتز گام به گام کامپوزیت آئروژل گوگرد/گرافن با پوشش \english{PPy} (\english{PPy@S/GA-VD}) که به عنوان ماده کاتدی برای باتری‌های \english{Li-S} استفاده شده است. بازتولید شده از مرجع ۵۷ با مجوز از انجمن شیمی آمریکا، کپی‌رایت ۲۰۲۰.
    \newline
    چارچوبی برای بهره‌برداری از گوگرد. (۲) بارگذاری فیلم \english{FeToS}: هیدروژل در محلول آهن(III) پی-تولوئن سولفونات (\english{FeToS}) غوطه‌ور می‌شود. \english{FeToS} به عنوان اکسیدان برای فرآیند پلیمریزاسیون بعدی عمل می‌کند. (۳) رسوب بخار پیرول: هیدروژل در معرض بخار پیرول قرار می‌گیرد و به مونومرهای پیرول اجازه می‌دهد تا در ساختار سه‌بعدی نفوذ کنند. این امر توزیع یکنواخت پیرول را در سراسر ماده تضمین می‌کند. (۴) فرآیند پلیمریزاسیون: پیرول روی ماتریس هیدروژل پلیمریزه شده و یک پوشش نازک و یکنواخت \english{PPy} ایجاد می‌کند. این لایه \english{PPy} جایگاه‌های پیوند شیمیایی قوی برای به دام انداختن پلی‌سولفیدهای لیتیوم (\english{Li₂Sₙ}) فراهم کرده و اثر شاتل را کاهش می‌دهد. (۵) ساختار نهایی: کامپوزیت \english{PPy@S/GA-VD}؛ کامپوزیت نهایی دارای ساختاری چروکیده و متخلخل است که انتقال الکترون و نگهداری پلی‌سولفید را افزایش می‌دهد. پوشش \english{PPy} از انحلال پلی‌سولفید جلوگیری کرده و در نتیجه پایداری چرخه‌ای و عملکرد نرخ باتری را بهبود می‌بخشد. نمودار داخلی در شکل ۳ سمت راست، نقش لایه \english{PPy} را در جذب مؤثر \english{Li₂Sₙ}، کاهش اتلاف ماده فعال و بهبود عملکرد باتری برجسته می‌کند.$^{۵۷}$ لایه \english{PPy} جذب قوی پلی‌سولفید را فراهم کرده، اثر شاتل را کاهش داده و پایداری چرخه‌ای را بهبود می‌بخشد. محاسبات \english{DFT} تأیید می‌کند که \english{PPy} انرژی پیوندی بالاتری (\english{0.72 eV}) برای پلی‌سولفیدها نسبت به گرافن خالص (\english{0.13 eV}) دارد که منجر به نگهداری بهتر پلی‌سولفید می‌شود.$^{۵۷}$ ساختار چروکیده تشکیل شده در طول رسوب‌دهی، بهره‌برداری از گوگرد را افزایش می‌دهد و منجر به ظرفیت دشارژ بالا (\english{1167 mAh g⁻¹} در \english{0.2C}) و عمر چرخه‌ای طولانی (\english{698 mAh g⁻¹} پس از ۵۰۰ چرخه در \english{0.5C} با نرخ زوال \english{0.03\%} در هر چرخه) می‌شود.$^{۵۷}$
    }}
    \label{fig:3_cont}
\end{figure*}


\subsubsection*{\persian{۲.۱.۱. معماری‌های نوآورانه و مواد کامپوزیتی برای باتری‌های \english{Li-S}}}
\persian{
معرفی کاتدهای فوم گرافن دوپ‌شده با نیتروژن سه‌بعدی بدون بایندر (\english{NGF}) از طریق تکنیک‌های متالورژی پودر و آنیلینگ، این حوزه را بیشتر پیش برده است.$^{۳۷}$ با استفاده از یک قالب نیکل و ملامین به عنوان منبع نیتروژن، \english{NGF} حاصل، یک ساختار مزوپور متصل به هم با سطح ویژه بالا (\english{656 m² g⁻¹}) و حجم حفرات (\english{3.0584 cm³ g⁻¹}) را به نمایش می‌گذارد. این ویژگی‌ها انتقال سریع بار را تسهیل کرده، مهاجرت پلی‌سولفید را سرکوب می‌کنند و انبساط حجمی گوگرد را در خود جای می‌دهند. آزمون‌های الکتروشیمیایی ظرفیت دشارژ اولیه \english{987 mAh g⁻¹} در \english{0.2C} را با حفظ ظرفیت \english{82.9\%} پس از ۲۰۰ چرخه و نرخ زوال ظرفیت پایین \english{0.061\%} در هر چرخه طی ۵۰۰ چرخه در \english{2C} نشان داد.$^{۳۷}$
}
\persian{
یک پیشرفت قابل توجه در تثبیت پلی‌سولفید با توسعه کامپوزیت‌های نانوسیم گرافن/تیتانیوم نیترید دوپ‌شده با نیتروژن سه‌بعدی (\english{3DNG/TiN}) حاصل شد.$^{۴۱}$ این ماده هیبریدی از رسانایی بالا و انعطاف‌پذیری مکانیکی گرافن سه‌بعدی بهره می‌برد و همزمان از قابلیت قوی تیتانیوم نیترید (\english{TiN}) برای لنگراندازی پلی‌سولفید استفاده می‌کند. الکترود کامپوزیتی ظرفیت ویژه اولیه \english{1510 mAh g⁻¹} در \english{0.5C} را نشان داد و ظرفیت \english{1267 mAh g⁻¹} را پس از ۱۰۰ چرخه حفظ کرد. علاوه بر این، قابلیت بارگذاری بالای گوگرد آن منجر به ظرفیت مساحتی \english{12.0 mAh cm⁻²} در بارگذاری گوگرد \english{9.6 mg cm⁻²} شد که از عملکرد باتری‌های لیتیوم-یون متداول فراتر رفت. رسانایی برتر \english{TiN} (\english{4.0 × 10⁴ S m⁻¹}) در مقایسه با میزبان‌های مبتنی بر اکسید به این عملکرد الکتروشیمیایی قابل توجه کمک کرد.
}
\persian{
برای مقابله بیشتر با اثرات شاتل پلی‌سولفید، یک اسفنج اکسید گرافن کاهش‌یافته سه‌بعدی (\english{rGO}) با حجم حفرات بسیار بالا و گروه‌های عاملی اکسیژن بهینه‌سازی شده توسعه یافت.$^{۳۰}$ اسفنج \english{3D rGO} که از طریق فرآیندهای کاهش حرارتی و مایکروویو سنتز شده بود، حجم حفرات استثنایی \english{6.4 cm³ g⁻¹} را نشان داد که بالاترین مقدار گزارش شده برای چنین موادی است. ساختار متخلخل سلسله‌مراتبی آن، محبوس‌سازی پلی‌سولفید را افزایش داده، اثرات شاتل را به حداقل رسانده و عمر چرخه‌ای را به طور قابل توجهی بهبود بخشید. آزمون‌های الکتروشیمیایی ظرفیت دشارژ چشمگیر \english{1607 mAh g⁻¹} در \english{0.1C} و \english{1330 mAh g⁻¹} با بارگذاری گوگرد \english{6.6 mg cm⁻²} را نشان داد. علاوه بر این، آزمون‌های چرخه‌زنی بلندمدت نرخ زوال ظرفیت پایین \english{0.065\%} در هر چرخه طی ۲۰۰ چرخه در \english{1.0C} را با بازده کولمبی نزدیک به ۹۸٪ نشان داد.$^{۳۰}$ شکل ۴ سنتز، مشخصه‌یابی ساختاری و عملکرد الکتروشیمیایی اکسید گرافن کاهش‌یافته سه‌بعدی را نشان می‌دهد.
}

\begin{figure*}[t]
    \centering
    \includegraphics[width=\textwidth]{example-image-c} % Placeholder for Figure 4
    \caption{\persian{
    \farsibold{شکل ۴.} (الف) تصویر شماتیک از مکانیزم تشکیل \english{3D-rGO} از طریق فرآیندهای کاهش مایکروویو و حرارتی، شامل برهمکنش جهشی، برهمکنش تونلی و آب‌زدایی که منجر به مزدوج شدن خوشه‌ای می‌شود. (ب) و (ج) تصاویر \english{SEM} از \english{rGO₁₂.₇} که ساختار متخلخل به‌هم‌پیوسته آن را نشان می‌دهد. (د) و (ه) تصاویر \english{TEM} از \english{rGO₁₂.₇} که نانوساختار دقیق آن را آشکار می‌سازد. (و) قابلیت نرخ باتری‌های \english{Li-S} (\english{LSBs}) با کاتدهای مختلف \english{rGO} در بارگذاری گوگرد \english{2.2 mg cm⁻²}. (ز) پروفایل‌های ولتاژ شارژ-دشارژ \english{LSBs} با بارگذاری گوگرد \english{2.2 mg cm⁻²} در نرخ‌های \english{C} مختلف. (ح) عملکرد نرخ \english{LSBs} با بارگذاری گوگرد \english{6.6 mg cm⁻²}. (ط) پروفایل‌های ولتاژ شارژ-دشارژ \english{LSBs} با بارگذاری گوگرد \english{6.6 mg cm⁻²} در نرخ‌های \english{C} مختلف. (ی) و (ک) عملکرد چرخه‌زنی بلندمدت \english{LSBs} با بارگذاری گوگرد \english{2.2 mg cm⁻²} و \english{6.6 mg cm⁻²} که حفظ ظرفیت و بازده کولمبی را در طی چرخه‌های متعدد نشان می‌دهد. بازتولید شده از مرجع ۳۰ با مجوز از \english{Elsevier Ltd.}، کپی‌رایت ۲۰۲۴.
    }}
    \label{fig:4}
\end{figure*}


% --- End of content for page 6 ---
% --- SECTION 7: Page 7 of Translation ---
% --- Notes: Concludes 3D graphene section and introduces MOFs.
% --------------------------------------------------------------------------

\persian{
(rGO) برای باتری‌های لیتیوم-گوگرد (\english{LSBs}).$^{۳۰}$ شکل ۴(الف) مکانیزم تشکیل پیشنهادی \english{3D-rGO} را از طریق فرآیندهای کاهش مایکروویو و حرارتی ارائه می‌دهد و مراحل کلیدی تبدیل را برجسته می‌کند. این فرآیند با آئروژل‌های اکسید گرافن (\english{GO}) آغاز می‌شود که از طریق برهمکنش‌های جهشی، تونلی و واکنش‌های آب‌زدایی، دچار بازآرایی ساختاری شده و در نهایت یک شبکه متخلخل به‌هم‌پیوسته را تشکیل می‌دهند. این تحول به طور قابل توجهی سطح ویژه و حجم حفرات ماده را افزایش می‌دهد و سازگاری آن را به عنوان میزبان برای \english{Li₂S₆} در \english{LSBs} بهبود می‌بخشد. مورفولوژی نمونه بهینه‌سازی شده \english{rGO₁₂.₇} در تصاویر \english{SEM} در شکل‌های ۴(ب) و (ج) و تصاویر \english{TEM} در شکل‌های ۴(د) و (ه) نشان داده شده است. تصاویر \english{SEM} یک ساختار متخلخل سلسله‌مراتبی به خوبی توسعه‌یافته را نشان می‌دهند که نفوذ الکترولیت و انتقال بار را تسهیل می‌کند. تصاویر \english{TEM} بیشتر شبکه گرافنی به‌هم‌پیوسته را تأیید می‌کنند که به سرکوب نفوذ پلی‌سولفید کمک کرده و در نتیجه اثر شاتلی که اغلب عملکرد باتری را تخریب می‌کند، به حداقل می‌رساند. این معماری متخلخل بهینه‌شده در افزایش بهره‌برداری از گوگرد و پایداری چرخه‌ای بلندمدت در \english{LSBs} حیاتی است. عملکرد الکتروشیمیایی \english{LSBs} با استفاده از \english{rGO} با نسبت‌های مختلف \english{C/O} در شکل‌های ۴(و) تا (ک) تحلیل شده است. آزمون قابلیت نرخ (و) نشان می‌دهد که \english{rGO₁₂.₇} بالاترین ظرفیت دشارژ را ارائه می‌دهد و از نمونه‌های دیگر در نرخ‌های \english{C} مختلف بهتر عمل می‌کند. پروفایل‌های ولتاژ شارژ-دشارژ همانطور که در شکل‌های ۴(ز) و (ط) نشان داده شده، پلاریزاسیون پایین \english{rGO₁₂.₇} را بیشتر نشان می‌دهد که بیانگر رسانایی افزایش‌یافته و واکنش‌های ردوکس کارآمد است. شکل‌های ۴(ح) و (ط) عملکرد را در بارگذاری بالاتر گوگرد (\english{6.6 mg cm⁻²}) مقایسه می‌کنند و نشان می‌دهند که \english{rGO₁₂.₇} حفظ ظرفیت و برگشت‌پذیری برتری را حفظ می‌کند.
}
\persian{
در نهایت، پایداری چرخه‌ای بلندمدت همانطور که در شکل‌های ۴(ی) و (ک) نشان داده شده، حفظ ظرفیت قابل توجه و بازده کولمبی \english{LSBs} مبتنی بر \english{rGO} را برجسته می‌کند. کاتد بهینه‌شده \english{rGO₁₂.₇} نرخ زوال ظرفیت پایینی (حدود \english{0.065\%} در هر چرخه) و بازده کولمبی ۹۸٪ را در طی ۲۰۰ چرخه حفظ می‌کند که قابلیت آن را برای کاربردهای عملی \english{LSB} با انرژی بالا نشان می‌دهد. این شکل در مجموع همبستگی بین ساختار \english{rGO}، محبوس‌سازی پلی‌سولفید و عملکرد الکتروشیمیایی را نشان می‌دهد و پتانسیل اسفنج‌های \english{3D rGO} را به عنوان یک ماده میزبان پیشرفته گوگرد تقویت می‌کند.
}
\subsubsection*{\persian{۲.۱.۲. چشم‌اندازهای آینده و چالش‌ها}}
\persian{
در حالی که گرافن سه‌بعدی و کامپوزیت‌های آن پتانسیل قابل توجهی در پرداختن به چالش‌های کلیدی باتری‌های \english{Li-S} نشان داده‌اند، چندین مسئله باقی مانده است. سنتز در مقیاس بزرگ گرافن سه‌بعدی با معماری حفرات کنترل‌شده و دوپینگ یکنواخت هترواتم نیازمند بهینه‌سازی بیشتر برای افزایش تکرارپذیری و صرفه اقتصادی است. علاوه بر این، توسعه فرمولاسیون‌های الکترولیت پایدار و سازگار با چارچوب‌های گرافن سه‌بعدی برای دستیابی به پایداری چرخه‌ای بلندمدت حیاتی خواهد بود. مطالعات بیشتر که گرافن سه‌بعدی را با الکترولیت‌های حالت جامد ادغام می‌کنند، می‌توانند مسیرهای امیدوارکننده‌ای به سوی باتری‌های \english{Li-S} تمام-جامد ایمن‌تر و با عملکرد بالا ارائه دهند.
}
\persian{
به طور خلاصه، استفاده از معماری‌های مبتنی بر گرافن سه‌بعدی یک رویکرد تحول‌آفرین برای بهبود عملکرد الکتروشیمیایی باتری‌های \english{Li-S} است. ویژگی‌های ساختاری منحصربه‌فرد آنها—مانند رسانایی الکتریکی بالا، تخلخل سلسله‌مراتبی و لنگراندازی قوی پلی‌سولفید—چالش‌های عمده مرتبط با کاتدهای گوگرد، از جمله عمر چرخه‌ای ضعیف، اثرات شاتل و سینتیک واکنش کند را برطرف می‌کند. پیشرفت‌های اخیر در کامپوزیت‌های گرافن سه‌بعدی، از جمله دوپینگ هترواتم و ادغام با ترکیبات فلزات واسطه، بهبودهای قابل توجهی در ظرفیت ویژه، پایداری چرخه‌ای و عملکرد نرخ را ممکن ساخته است. تحقیقات آینده باید بر بهینه‌سازی تکنیک‌های سنتز، سازگاری الکترولیت و روش‌های تولید مقیاس‌پذیر برای تسهیل کاربرد عملی این مواد در سیستم‌های ذخیره‌سازی انرژی نسل بعد متمرکز شود.
}

\subsection*{\persian{۲.۲. چارچوب‌های فلزی-آلی (\english{MOFs}) برای باتری‌های پیشرفته \english{Li-S}}}
\persian{
چارچوب‌های فلزی-آلی (\english{MOFs}) به عنوان مواد امیدوارکننده‌ای برای بهبود عملکرد باتری \english{Li-S} با پرداختن به چالش‌های کلیدی مانند اثر شاتل پلی‌سولفید، سینتیک واکنش کند و بهره‌برداری ناکارآمد از گوگرد ظهور کرده‌اند.$^{۱۳, ۵۸-۹۰}$ این مواد بسیار متخلخل، که از گره‌های فلزی هماهنگ با لیگاندهای آلی تشکیل شده‌اند، ساختارهای قابل تنظیم، سطوح ویژه بالا و پایداری شیمیایی عالی ارائه می‌دهند که آنها را به گزینه‌های ایده‌آلی برای افزایش کارایی باتری \english{Li-S} تبدیل می‌کند.
}
\persian{
\english{MOFs} چندین عملکرد در باتری‌های \english{Li-S} ایفا می‌کنند. به عنوان میزبان کاتد، \english{MOFs} گوگرد را کپسوله کرده و پلی‌سولفیدهای لیتیوم را به صورت فیزیکی محبوس می‌کنند تا انحلال و مهاجرت آنها را کاهش دهند. \english{MOFs} عامل‌دار شده با گروه‌های قطبی (مانند اکسیژن، نیتروژن یا مراکز فلزی) لنگراندازی شیمیایی برای پلی‌سولفیدها فراهم می‌کنند و در نتیجه اثر شاتل را سرکوب کرده و عمر چرخه‌ای را بهبود می‌بخشند. علاوه بر این، \english{MOFs} با ارائه جایگاه‌های فعالی که واکنش‌های ردوکس گوگرد را تسهیل می‌کنند، به بهبود سینتیک واکنش کمک می‌کنند. فراتر از کاربردهای کاتدی، \english{MOFs} همچنین در پوشش‌های جداکننده و لایه‌های میانی برای به دام انداختن پلی‌سولفیدها و کاهش بیشتر اتلاف ظرفیت به کار گرفته شده‌اند. علاوه بر این، مواد مشتق از \english{MOF}، از جمله \english{MOFs} کربنیزه شده و کاتالیزورهای مبتنی بر \english{MOF}، با ترویج واکنش‌های تبدیل الکتروشیمیایی کارآمد، بهره‌برداری از گوگرد را بهبود می‌بخشند.
}
\persian{
علی‌رغم این مزایا، چالش‌هایی از جمله روش‌های سنتز پیچیده، نگرانی‌های مربوط به پایداری و محدودیت‌ها در رسانایی باقی مانده است. تلاش‌های پژوهشی کنونی بر بهینه‌سازی ساختارهای \english{MOF}، افزایش رسانایی الکتریکی و مقیاس‌بندی تولید برای تسهیل کاربردهای تجاری متمرکز است. اگر این چالش‌ها برطرف شوند، \english{MOFs} پتانسیل قابل توجهی برای تحقق قابلیت تجاری‌سازی باتری‌های \english{Li-S} دارند.
}
\subsubsection*{\persian{۲.۲.۱. پیشرفت‌های اخیر در جداکننده‌ها و لایه‌های میانی مشتق از \english{MOF}}}
\persian{
بای و همکاران یک جداکننده نوین متشکل از یک \english{MOF} (\english{HKUST-1}) ادغام شده با اکسید گرافن (\english{MOF@GO}) را برای عمل به عنوان یک غربال یونی توسعه دادند.$^{۹۰}$ ساختار میکرو متخلخل \english{MOF} (حدود \english{9 Å}) به طور مؤثری مهاجرت گونه‌های پلی‌سولفید (\english{Li₂Sₙ, 4 < n ≤ 8}) را مسدود می‌کند در حالی که به یون‌های لیتیوم اجازه عبور آزادانه می‌دهد. لایه‌های \english{GO} پایداری مکانیکی را تقویت کرده و یک غشای خود ایستا ایجاد می‌کنند. جداکننده \english{MOF@GO} به طور قابل توجهی نفوذ پلی‌سولفید را سرکوب می‌کند و عملکرد الکتروشیمیایی فوق‌العاده‌ای ارائه می‌دهد: نرخ زوال ظرفیت به پایینی \english{0.019\%} در هر چرخه طی ۱۵۰۰ چرخه در \english{1C}. در مقایسه با جداکننده‌های فقط با \english{GO}، \english{MOF@GO} پایداری چرخه‌ای بلندمدت و قابلیت نرخ برتری را نشان می‌دهد. این مطالعه پتانسیل مواد مبتنی بر \english{MOF} را به عنوان جداکننده‌های عملکردی در باتری‌های \english{Li-S} برجسته می‌کند.
}

% --- End of content for page 7 ---
% --- SECTION 8: Page 8 of Translation ---
% --- Notes: Includes Figure 5 and its detailed caption, introduces MOF-derived hosts.
% --------------------------------------------------------------------------

\persian{
باتری‌ها، راه‌حلی ساده و در عین حال مؤثر برای یکی از پایدارترین چالش‌های این فناوری ارائه می‌دهد. گوئو و همکاران یک \english{MOF} رسانای مزدوج دو بعدی \english{π-d} نوین، \english{Ni-HAB} (نیکل-هگزآمینوبنزن)، را که با نانولوله‌های کربنی (\english{CNTs}) یکپارچه شده بود، به عنوان یک لایه اصلاح‌کننده جداکننده (\english{Ni-HAB@CNT}) برای مقابله با نفوذ پلی‌سولفید معرفی کردند.$^{۷۵}$ شکل ۵ سنتز، ویژگی‌های ساختاری و عملکرد الکتروشیمیایی جداکننده اصلاح‌شده با \english{Ni-HAB@CNT} را برای باتری‌های \english{Li-S} نشان می‌دهد.$^{۷۵}$ شکل ۵(الف) یک نمودار شماتیک از فرآیند رشد درجا (\english{in situ}) \english{MOF} \english{Ni-HAB} بر روی نانولوله‌های کربنی (\english{CNTs}) را ارائه می‌دهد. این فرآیند شامل \english{Ni(NO₃)₂·6H₂O} و هگزآمینوبنزن (\english{HAB}) است که بر روی سطح \english{CNT} به یک ساختار \english{MOF} رسانای دو بعدی با بلورینگی بالا مونتاژ شده و کامپوزیت نهایی \english{Ni-HAB@CNT} را تشکیل می‌دهند. شکل ۵(ب) ساختار اتمی \english{Ni-HAB} را نشان می‌دهد و بر میکروحفره‌های منظم (حدود \english{8 Å}) تأکید می‌کند که به طور انتخابی به یون‌های \english{Li⁺} اجازه مهاجرت می‌دهند در حالی که نفوذ پلی‌سولفید لیتیوم (\english{LPS}) را مسدود کرده و اثر شاتل را کاهش می‌دهند. ساختار نوار الکترونیکی در شکل ۵(ج) ماهیت فلزی \english{Ni-HAB} را تأیید می‌کند که به مزدوج‌شدگی قوی \english{π-d} نسبت داده می‌شود و انتقال الکترون و فعالیت کاتالیزوری را افزایش می‌دهد. شکل‌های ۵(د-و) نقش عملکردی \english{Ni-HAB@CNT} را در بهبود عملکرد باتری \english{Li-S} نشان می‌دهند. شکل ۵(د) فرآیند تبدیل ردوکس گوگرد را که بر روی سطح کامپوزیت \english{Ni-HAB@CNT} در یک باتری لیتیوم-گوگرد (\english{Li-S}) رخ می‌دهد، به تصویر می‌کشد. در حین عملکرد باتری، پلی‌سولفیدهای لیتیوم با زنجیره بلند (\english{LPSs}) مانند \english{Li₂S₈} و \english{Li₂S₆} که در کاتد تولید می‌شوند، یک سری واکنش‌های کاهش مرحله‌ای را طی کرده و به گونه‌های با زنجیره کوتاه‌تر (\english{Li₂S₄, Li₂S₂}) و در نهایت \english{Li₂S} نامحلول تبدیل می‌شوند. چارچوب \english{Ni-HAB@CNT} نقش حیاتی کاتالیزوری و محبوس‌سازی را در این تبدیل ایفا می‌کند. نانوورقه‌های \english{Ni-HAB} که دارای جایگاه‌های هماهنگی \english{Ni-N} فراوان و رسانایی فلزی ذاتی هستند، به طور قابل توجهی واکنش‌های ردوکس را با کاهش موانع انرژی برای تبدیل \english{LPS} تسریع می‌کنند. همزمان، ساختار متخلخل \english{Ni-HAB} (اندازه حفرات حدود \english{8 Å}) به طور انتخابی به یون‌های لیتیوم اجازه عبور می‌دهد در حالی که نفوذ گونه‌های پلی‌سولفید را محدود کرده و به طور مؤثری اثر شاتل را کاهش می‌دهد. علاوه بر این، شبکه \english{CNT} یک ستون فقرات رسانا برای انتقال کارآمد بار و پراکندگی یکنواخت \english{Ni-HAB} فراهم می‌کند و قرار گرفتن بالای جایگاه‌های کاتالیزوری و گوگرد پایدار را تضمین می‌نماید. در شکل ۵(ه)، یک جداکننده پلی‌پروپیلن (\english{PP}) سنتی نشان داده شده است که از شاتل شدید \english{LPS} رنج می‌برد، جایی که پلی‌سولفیدهای محلول به آند لیتیوم نفوذ کرده و باعث افت ظرفیت و عمر چرخه‌ای ضعیف می‌شوند. این امر منجر به چرخه‌های شارژ/دشارژ ناکارآمد می‌شود که با چهره غمگین نشان داده شده است. برعکس، شکل ۵(و) جداکننده اصلاح‌شده پوشش داده شده با \english{Ni-HAB@CNT} را نشان می‌دهد که به عنوان یک مانع فیزیکی و شیمیایی کارآمد عمل می‌کند. ساختار رسانا و متخلخل نه تنها مهاجرت \english{LPS} را مهار می‌کند، بلکه انتقال انتخابی یون \english{Li⁺} را نیز امکان‌پذیر می‌سازد. این امر منجر به بهبود پایداری باتری، افزایش بهره‌برداری از گوگرد و بازده کولمبی بالاتر می‌شود که با چهره شاد نمایش داده شده است. اثر هم‌افزایی \english{CNTs} رسانا و چارچوب کاتالیزوری \english{Ni-HAB} منجر به عملکرد الکتروشیمیایی برتر می‌شود.
}

\begin{figure*}[t]
    \centering
    \includegraphics[width=\textwidth]{example-image-b} % Placeholder for Figure 5
    \caption{\persian{
    \farsibold{شکل ۵.} (الف) تصویر شماتیک از فرآیند سنتز \english{Ni-HAB@CNT}. (ب) نماهای بالا و کنار از ساختار \english{Ni-HAB} که میکروحفره‌های منظم (حدود \english{8 Å}) را نشان می‌دهد. (ج) ساختار نوار الکترونیکی \english{Ni-HAB} حجیم در امتداد نقاط با تقارن بالا که ماهیت رسانای آن را نشان می‌دهد. انرژی فرمی روی صفر تنظیم شده است. (د) نمایش شماتیک از فرآیند تبدیل گوگرد بر روی \english{Ni-HAB@CNT} که فعالیت کاتالیزوری آن را در باتری‌های \english{Li-S} برجسته می‌کند. (ه) تصویری از یک جداکننده پلی‌پروپیلن (\english{PP}) متداول که شاتل شدید پلی‌سولفید را تجربه می‌کند. (و) جداکننده اصلاح‌شده با \english{Ni-HAB@CNT} به طور قابل توجهی شاتل پلی‌سولفید را سرکوب کرده و پایداری باتری را بهبود می‌بخشد. (ز) مقایسه عملکرد نرخ جداکننده‌های مختلف در دانسیته‌های جریان متفاوت. (ح) پروفایل‌های شارژ-دشارژ گالوانواستاتیک باتری‌های \english{Li-S} با جداکننده‌های اصلاح‌شده با \english{Ni-HAB@CNT} در نرخ‌های \english{C} مختلف. (ط) مقادیر پتانسیل پلاریزاسیون جداکننده‌های مختلف که اورپتانسیل کاهش‌یافته با \english{Ni-HAB@CNT} را نشان می‌دهد. (ی) پایداری چرخه‌ای باتری‌های \english{Li-S} در \english{0.2C} با مقایسه جداکننده‌های \english{Ni-HAB@CNT}, \english{CNT} و \english{Ni-HAB}. (ک) پایداری چرخه‌ای بلندمدت در \english{1C} که حفظ ظرفیت بالا با \english{Ni-HAB@CNT} را نشان می‌دهد. (ل) عملکرد باتری‌های \english{Li-S} با جداکننده‌های \english{Ni-HAB@CNT} در بارگذاری بالای گوگرد (\english{6.5 mg cm⁻²}) و نسبت پایین الکترولیت به گوگرد (\english{5 µL mg⁻¹}) در \english{0.2C} که از باتری‌های لیتیوم-یون تجاری بهتر عمل می‌کند. (م) عملکرد الکتروشیمیایی سلول‌های کیسه‌ای \english{Li-S} با بارگذاری گوگرد \english{2.5 mg cm⁻²} و نسبت \english{E/S} \english{6 µL mg⁻¹} در \english{0.1C}؛ تصویر داخلی یک عکس دیجیتال از سلول کیسه‌ای تازه با ولتاژ اولیه آن را نشان می‌دهد. (ن) پروفایل‌های شارژ-دشارژ سلول کیسه‌ای \english{Li-S} در طی چرخه‌های متعدد. تصویر داخلی یک عکس دیجیتال از سلول کیسه‌ای مبتنی بر \english{Ni-HAB@CNT} را نشان می‌دهد که ۶۷ لامپ \english{LED} را روشن می‌کند. بازتولید شده از مرجع ۷۵ با مجوز از \english{Wiley-VCH}، کپی‌رایت ۲۰۲۳.
    }}
    \label{fig:5}
\end{figure*}

\persian{
عملکرد الکتروشیمیایی در شکل‌های ۵(ز) تا (ن) تحلیل می‌شود. شکل ۵(ز) عملکرد نرخ را مقایسه می‌کند و نشان می‌دهد که \english{Ni-HAB@CNT} به طور قابل توجهی ظرفیت ویژه را در دانسیته‌های جریان مختلف (\english{0.2C} تا \english{3C}) بهبود می‌بخشد. پروفایل‌های شارژ-دشارژ در شکل ۵(ح) یک پلاتوی ولتاژ پایدار را نشان می‌دهند که بیانگر افزایش بهره‌برداری از گوگرد است. شکل ۵(ط) پتانسیل‌های پلاریزاسیون را کمی‌سازی کرده و نشان می‌دهد که \english{Ni-HAB@CNT} اورپتانسیل را کاهش داده و سینتیک واکنش را بهبود می‌بخشد.
}
\persian{
پایداری چرخه‌ای بلندمدت در شکل‌های ۵(ی) و (ک) برجسته شده است. در \english{0.2C}، \english{Ni-HAB@CNT} حفظ ظرفیت \english{85.2\%} را پس از ۲۰۰ چرخه نشان می‌دهد، در حالی که در \english{1C}، پایداری چرخه‌ای بلندمدت برتری را نسبت به \english{CNT} و \english{Ni-HAB} به تنهایی حفظ می‌کند. شکل ۵(ل) عملکرد را تحت بارگذاری بالای گوگرد (\english{6.5 mg cm⁻²}) با نسبت پایین الکترولیت به گوگرد (\english{5 µL mg⁻¹}) ارزیابی می‌کند و ظرفیت مساحتی برتری را در مقایسه با باتری‌های لیتیوم-یون تجاری نشان می‌دهد. شکل‌های ۵(م) و (ن) عملکرد سلول کیسه‌ای را برجسته می‌کنند، جایی که \english{Ni-HAB@CNT} ظرفیت پایدار \english{791 mAh g⁻¹} را پس از ۵۰ چرخه در \english{0.1C} حفظ می‌کند، حتی با استفاده حداقل از الکترولیت. تصویر داخلی در شکل ۵(ن) کاربرد عملی آن را نشان می‌دهد، زیرا یک سلول کیسه‌ای با موفقیت ۶۷ لامپ \english{LED} را روشن می‌کند. در مجموع، شکل ۵ نشان می‌دهد که چگونه \english{Ni-HAB@CNT} عملکرد باتری \english{Li-S} را از طریق رسانایی افزایش‌یافته، سرکوب برتر پلی‌سولفید و سینتیک واکنش تسریع‌شده بهبود می‌بخشد و راه را برای باتری‌های با چگالی انرژی بالا نسل بعد هموار می‌کند.$^{۷۵}$
}
\subsubsection*{\persian{۲.۲.۲. پیشرفت‌های اخیر در میزبان‌های کاتدی مشتق از \english{MOF}}}
\persian{
در یک مطالعه اخیر، یک میزبان گوگرد نوین، یک \english{MOF-TOC} (\english{MIL-101(Cr)}) که با خوشه‌های زیر-نانو اکسید تیتانیوم یکپارچه شده بود، توسعه یافت.
}

% --- End of content for page 8 ---
% --- SECTION 9: Page 9 of Translation ---
% --- Notes: Includes Figure 6 and discussion of redox-active and hierarchical MOFs.
% --------------------------------------------------------------------------

\persian{
تا اثر شاتل و سینتیک ردوکس کند را کاهش دهد.$^{۶۹}$ خوشه‌های زیر-نانو \english{Ti-O} (\english{TOCs}) که در حفرات \english{MOF} تعبیه شده‌اند، به عنوان جایگاه‌های کاتالیزوری عمل می‌کنند و به طور مؤثر پلی‌سولفیدهای لیتیوم را محبوس کرده و تبدیل آنها را از طریق هیبریداسیون اوربیتال \english{d-p} بین تیتانیوم و گوگرد افزایش می‌دهند. مشخصه‌یابی الکتروشیمیایی سیستماتیک و محاسبات \english{DFT} تأیید می‌کنند که ساختار \english{MOF-TOC} برهمکنش‌های قوی با پلی‌سولفیدهای لیتیوم را امکان‌پذیر می‌سازد، موانع انرژی فعال‌سازی برای تبدیل گوگرد را کاهش می‌دهد و انتقال بار کارآمد را تسهیل می‌کند. کاتد حاصل به ظرفیت مساحتی \english{7.9 mAh cm⁻²} با عمر چرخه‌ای طولانی تحت بارگذاری بالای گوگرد و شرایط الکترولیت کم‌مقدار دست می‌یابد.$^{۶۹}$
}
\persian{
لو و همکاران یک \english{MOF} رسانای مبتنی بر کبالت نوین (\english{Co-HTP}) را که بر روی گرافن کربوکسیل (\english{CG}) لنگر انداخته بود، برای افزایش واکنش‌های ردوکس پلی‌سولفید توسعه دادند.$^{۶۴}$ اصلاح ساختاری جایگاه‌های کبالت از واحدهای چهارضلعی \english{Co-N₄} به پیکربندی‌های نامتقارن \english{O-Co-N₄}، پلاریزاسیون اسپین و عدم استقرار الکترون \english{3d} را القا می‌کند که منجر به بهبود فعالیت کاتالیزوری و جذب پلی‌سولفید می‌شود. تحلیل الکتروشیمیایی نشان می‌دهد که کاتد \english{Co-HTP/CG} به ظرفیت برگشت‌پذیر \english{1137 mAh g⁻¹} در \english{0.1C}، عملکرد نرخ فوق‌العاده و پایداری بلندمدت طی ۵۰۰ چرخه در \english{1C} با زوال ظرفیت تنها \english{0.052\%} در هر چرخه دست می‌یابد.$^{۶۴}$
}
\persian{
وانگ و همکاران یک کاتالیزور مبتنی بر \english{MOF} با نقص دوگانه، \english{UiO-66D₂}، را معرفی کردند که دارای نقص‌های لینکر و خوشه برای افزایش سینتیک کاهش گوگرد است.$^{۷۲}$ این مهندسی نقص، تبدیل \english{S₈} به \english{Li₂S₄} را تسریع می‌کند در حالی که تبدیل \english{Li₂S₄} به \english{Li₂S} را بهبود می‌بخشد. سلول‌های \english{Li-S} با \english{UiO-66D₂} ظرفیت دشارژ اولیه \english{1087 mAh g⁻¹} را در \english{0.2C} پس از ۱۰۰ چرخه، همراه با ظرفیت مساحتی چشمگیر \english{10.4 mAh cm⁻²} در \english{0.05C} با بارگذاری گوگرد \english{12.9 mg cm⁻²} نشان دادند. پایداری چرخه‌ای بلندمدت نیز به طور قابل توجهی بهبود یافت و ظرفیت باقی‌مانده \english{785 mAh g⁻¹} پس از ۵۰۰ چرخه در \english{1C} بود.$^{۷۲}$
}

\subsubsection*{\persian{۲.۲.۳. پیشرفت‌های اخیر در جداکننده‌های عملکردی مبتنی بر \english{MOF}}}
\persian{
معرفی \english{MOF}های فعال در واکنش ردوکس (\english{RM-MOFs}) نیز منجر به بهبودهای قابل توجهی در عملکرد باتری \english{Li-S} شده است. یک مطالعه اخیر \english{MIL-101(Cr)} را با دی‌تیوتریتول (\english{DTT}) عامل‌دار کرد تا یک جداکننده \english{RM-MOF} ایجاد کند که پلی‌سولفیدهای لیتیوم را تثبیت کرده و از مهاجرت آنها به آند جلوگیری می‌کند.$^{۹۱}$ این استراتژی \english{RM-MOF} محبوس‌سازی پلی‌سولفید لیتیوم را افزایش داد و عملکرد الکتروشیمیایی برتری را امکان‌پذیر ساخت، از جمله افزایش ۱۵۰ درصدی ظرفیت دشارژ در نرخ‌های \english{C} بالا (\english{883.6 mAh g⁻¹} در \english{3C})، کاهش بیش از ۹۰ درصدی در نرخ زوال ظرفیت و ظرفیت مساحتی بالای \english{13.8 mAh cm⁻²} تحت شرایط الکترولیت کم‌مقدار.$^{۹۱}$ شکل ۶ یک تصویر جامع از چارچوب فلزی-آلی فعال در واکنش ردوکس (\english{RM-MOF}) و تأثیر آن بر عملکرد باتری لیتیوم-گوگرد (\english{LSB}) ارائه می‌دهد. جنبه‌های ساختاری در شکل‌های ۶(الف-و) به تصویر کشیده شده‌اند و نشان می‌دهند که چگونه \english{MIL-101(Cr)} به عنوان یک میزبان گوگرد متخلخل با حفره‌های مزوپور (۲۹-۳۴ آنگستروم) و جایگاه‌های فلزی باز (\english{OMS}) برای محبوس‌سازی مؤثر \english{Li₂Sₙ} عمل می‌کند. توزیع پتانسیل الکترواستاتیکی \english{DTT} در شکل ۶(ب) نواحی قطبی آن را برجسته می‌کند که نقش حیاتی در میانجی‌گری ردوکس ایفا می‌کنند. شکل ۶(ج) برهمکنش قوی بین \english{DTT} و \english{Li₂S₆} را نشان می‌دهد که با تغییر رنگ پس از واکنش مشخص می‌شود و توانایی آن در تثبیت \english{Li₂Sₙ} را تأیید می‌کند. مکانیزم پیوند \english{DTT} درون \english{RM-MOF} در شکل ۶(د) نشان داده شده است که توزیع یکنواخت را بدون اتلاف در طول چرخه‌زنی تضمین می‌کند. نمودار تفاوت چگالی بار در شکل ۶(ه) و پیکربندی جذب بهینه‌شده در شکل ۶(و) برهمکنش‌های الکترونیکی قوی بین \english{DTT} و جایگاه‌های فلزی کروم را بیشتر تأیید کرده و سینتیک ردوکس گوگرد افزایش‌یافته را تسهیل می‌کنند.$^{۹۱}$
}
\persian{
عملکرد الکتروشیمیایی کاتدهای مختلف گوگرد در شکل‌های ۶(ز-ی) به تصویر کشیده شده است، جایی که \english{RM-MOF} عملکرد برتری را در مقایسه با \english{S/rGO}، \english{S/rGO-DTT} و \english{S/rGO-MIL-101(Cr)} نشان می‌دهد. قابلیت نرخ در شکل ۶(ز) بالاترین ظرفیت ویژه را برای \english{S/rGO-RM-MOF} در تمام نرخ‌های \english{C} (\english{0.1C-3C}) نشان می‌دهد که سینتیک عالی آن را برجسته می‌کند. پایداری چرخه‌ای بلندمدت در \english{0.2C} در شکل ۶(ح) نشان می‌دهد که \english{RM-MOF} به طور قابل توجهی زوال ظرفیت را طی ۱۰۰ چرخه سرکوب می‌کند. شکل ۶(ط) یک سلول کیسه‌ای \english{2.6 Ah} را ارائه می‌دهد که بر مقیاس‌پذیری \english{LSBs} مبتنی بر \english{RM-MOF} برای کاربردهای عملی تأکید می‌کند. در نهایت، شکل ۶(ی) پروفایل‌های شارژ-دشارژ را نمایش می‌دهد و چگالی انرژی بالای \english{316.5 Wh kg⁻¹} را تأیید می‌کند که نشان‌دهنده بهره‌برداری بهینه از گوگرد و عملکرد پایدار است. در مجموع، این نتایج کارایی \english{RM-MOF} را در کاهش اثر شاتل، افزایش سینتیک ردوکس و بهبود پایداری چرخه‌ای تأیید می‌کنند و آن را به عنوان یک استراتژی امیدوارکننده برای نسل بعدی \english{LSBs} معرفی می‌کنند.$^{۹۱}$
}
\persian{
علاوه بر این، یک \english{MOF} کاتالیزوری متخلخل سلسله‌مراتبی (\english{HPC-MOF}) که توسط شیه و همکاران توسعه یافته، به طور مؤثری بین انتقال جرم و چگالی جایگاه کاتالیزوری تعادل برقرار می‌کند.$^{۶۲}$ برخلاف \english{MOF}های میکرو متخلخل متداول که از نفوذ کند پلی‌سولفید لیتیوم رنج می‌برند، \english{HPC-MOF} دارای ماکروحفره‌های بزرگ برای انتقال جرم افزایش‌یافته و میکروحفره‌ها با خوشه‌های کاتالیزوری با چگالی بالا برای جذب کارآمد پلی‌سولفید لیتیوم است. ارزیابی‌های الکتروشیمیایی افزایش \english{164.6\%} در ظرفیت دشارژ و کاهش \english{83.3\%} در زوال ظرفیت را طی ۵۰۰ چرخه تأیید می‌کند.$^{۶۲}$
}

\begin{figure*}[t]
    \centering
    \includegraphics[width=\textwidth]{example-image-c} % Placeholder for Figure 6
    \caption{\persian{
    \farsibold{شکل ۶.} تصویر شماتیک و ساختار مواد آماده‌شده: (الف) ساختار متخلخل \english{MIL-101(Cr)}. (ب) توزیع چگالی پتانسیل الکترواستاتیکی \english{DTT}. (ج) عکس‌های دیجیتال از تغییر رنگ محلول \english{Li₂S₆} قبل و بعد از افزودن پودر \english{DTT}. (د) پیوند \english{RM-MOF} با \english{DTT}. (ه) نمودار تفاوت چگالی بار برای \english{RM-MOF}. (و) پیکربندی جذب بهینه‌شده \english{DTT} بر روی \english{OMS} در \english{RM-MOF}. عملکردهای الکتروشیمیایی کاتدهای گوگرد آماده‌شده در \english{LSB}: (ز) عملکرد نرخ \english{LSBs} با کاتدهای مختلف در نرخ‌های \english{C} از ۰.۱ تا ۳. (ح) عملکرد چرخه‌ای \english{LSBs} با کاتدهای مختلف در \english{0.2C}. (ط) عکس دیجیتال از سلول کیسه‌ای \english{Li-S} در سطح آمپر-ساعت. (ی) پروفایل‌های شارژ/دشارژ سلول کیسه‌ای \english{Li-S} در سطح آمپر-ساعت با کاتد \english{S/rGO-RM-MOF}. بازتولید شده از مرجع ۹۱ با مجوز از انجمن سلطنتی شیمی، کپی‌رایت ۲۰۲۵.
    }}
    \label{fig:6}
\end{figure*}

% --- End of content for page 9 ---
% --- SECTION 10: Page 10 of Translation ---
% --- Notes: Includes Figure 7, concludes MOFs section, and introduces COFs.
% --------------------------------------------------------------------------

\persian{
رزاق و همکاران پیشرفت قابل توجهی را از طریق توسعه یک جداکننده دوکاره متشکل از \english{ZIF-8} دوپ‌شده با آهن (\english{Fe-ZIF-8}) پوشش داده شده بر روی یک غشای پلی‌پروپیلن (\english{PP}) گزارش کردند.$^{۹۲}$ \english{Fe-ZIF-8} گل-مانند در آب و در دمای ۳۵ درجه سانتی‌گراد با استفاده از یک فرآیند مقرون‌به‌صرفه و سازگار با محیط زیست سنتز شد. این \english{MOF} دو فلزی، هنگامی که بر روی یک جداکننده \english{PP} پوشش داده شد، غربال‌گری در مقیاس نانو برای مسدود کردن فیزیکی \english{LiPS} فراهم می‌کند و همزمان تبدیل الکتروشیمیایی آن را از طریق جایگاه‌های فعال آهن کاتالیز می‌کند. جداکننده \english{Fe-ZIF-8/PP} حاصل، به طور مؤثری مهاجرت پلی‌سولفید را برای بیش از ۱۲ ساعت در آزمون‌های نفوذ بصری سرکوب کرد و عملکرد الکتروشیمیایی را به طور قابل توجهی بهبود بخشید و به ۱۰۰۰ چرخه شارژ-دشارژ پایدار با تنها \english{0.05\%} افت ظرفیت در هر چرخه در \english{0.5C} دست یافت. علاوه بر این، سلول‌های متقارن \english{Li || Li} با \english{Fe-ZIF-8/PP} شار یونی یکنواخت و رسوب بدون دندریت را برای بیش از ۴۰۰۰ ساعت نشان دادند.
}
\persian{
شکل ۷ کارایی یک جداکننده \english{Fe-ZIF-8/PP} را در سرکوب مهاجرت پلی‌سولفید در باتری‌های \english{Li-S}، یک چالش کلیدی که بر عملکرد بلندمدت آنها تأثیر می‌گذارد، نشان می‌دهد.$^{۹۲}$ شکل ۷(الف) یک مقایسه شماتیک بین یک جداکننده \english{PP} متداول و جداکننده اصلاح‌شده با \english{Fe-ZIF-8} ارائه می‌دهد. در مورد جداکننده \english{Celgard} بکر (\english{PP})، پلی‌سولفیدهای لیتیوم حل‌شده به راحتی از طریق جداکننده متخلخل از کاتد گوگرد به آند لیتیوم مهاجرت می‌کنند و منجر به «اثر شاتل» شناخته‌شده‌ای می‌شوند که باعث اتلاف ماده فعال، بازده کولمبی ضعیف و تخریب آند لیتیوم می‌شود. در مقابل، جداکننده \english{Fe-ZIF-8/PP} نه تنها پلی‌سولفیدها را از طریق میکروحفره‌های یکنواخت خود به صورت فیزیکی مسدود می‌کند، بلکه آنها را از طریق جایگاه‌های فعال آهن به صورت شیمیایی جذب و به طور کاتالیزوری تبدیل می‌کند، در نتیجه رفتار شاتل را به طور قابل توجهی کاهش می‌دهد در حالی که همچنان امکان انتقال کارآمد یون لیتیوم را فراهم می‌کند. شکل‌های ۷(ب-د) یک آزمون نفوذ بصری پلی‌سولفید را با استفاده از یک سلول نوع \english{H} نشان می‌دهند، جایی که محفظه چپ حاوی محلول \english{0.1 M Li₂S₆} و محفظه راست حاوی الکترولیت خالی است. در طول ۱۲ ساعت، جداکننده \english{PP} در شکل ۷(ب) امکان نفوذ سریع پلی‌سولفید را فراهم می‌کند که از تغییر رنگ محفظه راست در عرض چند دقیقه مشهود است. جداکننده \english{ZIF-8/PP} در شکل ۷(ج) تا حدی نفوذ را به تأخیر می‌اندازد اما همچنان تا ۶ ساعت نشتی قابل مشاهده را مجاز می‌داند. قابل توجه است که جداکننده \english{Fe-ZIF-8/PP} در شکل ۷(د) به طور مؤثری مهاجرت پلی‌سولفید را برای بیش از ۱۲ ساعت مسدود می‌کند و محفظه راست تقریباً شفاف باقی می‌ماند. این سرکوب قوی به اثر هم‌افزایی دوپینگ آهن و ساختار متخلخل \english{MOF} نسبت داده می‌شود. در مجموع، شکل ۷ به صورت بصری و مکانیکی عملکرد برتر جداکننده‌های اصلاح‌شده با \english{Fe-ZIF-8} را در کاهش شاتل پلی‌سولفید تأیید می‌کند و پتانسیل آنها را در پیشبرد باتری‌های \english{Li-S} بادوام و با کارایی بالا برجسته می‌نماید.
}

\begin{figure*}[t]
    \centering
    \includegraphics[width=\textwidth]{example-image-b} % Placeholder for Figure 7
    \caption{\persian{
    \farsibold{شکل ۷.} سرکوب مهاجرت پلی‌سولفید با استفاده از جداکننده‌های اصلاح‌شده با \english{MOF} در باتری‌های \english{Li-S}. بازتولید شده از مرجع ۹۲ با مجوز از \english{Wiley-VCH}، کپی‌رایت ۲۰۲۴. (الف) تصویر شماتیک مقایسه‌ای بین انحلال و شاتل پلی‌سولفید از طریق یک جداکننده \english{Celgard} بکر (\english{PP}) در مقابل جذب و تبدیل کاتالیزوری با استفاده از یک جداکننده اصلاح‌شده با \english{MOF} دو فلزی (\english{Fe-ZIF-8}). در حالی که جداکننده \english{PP} امکان نفوذ کنترل‌نشده پلی‌سولفیدهای لیتیوم (\english{Li₂Sₓ}) را فراهم می‌کند که منجر به اثرات شاتل شدید می‌شود، جداکننده اصلاح‌شده با \english{Fe-ZIF-8} به صورت فیزیکی پلی‌سولفیدها را مسدود و به صورت شیمیایی جذب می‌کند و پایداری چرخه‌ای را افزایش می‌دهد. (ب)-(د) آزمون‌های نفوذ بصری پلی‌سولفید در سلول‌های نوع \english{H} با استفاده از سه جداکننده مختلف در فواصل زمانی افزایشی (۱۰ دقیقه، ۳ ساعت، ۶ ساعت و ۱۲ ساعت): (ب) جداکننده \english{PP} نفوذ سریع و تغییر رنگ الکترولیت خالی را نشان می‌دهد، (ج) جداکننده \english{ZIF-8/PP} سرکوب جزئی با نفوذ تأخیری اما قابل مشاهده را نشان می‌دهد، و (د) جداکننده \english{Fe-ZIF-8/PP} سرکوب قوی را با مهاجرت ناچیز پلی‌سولفید حتی پس از ۱۲ ساعت نشان می‌دهد.
    }}
    \label{fig:7}
\end{figure*}

\persian{
به طور خلاصه، پیشرفت \english{MOFs} در باتری‌های \english{Li-S} فرصتی تحول‌آفرین برای سیستم‌های ذخیره‌سازی انرژی ارائه می‌دهد. این مواد قابلیت‌های قابل توجهی در محبوس‌سازی گوگرد، کاتالیز پلی‌سولفید لیتیوم و بهبود سینتیک واکنش نشان داده‌اند. با این حال، چالش‌هایی مانند پیچیدگی سنتز، پایداری تحت شرایط چرخه‌زنی و رسانایی محدود همچنان باید برای تحقق کامل باتری‌های \english{Li-S} مبتنی بر \english{MOF} برای کاربردهای تجاری برطرف شوند. تحقیقات آینده باید بر بهینه‌سازی رسانایی \english{MOF}، بهره‌برداری از ساختارهای حفرات سلسله‌مراتبی و کاوش برهمکنش‌های جدید فلز-لیگاند برای افزایش فعالیت کاتالیزوری متمرکز شود. علاوه بر این، روش‌های سنتز مقیاس‌پذیر که یکپارچگی ساختاری و عملکرد \english{MOFs} را حفظ می‌کنند، برای اجرای عملی آنها ضروری است. اگر این چالش‌ها برطرف شوند، باتری‌های \english{Li-S} مبتنی بر \english{MOF} می‌توانند به طور قابل توجهی به توسعه فناوری‌های ذخیره‌سازی انرژی با چگالی انرژی بالا و عمر چرخه‌ای طولانی کمک کرده و کاربرد آنها را در وسایل نقلیه الکتریکی و سیستم‌های ذخیره‌سازی در مقیاس شبکه پیش ببرند.
}

\subsection*{\persian{۲.۳. چارچوب‌های آلی کووالانسی (\english{COFs}) برای باتری‌های پیشرفته \english{Li-S}}}
\persian{
کاربرد چارچوب‌های آلی کووالانسی (\english{COFs}) و مشتقات آنها در باتری‌های \english{Li-S} با کارایی بالا به دلیل خواص ساختاری و شیمیایی منحصربه‌فردشان توجه فزاینده‌ای را به خود جلب کرده است.$^{۹۳-۱۲۲}$ \english{COFs} به عنوان مواد مؤثری برای کاهش چالش‌های ذاتی باتری‌های لیتیوم-گوگرد (\english{LSBs}) با عمل کردن به عنوان میزبان گوگرد، اصلاح‌کننده‌های جداکننده و اجزای الکترولیت ظهور کرده‌اند. این مرور پیشرفت‌های اخیر در \english{COFs} خالص، مشتقات \english{COF}،
}

% --- End of content for page 10 ---
% --- SECTION 11: Page 11 of Translation ---
% --- Notes: Includes Figure 8 and focuses on COF-based separators.
% --------------------------------------------------------------------------

\persian{
و کامپوزیت‌های \english{COF} در پایدارسازی کاتدهای گوگرد، افزایش پایداری چرخه‌ای و بهبود طراحی الکترولیت را بررسی می‌کند. استراتژی‌های مختلفی، از جمله اصلاحات ساختاری، دوپینگ هترواتم و ادغام نمک‌های لیتیوم، برای بهینه‌سازی عملکرد \english{LSB} تحلیل می‌شوند. علاوه بر این، محاسبات نظری و مطالعات تجربی پتانسیل \english{COFs} را در پرداختن به محدودیت‌های بنیادین \english{LSBs} برجسته می‌کنند. این مرور با بحث در مورد چشم‌اندازهای مواد باتری مبتنی بر \english{COF} و پتانسیل تجاری‌سازی آنها در سیستم‌های ذخیره‌سازی انرژی نسل بعد به پایان می‌رسد.
}
\subsubsection*{\persian{۲.۳.۱. پیشرفت‌ها در اصلاح‌کننده‌های جداکننده مبتنی بر \english{COF}}}
\persian{
پیشرفت‌های اخیر در \english{COFs} مسیرهای نوینی برای پرداختن به محدودیت‌های مرتبط با باتری‌های \english{Li-S} فراهم کرده‌اند. یک مطالعه قابل توجه، یک چارچوب آلی کووالانسی پورفیرین مبتنی بر \english{EDOT} (\english{EDOT-Por-COF}) را به عنوان یک جداکننده عملکردی برای باتری‌های \english{Li-S} معرفی کرد.$^{۱۲۳}$ گنجاندن واحد ۳,۴-اتیلن‌دی‌اکسی‌تیوفن (\english{EDOT}) ساختار الکترونیکی اطراف جایگاه‌های کاتالیزوری کبالت (\english{Co}) را افزایش می‌دهد و در نتیجه تبدیل پلی‌سولفید را تسریع کرده و اثر شاتل را کاهش می‌دهد. تحقیقات تجربی و محاسباتی تأیید می‌کنند که ریزمحیط غنی از الکترون ایجاد شده توسط ادغام \english{EDOT} به طور قابل توجهی میل ترکیبی گوگرد و فعالیت کاتالیزوری را بهبود می‌بخشد. \english{EDOT-Por-COF} عملکرد الکتروشیمیایی استثنایی، از جمله ظرفیت دشارژ بالای \english{1585.9 mAh g⁻¹} در \english{0.1C}، پایداری چرخه‌ای قابل توجه با نرخ زوال تنها \english{0.031\%} در هر چرخه طی ۲۰۰۰ چرخه در \english{1C}، و ظرفیت نرخ بالای \english{763.9 mAh g⁻¹} در \english{5C} را نشان می‌دهد که از جداکننده‌های متداول و \english{TA-Por-COF} بهتر عمل می‌کند. علاوه بر این، ادغام \english{EDOT} انتقال برتر یون لیتیوم را تسهیل کرده و سطح آند لیتیوم را پایدار می‌کند و در نتیجه تشکیل «لیتیوم مرده» را کاهش می‌دهد.$^{۱۲۳}$
}
\persian{
یک رویکرد پیشین، استفاده از نانوورقه‌های چارچوب آلی کووالانسی لیتیومی شده (\english{Li-CONs}) را به عنوان یک جداکننده اصلاح‌شده برای افزایش انتقال یون لیتیوم و کاهش موانع نفوذ معرفی کرد.$^{۱۲۴}$ لایه \english{Li-CON} دارای جایگاه‌های لیتیومی شده منظمی است که به طور قابل توجهی انتقال یون لیتیوم را افزایش می‌دهد در حالی که همزمان به عنوان جایگاه‌های به دام‌اندازی پلی‌سولفید عمل می‌کند. ارزیابی‌های عملکرد الکتروشیمیایی نشان داد که سلول‌های \english{Li-S} با جداکننده‌های اصلاح‌شده با \english{Li-CON} پایداری چرخه‌ای برتری را با ظرفیت ویژه اولیه \english{982 mAh g⁻¹} در \english{1C} و حفظ \english{645 mAh g⁻¹} پس از ۶۰۰ چرخه، با نرخ زوال تنها \english{0.057\%} در هر چرخه نشان می‌دهند.$^{۱۲۴}$ جداکننده \english{Li-CON} همچنین اورپتانسیل‌های پایین‌تر و پلاریزاسیون کاهش‌یافته را نشان داد که منجر به بهبود قابلیت‌های نرخ شد. مطالعات محاسباتی بیشتر تأیید کردند که جایگاه‌های لیتیومی شده به طور مؤثری مانع فعال‌سازی برای مهاجرت یون لیتیوم را کاهش می‌دهند و رسانایی سریع یون را تضمین می‌کنند. سینتیک تبدیل پلی‌سولفید به دلیل وجود نانوکانال‌های رسانای یون لیتیوم تسریع شد و در نتیجه بهره‌برداری از گوگرد را افزایش داد. مطالعات تجربی، از جمله طیف‌سنجی رامان و آزمون‌های نفوذ پلی‌سولفید نوع \english{H}، توانایی \english{Li-CONs} را در سرکوب نفوذ پلی‌سولفید و افزایش واکنش‌های ردوکس \english{Li-S} تأیید کردند.
}

\begin{figure*}[t]
    \centering
    \includegraphics[width=0.9\textwidth]{example-image-b} % Placeholder for Figure 8
    \caption{\persian{
    \farsibold{شکل ۸.} (الف) نمایش شماتیک کانال‌های غربال یونی در \english{COF}های کاتیونی که نقش آنها را در تسهیل انتقال \english{Li⁺} و دافعه الکترواستاتیکی \english{LiPS}های آنیونی نشان می‌دهد. ساختارهای شیمیایی \english{COF}های کاتیونی: (ب) \english{COF-Br}، (ج) \english{COF-BF₄} و (د) \english{COF-TFSI}. (ه) پتانسیل الکترواستاتیکی و قطبیت مولکولی \english{COF-Br}، \english{COF-BF₄} و \english{COF-TFSI}. بازتولید شده از مرجع ۱۲۵ با مجوز از \english{Elsevier Ltd.}، کپی‌رایت ۲۰۲۴.
    \newline
    شکل ۸ طراحی و عملکرد \english{COF}های کاتیونی را به عنوان غشاهای غربال یونی برای باتری‌های \english{Li-S} نشان می‌دهد. شکل ۸(الف) مفهوم کانال غربال یونی را نشان می‌دهد: نمودار شماتیک در سمت چپ نشان می‌دهد که چگونه غشای \english{COF} کاتیونی به عنوان یک غربال یونی عمل می‌کند. نانوکانال‌های یک‌بعدی امکان انتقال انتخابی یون‌های \english{Li⁺} را فراهم می‌کنند و یک «بزرگراه \english{Li⁺}» ایجاد می‌کنند. همزمان، \english{Li₂Sₙ} با بار منفی دفع می‌شود و از اثر شاتل، که یک مسئله بزرگ در باتری‌های \english{Li-S} است، جلوگیری می‌کند. معرفی آنیون‌های متقابل مختلف (\english{Br⁻, BF₄⁻, TFSI⁻}) خواص الکترواستاتیکی غشای \english{COF} را تغییر می‌دهد. شکل‌های ۸(ب-د) ساختارهای شیمیایی \english{COF}های کاتیونی را نشان می‌دهند: شکل ۸(ب) ساختار شیمیایی \english{COF-Br} را نشان می‌دهد: این ساختار حاوی آنیون‌های برمید (\english{Br⁻}) است و یک برهمکنش الکترواستاتیکی پایه‌ای را فراهم می‌کند. شکل ۸(ج) ساختار شیمیایی \english{COF-BF₄} را نشان می‌دهد: این ساختار از تترافلوروبورات (\english{BF₄⁻}) استفاده می‌کند که اثرات الکترواستاتیکی را کمی افزایش می‌دهد. شکل ۸(د) ساختار شیمیایی \english{COF-TFSI} را نشان می‌دهد: این ساختار آنیون \english{TFSI⁻} را در خود جای داده و قطبیت الکترواستاتیکی غشا را به طور قابل توجهی افزایش می‌دهد که منجر به انتقال بهتر \english{Li⁺} و دافعه قوی‌تر پلی‌سولفید می‌شود. شکل ۸(ه) پتانسیل الکترواستاتیکی و قطبیت مولکولی را نشان می‌دهد: تصاویر پتانسیل الکترواستاتیکی با رنگ‌بندی، قطبیت مولکولی را نشان می‌دهند، جایی که قطبیت بالاتر (نواحی قرمز) انتخاب‌پذیری یونی را افزایش می‌دهد. در میان سه \english{COF}، \english{COF-TFSI} بالاترین قطبیت الکترواستاتیکی را نشان می‌دهد و آن را در بهبود رسانایی \english{Li⁺} و مسدود کردن پلی‌سولفیدها مؤثرتر می‌سازد.
    }}
    \label{fig:8}
\end{figure*}

\persian{
در تحقیقات پیشرفته اخیر، نویسندگان یک رویکرد نوآورانه با استفاده از \english{COF}های کاتیونی متصل به وینیلن و با قطبیت الکترواستاتیکی تنظیم‌شده به عنوان یک غشای غربال یونی برای بهبود عملکرد باتری \english{Li-S} پیشنهاد کردند.$^{۱۲۵}$ این مطالعه یک استراتژی جدید اصلاح جداکننده را معرفی می‌کند که قطبیت الکترواستاتیکی را با ادغام آنیون‌های متقابل مختلف (\english{Br⁻, BF₄⁻, TFSI⁻}) تنظیم می‌کند. در میان اینها، \english{COF-TFSI} به دلیل دافعه الکترواستاتیکی قوی در برابر پلی‌سولفیدهای با بار منفی و در عین حال افزایش همزمان انتقال یون لیتیوم، عملکرد برتری را نشان داد. جداکننده اصلاح‌شده با \english{COF-TFSI} به طور قابل توجهی اثر شاتل را سرکوب کرد و منجر به پایداری چرخه‌ای بلندمدت با نرخ زوال ظرفیت بسیار پایین تنها \english{0.041\%} در هر چرخه طی ۱۰۰۰ چرخه در \english{1C} شد. علاوه بر این، غشا بهره‌برداری بالای گوگرد را حفظ کرد و به ظرفیت مساحتی چشمگیر \english{4.23 mAh cm⁻²} با بارگذاری گوگرد \english{8.81 mg cm⁻²} دست یافت.
}

% --- End of content for page 11 ---
% --- SECTION 12: Page 12 of Translation ---
% --- Notes: Covers COFs as hosts/additives and introduces MXenes.
% --------------------------------------------------------------------------

\subsubsection*{\persian{۲.۳.۲. \english{COF}ها به عنوان میزبان گوگرد}}
\persian{
دوان و همکاران (۲۰۲۱) نانوورقه‌های چارچوب آلی کووالانسی پلی‌ایمیدی فوق‌نازک (\english{PI-CONs}) را به عنوان یک ماده میزبان گوگرد مقیاس‌پذیر و بسیار مؤثر توسعه دادند.$^{۱۱۳}$ نویسندگان \english{COF}های پلی‌ایمیدی با ساختار لایه‌ای (\english{PI-COFs}) را با استفاده از یک روش سل-و-حرارتی سنتز کرده و آنها را از طریق لایه‌برداری با کمک حلال به نانوورقه‌های دو بعدی فوق‌نازک (ضخامت حدود ۱.۲ نانومتر، اندازه جانبی حدود ۶ میکرومتر) تبدیل کردند. این نانوورقه‌ها سطح ویژه بالا، برهمکنش‌های قوی اکسیژن-لیتیوم و جذب عالی پلی‌سولفید را از خود نشان می‌دهند و در نتیجه اثر شاتل را کاهش داده و بهره‌برداری از گوگرد را بهبود می‌بخشند. کامپوزیت \english{PI-CONs/S} عملکرد الکتروشیمیایی برتری را نشان داد و به ظرفیت ویژه بالای \english{1330 mAh g⁻¹} در \english{0.1C}، قابلیت نرخ عالی (\english{620 mAh g⁻¹} در \english{4C}) و حفظ ظرفیت ۹۶٪ پس از ۱۰۰ چرخه در \english{0.2C} دست یافت که از بیشتر میزبان‌های گوگرد آلی گزارش‌شده فراتر بود.$^{۱۱۳}$
}
\persian{
\english{COF}های سه‌بعدی مزدوج (\english{3D}) به دلیل تخلخل بالا، عدم استقرار گسترده الکترون‌های \english{π} و خواص عالی انتقال بار، به عنوان مواد امیدوارکننده‌ای برای \english{LSBs} ظهور کرده‌اند. با این حال، توسعه \english{COF}های سه‌بعدی به دلیل در دسترس بودن محدود بلوک‌های سازنده مزدوج، چالش‌برانگیز بوده است. لیو و همکاران (۲۰۲۲) با موفقیت دو \english{COF} سه‌بعدی نوین، \english{3D-scu-COF-1} و \english{3D-scu-COF-2}، را با متراکم کردن یک بلوک سازنده ۸-متصلی مبتنی بر پنتیپتیسن با تقارن \english{D₂h} (\english{DMOPTP}) با لینکرهای ۴-متصلی مربع-مسطح سنتز کردند که منجر به چارچوب‌های بسیار متخلخل با توپولوژی‌های \english{scu} دوگانه درهم‌تنیده شد.$^{۱۲۶}$ این مواد سطح ویژه استثنایی (۲۳۴۰ و ۱۶۰۲ مترمربع بر گرم) و رسانایی ذاتی بالا (\english{3.2-3.5 × 10⁻⁵ S cm⁻¹}) را از خود نشان می‌دهند که جذب لیتیوم-پلی‌سولفید و انتقال بار را در \english{LSBs} افزایش می‌دهد. ارزیابی‌های الکتروشیمیایی نشان داد که این \english{COF}ها، هنگامی که به عنوان میزبان گوگرد استفاده می‌شوند، به ظرفیت‌های ویژه بالا (\english{1035-1155 mAh g⁻¹} در \english{0.2C})، عملکرد نرخ عالی (\english{713-757 mAh g⁻¹} در \english{5.0C}) و پایداری چرخه‌ای فوق‌العاده (۷۱-۸۳٪ حفظ ظرفیت پس از ۵۰۰ چرخه در \english{2.0C}) دست یافتند و از بیشتر کاتدهای \english{LSB} آلی گزارش‌شده بهتر عمل کردند.$^{۱۲۶}$
}

\subsubsection*{\persian{۲.۳.۳. \english{COF}ها برای اصلاح الکترولیت}}
\persian{
مطالعات اخیر \english{COF}ها را به عنوان افزودنی‌های الکترولیت برای افزایش انتقال یون لیتیوم و بهبود عملکرد کاتد گوگرد بررسی کرده‌اند. لیو و همکاران یک \english{COF} تثبیت‌شده با تک یون کبالت (\english{USTB-27-Co}) را معرفی کردند که به طور قابل توجهی عملکرد الکتروشیمیایی \english{LSBs} را با فراهم کردن جذب قوی پلی‌سولفید و فعالیت کاتالیزوری افزایش می‌دهد.$^{۱۰۶}$ فلزدار کردن پساسنتزی \english{USTB-27} با کبالت، توانایی آن را در لنگر انداختن پلی‌سولفیدهای لیتیوم و کاهش مانع انرژی برای تبدیل آنها بهبود بخشید و منجر به بهبود قابلیت نرخ و پایداری چرخه‌ای شد. آزمون‌های الکتروشیمیایی نشان دادند که کاتد \english{S@USTB-27-Co} ظرفیت دشارژ اولیه \english{1063 mAh g⁻¹} را در \english{0.1C} نشان داد و ۶۱٪ ظرفیت را در \english{5.0C} حفظ کرد، که به طور قابل توجهی از همتای غیر-کبالتی خود بهتر بود.$^{۱۰۶}$
}
\persian{
هو و همکاران یک \english{COF} دهنده-پذیرنده متصل به اگزازول (\english{BTT-DABD}) را سنتز کردند که پایداری شیمیایی و فعالیت الکتروکاتالیزوری برتری نسبت به \english{COF}های متصل به ایمین سنتی نشان داد.$^{۱۰۸}$ هنگامی که به عنوان یک اصلاح‌کننده جداکننده به کار گرفته شد، \english{BTT-DABD} به طور قابل توجهی بهره‌برداری از گوگرد را بهبود بخشید و به ظرفیت دشارژ اولیه بالای \english{1383 mAh g⁻¹} در \english{0.1C}، قابلیت نرخ عالی \english{674 mAh g⁻¹} در \english{5C} و پایداری چرخه‌ای فوق‌العاده با تنها \english{0.061\%} زوال ظرفیت در هر چرخه طی ۵۰۰ چرخه در \english{3C} دست یافت.$^{۱۰۸}$
}
\persian{
علی‌رغم پیشرفت قابل توجه در \english{LSBs} مبتنی بر \english{COF}، چندین چالش قبل از تجاری‌سازی گسترده آنها باقی مانده است. تحقیقات آینده باید بر بهینه‌سازی فرآیندهای سنتز، مقیاس‌بندی تولید و کاوش اصلاحات پساسنتزی اضافی برای افزایش پایداری شیمیایی، رسانایی و سازگاری با الکترودهای گوگرد با بارگذاری بالا متمرکز شود. علاوه بر این، نگرانی‌های مربوط به پایداری چرخه‌ای بلندمدت و ایمنی مرتبط با مواد مبتنی بر \english{COF} باید مورد توجه قرار گیرد. پیشرفت‌ها در مدل‌سازی محاسباتی، همراه با تکنیک‌های تجربی نوآورانه، روابط ساختار-خاصیت \english{COF}ها را در باتری‌های \english{Li-S} بیشتر روشن خواهد کرد. با افزایش تقاضا برای راه‌حل‌های ذخیره‌سازی انرژی نسل بعد، \english{COF}ها مسیر امیدوارکننده‌ای به سوی دستیابی به \english{LSBs} با کارایی بالا و طولانی‌مدت ارائه می‌دهند.
}

\subsection*{\persian{۲.۴. \english{MXene}ها برای باتری‌های پیشرفته \english{Li-S}}}
\persian{
\english{MXene}ها، دسته‌ای از کاربیدها، نیتریدها و کربونیتریدهای فلزات واسطه دو بعدی (\english{2D})، به دلیل رسانایی الکتریکی استثنایی، شیمی سطح قابل تنظیم و پایداری مکانیکی بالا، به عنوان مواد امیدوارکننده‌ای برای باتری‌های پیشرفته لیتیوم-گوگرد (\english{Li-S}) توجه قابل توجهی را به خود جلب کرده‌اند.$^{۱۲۷-۱۴۶}$ این خواص منحصربه‌فرد به \english{MXene}ها اجازه می‌دهد تا به طور مؤثری به چالش‌های حیاتی در فناوری باتری \english{Li-S}، از جمله سینتیک ردوکس کند واکنش‌های تبدیل گوگرد، انحلال پلی‌سولفید و پایداری چرخه‌ای ضعیف، رسیدگی کنند. رسانایی فلزی آنها انتقال سریع الکترون را تضمین کرده، انتقال بار کارآمد را تسهیل و بهره‌برداری از گوگرد را بهبود می‌بخشد. علاوه بر این، گروه‌های عاملی سطحی آنها (مانند \english{-OH, -O, -F}) نقش حیاتی در شیمی‌جذب پلی‌سولفیدهای لیتیوم (\english{LiPSs}) ایفا می‌کنند و در نتیجه اثر شاتل را کاهش داده و افت ظرفیت را کم می‌کنند. علاوه بر این، \english{MXene}ها انعطاف‌پذیری ساختاری از خود نشان می‌دهند و طراحی معماری‌های مختلف الکترود، مانند فیلم‌های خود ایستا، کامپوزیت‌های هیبریدی و لایه‌های میانی را امکان‌پذیر می‌سازند که پایداری چرخه‌ای و قابلیت نرخ را افزایش می‌دهند. با ادغام \english{MXene}ها در کاتدهای گوگرد، لایه‌های میانی یا جداکننده‌های اصلاح‌شده، محققان بهبودهای قابل توجهی در عملکرد باتری \english{Li-S}، از جمله بارگذاری بالای گوگرد، عمر چرخه‌ای طولانی و قابلیت نرخ افزایش‌یافته را نشان داده‌اند. پیشرفت‌های آینده در مهندسی سطح، طراحی هتروساختار و استراتژی‌های عامل‌دار کردن، نقش \english{MXene}ها را در باتری‌های \english{Li-S} نسل بعد بیشتر ارتقا خواهد داد و آنها را به عنوان مواد حیاتی برای کاربردهای با چگالی انرژی بالا معرفی خواهد کرد.
}
\subsubsection*{\persian{۲.۴.۱. \english{MXene}های \english{Ti₃C₂Tₓ} اکسید شده با \english{CO₂} برای سرکوب اثر شاتل}}
\persian{
یکی از چالش‌های اصلی در باتری‌های \english{Li-S}، انحلال \english{LiPSs} در الکترولیت‌های مایع است که منجر به زوال ظرفیت و بازده کولمبی پایین می‌شود. برای پرداختن به این مسئله، \english{MXene}های \english{Ti₃C₂Tₓ} اکسید شده با \english{CO₂} به عنوان یک راه‌حل امیدوارکننده ظهور کرده‌اند.$^{۱۴۳}$ لی و همکاران یک رویکرد نوین با اکسید کردن \english{MXene}های لایه‌برداری شده با \english{CO₂} در دمای ۹۰۰ درجه سانتی‌گراد معرفی کردند که نانوکریستال‌های روتیل-\english{TiO₂} را بر روی ورقه‌های دو بعدی آنها تشکیل می‌دهد و در نتیجه خواص فیزیکی و الکتروشیمیایی را افزایش می‌دهد.$^{۱۴۳}$ شکل ۹ فرآیند سنتز، عملکرد الکتروشیمیایی و قابلیت جذب پلی‌سولفید \english{MXene}های \english{Ti₃C₂Tₓ} اکسید شده با \english{CO₂} (\english{Oxi-d-MXenes}) برای باتری‌های \english{Li-S} را نشان می‌دهد.$^{۱۴۳}$ این شکل نشان می‌دهد...
}

% --- End of content for page 12 ---
% --- SECTION 13: Page 13 of Translation ---
% --- Notes: Includes Figure 9 and begins analysis of Figure 10.
% --------------------------------------------------------------------------

\persian{
که چگونه \english{MXene}های \english{Oxi-d} عملکرد باتری \english{Li-S} را با پایدارسازی کاتدهای گوگرد و به دام انداختن پلی‌سولفیدها به طور قابل توجهی بهبود می‌بخشند.$^{۱۴۳}$
}

\begin{figure*}[t]
    \centering
    \includegraphics[width=\textwidth]{example-image-b} % Placeholder for Figure 9
    \caption{\persian{
    \farsibold{شکل ۹.} (الف) نمایش شماتیک از فرآیند سنتز \english{Oxi-d-MXenes}. کامپوزیت \english{S/Oxi-d-MXene900} با نفوذ مذاب گوگرد تهیه شد. (ب) جداکننده اصلاح‌شده با \english{Oxi-d-MXene600} با استفاده از یک روش ساده فیلتراسیون خلاء بر روی یک جداکننده الیاف شیشه (\english{GF}) ساخته شد. (ج) منحنی‌های شارژ/دشارژ گالوانواستاتیک الکترود \english{S/Oxi-d-MXene900} با یک جداکننده اصلاح‌نشده و اصلاح‌شده در دانسیته جریان \english{0.2C}. (د) عملکرد چرخه‌ای الکترود \english{S/Oxi-d-MXene900} با جداکننده‌های اصلاح‌نشده و اصلاح‌شده طی ۱۰۰ چرخه در \english{0.2C}. (ه) عملکرد نرخ الکترود \english{S/Oxi-d-MXene900} با جداکننده اصلاح‌شده در دانسیته‌های جریان مختلف. (و) عکس‌های دیجیتال که قابلیت جذب مواد مختلف (\english{Li₂S₆} solution, \english{d-MXene}, \english{Oxi-d-MXene600} و \english{Oxi-d-MXene900}) را پس از ۲۴ ساعت نشان می‌دهند. (ز) عملکرد چرخه‌ای بلندمدت الکترود \english{S/Oxi-d-MXene900} با جداکننده اصلاح‌شده در \english{1C} طی ۳۰۰ چرخه. بازتولید شده از مرجع ۱۴۳ با مجوز از انجمن شیمی آمریکا، کپی‌رایت ۲۰۲۰.
    }}
    \label{fig:9}
\end{figure*}

\persian{
تأثیر \english{Oxi-d-MXenes} بر بهبود پایداری کاتد گوگرد و سرکوب اثر شاتل پلی‌سولفیدهای لیتیوم (\english{LPSs}). شکل ۹(الف) سنتز \english{Oxi-d-MXenes} را برای کاتدهای گوگرد و جداکننده‌های اصلاح‌شده نشان می‌دهد. نمودار شماتیک تبدیل فاز \english{MAX} \english{Ti₃AlC₂} به \english{MXene}های لایه‌برداری شده (\english{d-MXenes}) را از طریق یک فرآیند اچینگ تشریح می‌کند. سپس \english{d-MXenes} تحت اکسیداسیون با \english{CO₂} در دمای ۹۰۰ درجه سانتی‌گراد قرار گرفتند و \english{Ti₃C₂Tₓ} به نانوکریستال‌های روتیل-\english{TiO₂} که در ورقه‌های لایه‌ای \english{MXene} تعبیه شده بودند، تبدیل شد. این اکسیداسیون توانایی ماده را برای جذب شیمیایی و فیزیکی \english{LPSs} افزایش می‌دهد. \english{Oxi-d-MXene900} با استفاده از یک فرآیند نفوذ مذاب در یک کاتد گوگرد (\english{S/Oxi-d-MXene900}) گنجانده شد، در حالی که \english{Oxi-d-MXene600} برای ساخت یک جداکننده اصلاح‌شده از طریق فیلتراسیون خلاء بر روی یک غشای الیاف شیشه (\english{GF}) استفاده شد. شکل ۹(ب) پروفایل‌های شارژ/دشارژ را در \english{0.2C} نشان می‌دهد. این نمودار منحنی‌های شارژ/دشارژ گالوانواستاتیک کاتد \english{S/Oxi-d-MXene900} را که با یک جداکننده اصلاح‌شده یا اصلاح‌نشده در \english{0.2C} جفت شده است، مقایسه می‌کند. باتری با جداکننده اصلاح‌شده ظرفیت دشارژ بالاتر و هیسترزیس ولتاژ کمتری را نشان می‌دهد که بیانگر بهبود بهره‌برداری از گوگرد و محبوس‌سازی بهتر پلی‌سولفید است. شکل ۹(ج) عملکرد چرخه‌ای را در \english{0.2C} نشان می‌دهد. طی ۱۰۰ چرخه، جداکننده اصلاح‌شده به طور قابل توجهی عملکرد باتری را با افزایش حفظ ظرفیت و بازده کولمبی (حدود ۹۹٪) بهبود می‌بخشد. جداکننده اصلاح‌نشده به دلیل نفوذ کنترل‌نشده \english{LPS}، افت سریع ظرفیت را نشان می‌دهد که منجر به اتلاف ماده فعال می‌شود. شکل ۹(د) آزمون قابلیت نرخ را نشان می‌دهد. این نمودار عملکرد باتری را در دانسیته‌های جریان فزاینده از \english{0.2C} تا \english{5C} بررسی می‌کند. الکترود \english{S/Oxi-d-MXene900} با جداکننده اصلاح‌شده ظرفیت شارژ/دشارژ پایداری را حفظ می‌کند و قابلیت نرخ عالی را نشان می‌دهد. پس از بازگشت به \english{0.2C}، ظرفیت به طور کامل بازیابی می‌شود که پایداری ساختاری خوب را تأیید می‌کند. شکل ۹(ه) آزمون جذب پلی‌سولفیدهای لیتیوم (\english{LPSs}) را نشان می‌دهد. یک آزمون جذب بصری با استفاده از محلول \english{Li₂S₆} برای بررسی توانایی \english{Oxi-d-MXenes} در سرکوب اثر شاتل انجام شد. پس از ۲۴ ساعت، محلول با \english{Oxi-d-MXene600} و \english{Oxi-d-MXene900} تقریباً بی‌رنگ شد که نشان‌دهنده جذب قوی پلی‌سولفید است (شکل ۹(و)). شکل ۹(ز) چرخه‌زنی بلندمدت را در \english{1C} نشان می‌دهد. در \english{1C} برای ۳۰۰ چرخه، باتری با جداکننده اصلاح‌شده ظرفیت دشارژ بالایی (حدود \english{900 mAh g⁻¹}) و بازده کولمبی حدود ۹۹٪ را حفظ می‌کند، که ثابت می‌کند \english{Oxi-d-MXenes} به طور مؤثری اثر شاتل را کاهش داده و طول عمر باتری را افزایش می‌دهند. در مجموع، شکل ۶ (خطا در متن اصلی، باید شکل ۹ باشد) نشان می‌دهد که چگونه \english{MXene}های اکسید شده با \english{CO₂} عملکرد باتری \english{Li-S} را با پایدارسازی کاتدهای گوگرد و به دام انداختن پلی‌سولفیدها به طور قابل توجهی بهبود می‌بخشند.$^{۱۴۳}$
}

\subsubsection*{\persian{۲.۴.۲. قالب‌گیری کروی \english{MXene}های \english{Ti₃C₂Tₓ} با \english{CoSe₂} برای بهبود عملکرد \english{Li-S}}}
\persian{
لیو و همکاران (۲۰۲۲) یک روش نوآورانه برای بهبود عملکرد باتری \english{Li-S} از طریق قالب‌گیری کروی نانوورقه‌های \english{MXene} \english{Ti₃C₂Tₓ} تزئین‌شده با نانوذرات پلی‌مورفیک \english{CoSe₂} (\english{S@MXene-CoSe₂}) معرفی کردند.$^{۱۳۶}$ با ادغام نانوذرات \english{CoSe₂} در نانوورقه‌های \english{MXene}، محققان به طور مؤثری از روی هم قرار گرفتن مجدد \english{MXene} جلوگیری کردند و در نتیجه انتقال یونی را بهبود بخشیده و الکتروکاتالیز مرحله‌ای را تسهیل کردند. محاسبات \english{DFT} نشان داد که هر دو فاز مکعبی (\english{C-CoSe₂}) و اورتورومبیک (\english{O-CoSe₂}) به طور قابل توجهی پیوند \english{LiPS} و فعالیت کاتالیزوری را افزایش می‌دهند و مسیر کاهش گوگرد را بهینه می‌کنند. کاتدهای \english{S@MXene-CoSe₂} عملکرد الکتروشیمیایی برتری را نشان دادند، از جمله نرخ نفوذ \english{Li⁺} بالاتر، سینتیک واکنش بهبودیافته و قابلیت نرخ افزایش‌یافته، که از کاتدهای متداول \english{S@MXene} بهتر عمل کردند. شکل ۱۰ یک تحلیل جامع از خواص ساختاری، محاسباتی و الکتروشیمیایی \english{S@MXene-CoSe₂} را در مقایسه با \english{S@MXene} ارائه می‌دهد و مزایای ادغام نانوذرات پلی‌مورفیک \english{CoSe₂} را در باتری‌های لیتیوم-گوگرد (\english{Li-S}) نشان می‌دهد.$^{۱۳۶}$ شکل ۱۰(الف) تفاوت‌های مورفولوژیکی بین \english{S@MXene} و \english{S@MXene-CoSe₂} را نشان می‌دهد. ساختار \english{S@MXene-CoSe₂} نانوورقه‌های \english{MXene} را به صورت آزادانه در اطراف کره گوگرد نشان می‌دهد که از تجمع جلوگیری کرده و انتقال یونی را افزایش می‌دهد. گنجاندن نانوذرات \english{CoSe₂} همچنین فعالیت کاتالیزوری اضافی را معرفی می‌کند و آن را از نانوورقه‌های به هم فشرده در \english{S@MXene} متمایز می‌کند. شکل‌های ۱۰(ب) و (ج) محاسبات \english{DFT} را ارائه می‌دهند که تفاوت‌های چگالی بار را برای جذب \english{Li₂S₆} و \english{Li₂S} بر روی سطوح \english{C-CoSe₂}، \english{O-CoSe₂} و \english{Ti₃C₂O₂} بررسی می‌کنند. این شبیه‌سازی‌ها برهمکنش‌های بار قوی بین پلی‌سولفیدهای لیتیوم و \english{CoSe₂} را نشان می‌دهند که منجر به شیمی‌جذب افزایش‌یافته می‌شود که در سرکوب اثر شاتل پلی‌سولفید در باتری‌های \english{Li-S} حیاتی است. شکل ۱۰(د) انرژی‌های پیوندی بین پلی‌سولفیدهای لیتیوم و سطوح مختلف را کمی‌سازی می‌کند و نشان می‌دهد که \english{O-CoSe₂} قوی‌ترین میل پیوندی را دارد که بیانگر قابلیت برتر برای تثبیت پلی‌سولفیدها و بهبود بهره‌برداری از گوگرد است. شکل ۱۰(ه) پروفایل انرژی آزاد گیبس را برای کاهش پلی‌سولفید بر روی این سطوح نشان می‌دهد و نشان می‌دهد که الکتروکاتالیز مرحله‌ای می‌تواند از طریق جذب انتخابی بر روی جایگاه‌های کاتالیزوری مختلف به دست آید. مسیر واکنش بهینه، که در شکل ۱۰(و) نشان داده شده، نشان می‌دهد که...
}

% --- End of content for page 13 ---
% --- SECTION 14: Page 14 of Translation ---
% --- Notes: Completes Figure 10 analysis and introduces Figure 11.
% --------------------------------------------------------------------------

\begin{figure*}[t]
    \centering
    \includegraphics[width=\textwidth]{example-image-b} % Placeholder for Figure 10
    \caption{\persian{
    \farsibold{شکل ۱۰.} (الف) تصویر شماتیک که تفاوت‌ها در روی هم قرار گرفتن نانوورقه‌های \english{MXene} را برای \english{S@MXene} و \english{S@MXene-CoSe₂} نشان می‌دهد. (ب) و (ج) شبیه‌سازی‌های \english{DFT} از نانوورقه‌های \english{MXene-CoSe₂}: (ب) نمودارهای تفاوت چگالی بار برای \english{Li₂S₆} و (ج) \english{Li₂S} جذب‌شده بر روی \english{C-CoSe₂}, \english{O-CoSe₂} و \english{Ti₃C₂O₂}. (د) انرژی‌های پیوندی پلی‌سولفیدهای لیتیوم بر روی \english{O-CoSe₂}, \english{C-CoSe₂} و \english{Ti₃C₂O₂}، که نشان می‌دهد \english{O-CoSe₂} قوی‌ترین جذب را از خود نشان می‌دهد. (ه) پروفایل‌های انرژی آزاد گیبس کاهش پلی‌سولفید لیتیوم بر روی سطوح مختلف. مقادیر با رنگ مشخص شده تغییرات در انرژی آزاد را نشان می‌دهند، در حالی که مقادیر زیرخط‌دار کمترین موانع انرژی را برای هر مرحله کاهش مشخص می‌کنند، که نشان می‌دهد \english{MXene-CoSe₂} الکتروکاتالیز مرحله‌ای را امکان‌پذیر می‌سازد. (و) نمایش شماتیک از مسیر بهینه الکتروکاتالیز مرحله‌ای بر اساس پروفایل انرژی آزاد گیبس. (ز) منحنی‌های ولتامتری چرخه‌ای (\english{CV}) \english{S@MXene-CoSe₂} در نرخ‌های اسکن از \english{0.1} تا \english{0.6 mV s⁻¹}. (ح) تحلیل جریان پیک منحنی‌های \english{CV} برای تعیین ضرایب نفوذ \english{Li⁺} در موقعیت‌های \english{A}, \english{B} و \english{C}. (ط) پروفایل‌های اولیه \english{CV} \english{S@MXene-CoSe₂} در نرخ اسکن \english{0.1 mV s⁻¹}، که برگشت‌پذیری الکتروشیمیایی را نشان می‌دهد. (ی) نمودارهای تافل که سینتیک واکنش \english{S@MXene} و \english{S@MXene-CoSe₂} را در پیک‌های مختلف مقایسه می‌کنند. (ک) مقایسه عملکرد نرخ کاتدهای \english{S@MXene-CoSe₂} و \english{S@MXene} در نرخ‌های مختلف شارژ/دشارژ. (ل) تحلیل نسبت \english{Q₂/Q₁}، جایی که نسبت بالاتر نشان‌دهنده تبدیل کارآمدتر گوگرد و سینتیک واکنش بهبودیافته است. (م) پایداری چرخه‌ای بلندمدت \english{S@MXene-CoSe₂} و \english{S@MXene} در \english{0.5C}، که پایداری افزایش‌یافته و زوال ظرفیت کمتر را در کاتد \english{S@MXene-CoSe₂} نشان می‌دهد. بازتولید شده از مرجع ۱۳۶ با مجوز از انجمن شیمی آمریکا، کپی‌رایت ۲۰۲۲.
    }}
    \label{fig:10}
\end{figure*}

\persian{
\english{C-CoSe₂}, \english{O-CoSe₂} و \english{Ti₃C₂O₂} در مجموع واکنش‌های ردوکس گوگرد کارآمدتری را امکان‌پذیر می‌سازند، موانع انرژی را کاهش داده و سینتیک باتری را بهبود می‌بخشند. شکل ۱۰(ز) منحنی‌های ولتامتری چرخه‌ای (\english{CV}) \english{S@MXene-CoSe₂} را در نرخ‌های اسکن مختلف نشان می‌دهد که پیک‌های ردوکس به خوبی تعریف‌شده‌ای را که نشان‌دهنده رفتار الکتروشیمیایی پایدار است، نمایش می‌دهد. رابطه خطی بین جریان پیک و نرخ اسکن در شکل ۱۰(ح) امکان محاسبه ضرایب نفوذ \english{Li⁺} را فراهم می‌کند که انتقال سریع‌تر یون در \english{S@MXene-CoSe₂} را تأیید می‌کند. نمودارهای تافل در شکل ۱۰(ی) مقادیر شیب کمتری را برای \english{S@MXene-CoSe₂} نشان می‌دهند که بیانگر سینتیک الکتروکاتالیزوری بهبودیافته است. شکل‌های ۱۰(ک) و (ل) عملکرد نرخ و نسبت‌های \english{Q₂/Q₁} را مقایسه می‌کنند، جایی که \english{S@MXene-CoSe₂} به طور مداوم ظرفیت‌های ویژه بالاتر و کارایی کاتالیزوری بهتری را در نرخ‌های مختلف شارژ/دشارژ ارائه می‌دهد. در نهایت، شکل ۱۰(م) پایداری چرخه‌ای بلندمدت را نشان می‌دهد، به طوری که \english{S@MXene-CoSe₂} بیش از \english{300 mAh g⁻¹} را پس از ۱۰۰۰ چرخه حفظ می‌کند، در حالی که \english{S@MXene} به سرعت تخریب می‌شود. این نتایج تأیید می‌کنند که تزئین با نانوذرات \english{CoSe₂} سینتیک واکنش، نگهداری پلی‌سولفید و عملکرد کلی باتری را افزایش می‌دهد و \english{S@MXene-CoSe₂} را به یک ماده امیدوارکننده برای باتری‌های \english{Li-S} با کارایی بالا تبدیل می‌کند.$^{۱۳۶}$
}

\subsubsection*{\persian{۲.۴.۳. \english{MXene}های عامل‌دار شده به عنوان مواد لنگراندازی پلی‌سولفید}}
\persian{
\english{MXene}های عامل‌دار شده قابلیت‌های لنگراندازی برتری برای \english{LiPSs} نشان داده‌اند و به طور مؤثری اثر شاتل را کاهش می‌دهند در حالی که رسانایی الکتریکی بالا و فعالیت کاتالیزوری را حفظ می‌کنند. ژو و همکاران (۲۰۲۳) از محاسبات \english{DFT} برای ارزیابی قابلیت‌های لنگراندازی \english{MXene}های عامل‌دار شده با \english{O/F} (\english{M₂CT₂}، که در آن \english{M = Mo, V, Ti, Zr, Nb}؛ \english{T = O} یا \english{F}) برای جذب \english{LiPS} استفاده کردند.$^{۱۲۸}$ شکل ۱۱ یک تحلیل جامع از رفتار جذب، برهمکنش‌های واندروالس و تغییرات انرژی آزاد مرتبط با استفاده از \english{MXene}های عامل‌دار شده (\english{M₂CT₂}) به عنوان مواد میزبان برای \english{LiPSs} در باتری‌های لیتیوم-گوگرد (\english{Li-S}) ارائه می‌دهد.$^{۱۲۸}$ شکل‌های ۱۱(الف-و) پیکربندی‌های جذب گوگرد (\english{S₈}) و گونه‌های مختلف \english{LiPS} (\english{Li₂S₈, Li₂S₆, Li₂S₄, Li₂S₂, Li₂S}) را بر روی سطوح \english{M₂CT₂} نشان می‌دهند. ساختارهای اتمی نشان می‌دهند که چگونه اتم‌های لیتیوم، گوگرد، اکسیژن و فلزات واسطه (\english{TM}) در مراحل مختلف لیتیاسیون برهمکنش دارند. این پیکربندی‌ها نشان می‌دهند که چگونه \english{MXene}های عامل‌دار شده جایگاه‌های لنگراندازی برای \english{LiPSs} فراهم می‌کنند و به طور بالقوه اثر شاتلی که باتری‌های \english{Li-S} را مختل می‌کند، کاهش می‌دهند. شکل‌های ۱۱(ز) و (ح) انرژی جذب \english{LiPS} را برای گونه‌های مختلف \english{M₂CO₂} و \english{M₂CF₂} به ترتیب ارائه می‌دهند. انرژی جذب یک پارامتر کلیدی است که قدرت برهمکنش بین \english{LiPSs} و ماده میزبان را تعیین می‌کند. انرژی جذب متوسط (حدود \english{2.5 eV}) مطلوب است، زیرا پیوند بیش از حد قوی می‌تواند منجر به تجزیه \english{LiPSs} شود، در حالی که پیوند ضعیف می‌تواند منجر به انحلال پلی‌سولفید و اثرات شاتل شود. نتایج نشان می‌دهد که \english{Mo₂CF₂} و \english{V₂CF₂} مطلوب‌ترین ویژگی‌های جذب را دارند. شکل‌های ۱۱(ط) و (ی) سهم برهمکنش‌های واندروالس (\english{vdW}) را در مراحل مختلف لیتیاسیون برای مواد \english{M₂CO₂} و \english{M₂CF₂} تحلیل می‌کنند. نمودارهای میله‌ای نشان می‌دهند که چگونه نسبت برهمکنش‌های \english{vdW} با تبدیل \english{LiPSs} از \english{S₈} به \english{Li₂S} تغییر می‌کند. نسبت بالای \english{vdW} در مراحل اولیه (\english{Li₂S₈} و \english{Li₂S₆}) و نسبت پایین‌تر در مراحل بعدی (\english{Li₂S₂} و \english{Li₂S}) نشان می‌دهد که با کاهش کامل گوگرد، برهمکنش‌های شیمیایی غالب می‌شوند. شکل‌های ۱۱(ک) و (ل) نمودارهای انرژی آزاد را برای \english{SRR} بر روی مواد \english{M₂CO₂} و \english{M₂CF₂} نمایش می‌دهند. این نمودارها ترمودینامیک...
}

\begin{figure*}[t]
    \centering
    \includegraphics[width=\textwidth]{example-image-c} % Placeholder for Figure 11
    \caption{\persian{
    \farsibold{شکل ۱۱.} (الف)-(و) پیکربندی‌های جذب \english{S₈} و \english{Li₂Sₓ} بر روی سطوح \english{M₂CT₂}. کره‌های خاکستری، زرد، بنفش، قرمز و سبز به ترتیب نمایانگر اتم‌های کربن، گوگرد، لیتیوم، اکسیژن و فلز واسطه (\english{TM}) هستند. (ز) انرژی‌های جذب \english{LiPS} برای گونه‌های مختلف \english{M₂CO₂}. (ح) انرژی‌های جذب \english{LiPS} برای گونه‌های مختلف \english{M₂CF₂}. (ط) نسبت برهمکنش واندروالس محاسبه‌شده بر روی \english{M₂CO₂} در پنج مرحله مختلف لیتیاسیون. (ی) نسبت برهمکنش واندروالس محاسبه‌شده بر روی \english{M₂CF₂} در پنج مرحله مختلف لیتیاسیون. (ک) نمودارهای انرژی آزاد کاهش گوگرد بر روی \english{M₂CO₂}. (ل) نمودارهای انرژی آزاد کاهش گوگرد بر روی \english{M₂CF₂}. بازتولید شده از مرجع ۱۲۸ با مجوز از انجمن شیمی آمریکا، کپی‌رایت ۲۰۲۳.
    }}
    \label{fig:11}
\end{figure*}

% --- End of content for page 14 ---
% --- SECTION 15: Page 15 of Translation ---
% --- Notes: Concludes MXenes, provides comparative evaluation, and introduces heterostructures.
% --------------------------------------------------------------------------

\persian{
تبدیل پلی‌سولفید را نشان می‌دهند و موانع انرژی در هر مرحله را برجسته می‌کنند. موانع انرژی پایین‌تر، سینتیک واکنش سریع‌تر را تسهیل کرده و عملکرد باتری را بهبود می‌بخشند. نتایج تأیید می‌کنند که \english{Mo₂CF₂} و \english{V₂CF₂} کمترین پتانسیل محدودکننده (\english{U_L > -0.6 V}) را از خود نشان می‌دهند و آنها را برای کاتالیز کردن کاهش گوگرد در باتری‌های \english{Li-S} بسیار مؤثر می‌سازند. در مجموع، شکل ۱۱ بینش‌های حیاتی در مورد نقش \english{MXene}های عامل‌دار شده در پایدارسازی \english{LiPSs}، ترویج نفوذ سریع لیتیوم و افزایش سینتیک کاهش گوگرد ارائه می‌دهد و در نهایت طراحی باتری‌های \english{Li-S} با کارایی بالا را پیش می‌برد.$^{۱۲۸}$
}
\subsubsection*{\persian{۲.۴.۴. کاتالیزورهای تک‌اتمی و \english{MXene}های تزئین‌شده با الکتروکاتالیست}}
\persian{
پیشرفت‌های اخیر در کاتالیزورهای تک‌اتمی (\english{SACs}) بهبودهای امیدوارکننده‌ای در عملکرد باتری \english{Li-S} نشان داده‌اند. یک استراتژی جدید اچینگ و خود-کاهش بدون فلوراید در دمای اتاق برای سنتز \english{FeN₄-O-NC-VN}، یک الکتروکاتالیست مشتق از \english{MXene} نیترید وانادیوم با ساختار فازی بازآرایی‌شده با آهن اتمی هماهنگ با نیتروژن، اکسیژن درون صفحه‌ای و یک پل کربنی دوپ‌شده با نیتروژن محوری به کار گرفته شد.$^{۱۲۷}$ این پیکربندی، شکافتگی اوربیتال \english{d} آهن را مدوله کرد، سطح انرژی \english{d_{z²}} را کاهش داد و در نتیجه جذب پلی‌سولفید را افزایش داد، انرژی فعال‌سازی واکنش را کاهش داد و اثر شاتل را کم کرد. اعتبارسنجی تجربی از طریق طیف‌سنجی جذب اشعه ایکس (\english{XANES})، رزونانس پارامغناطیس الکترون (\english{EPR})، طیف‌سنجی موسباوئر و شبیه‌سازی‌های \english{DFT} تأیید کرد که \english{FeN₄-O-NC-VN} سینتیک ردوکس برتر، لیتوفیلیستی افزایش‌یافته و قابلیت لنگراندازی قوی پلی‌سولفید را از خود نشان می‌دهد. آزمون‌های الکتروشیمیایی بهبودهای قابل توجهی در بهره‌برداری از کاتد گوگرد، سرکوب رشد دندریت لیتیوم و افزایش نفوذ یون لیتیوم را نشان دادند که منجر به ظرفیت ویژه بالا، پایداری چرخه‌ای عالی و سینتیک انتقال بار سریع شد. جداکننده اصلاح‌شده به طور مؤثری از عبور پلی‌سولفید جلوگیری کرد و عملکرد پایدار را طی ۵۰۰ چرخه شارژ/دشارژ تضمین نمود. علاوه بر این، آند لیتیوم اصلاح‌شده با \english{FeN₄-O-NC-VN} اورپتانسیل هسته‌زایی پایینی را نشان داد و رسوب یکنواخت لیتیوم و عمر چرخه‌ای طولانی را تضمین کرد. این نتایج نشان می‌دهند که \english{MXene}های اصلاح‌شده با \english{SAC} یک رویکرد مستحکم برای بهینه‌سازی عملکرد باتری \english{Li-S} و افزایش عمر چرخه‌ای ارائه می‌دهند و استراتژی‌های طراحی جدیدی برای سیستم‌های ذخیره‌سازی انرژی مبتنی بر گوگرد نسل بعد عرضه می‌کنند.$^{۱۲۷}$
}
\persian{
به طور خلاصه، \english{MXene}ها پتانسیل قابل توجهی در پیشبرد فناوری باتری \english{Li-S} با پرداختن به محدودیت‌های کلیدی، از جمله اثر شاتل پلی‌سولفید، بهره‌برداری ضعیف از گوگرد و سینتیک واکنش کند، نشان داده‌اند. استراتژی‌های مختلفی مانند عامل‌دار کردن سطح، تزئین با الکتروکاتالیست و طراحی هتروساختار برای بهینه‌سازی مواد مبتنی بر \english{MXene} برای بهبود عملکرد باتری \english{Li-S} مورد بررسی قرار گرفته‌اند. تحقیقات آینده باید بر تکنیک‌های سنتز مقیاس‌پذیر، پیاده‌سازی عملی در سلول‌های تجاری \english{Li-S} و ادغام بیشتر با سیستم‌های الکترولیت نوین برای باز کردن پتانسیل کامل \english{MXene}ها متمرکز شود. با پیشرفت‌های مداوم، مواد مبتنی بر \english{MXene} آماده ایفای نقش حیاتی در تجاری‌سازی باتری‌های \english{Li-S} با چگالی انرژی بالا نسل بعد هستند و راه را برای پذیرش گسترده آنها در وسایل نقلیه الکتریکی و کاربردهای ذخیره‌سازی انرژی در مقیاس شبکه هموار می‌کنند.
}
\subsubsection*{\persian{۲.۴.۵. ارزیابی مقایسه‌ای \english{MOFs}, \english{COFs} و \english{MXene}ها برای کاربردهای باتری \english{Li-S}}}
\persian{
برای ارائه یک مقایسه مختصر و واضح از استراتژی‌های متنوع مواد که در حال حاضر برای باتری‌های \english{Li-S} بررسی می‌شوند، ما معیارهای کلیدی عملکرد سیستم‌های مبتنی بر \english{MOF}, \english{COF} و \english{MXene} را در جدول ۳ ادغام کرده‌ایم. این بخش یک خلاصه ارزیابی ارائه می‌دهد که شامل پارامترهای حیاتی مانند ظرفیت دشارژ، بارگذاری گوگرد، قابلیت نرخ، پایداری چرخه‌ای و نسبت الکترولیت به گوگرد (\english{E/S}) است. با محک زدن این شاخص‌ها، جدول یک نمای کلی شهودی از مزایا و محدودیت‌های مرتبط با هر نوع ماده ارائه می‌دهد. چنین ارزیابی مقایسه‌ای نه تنها پیشرفت‌های حاصل در طراحی مواد برای بهبود بهره‌برداری از گوگرد و افزایش عملکرد باتری را برجسته می‌کند، بلکه چالش‌هایی را که باید برای کاربردهای عملی در مقیاس بزرگ برطرف شوند، شناسایی می‌کند. در بحث زیر، ما این معیارها را تحلیل کردیم تا بر روندها در عملکرد و مسیرهای بالقوه برای تحقیقات بیشتر در توسعه باتری‌های \english{Li-S} با چگالی انرژی بالا و عمر طولانی تأکید کنیم.
}

\subsection*{\persian{۲.۵. مواد هتروساختار به عنوان کاتالیزورهای کارآمد برای باتری‌های \english{Li-S}}}
\persian{
مواد هتروساختار به عنوان کاتالیزورهای بسیار مؤثری برای باتری‌های \english{Li-S} ظهور کرده‌اند و به چالش‌های کلیدی مانند اثر شاتل و سینتیک واکنش کند رسیدگی می‌کنند. این مواد که از دو یا چند جزء با خواص الکترونیکی و ساختاری متمایز تشکیل شده‌اند، اثرات هم‌افزایی از خود نشان می‌دهند که فعالیت کاتالیزوری و پایداری را افزایش می‌دهد. در باتری‌های \english{Li-S}، انحلال پلی‌سولفیدهای میانی (\english{Li₂S₆} و \english{Li₂S₄}) در الکترولیت منجر به افت ظرفیت و عملکرد چرخه‌ای ضعیف می‌شود. کاتالیزورهای هتروساختار—شامل اکسیدهای فلزی، سولفیدها و فسفیدها ترکیب‌شده با کربن‌های رسانا یا فلزات واسطه—به طور مؤثری پلی‌سولفیدها را جذب و تبدیل آنها را کاتالیز می‌کنند. فصل مشترک‌های هترو آنها انتقال بار را تسهیل کرده و در نتیجه هم واکنش‌های ردوکس گوگرد و هم تبدیل \english{Li₂S} نامحلول را تسریع می‌کنند. نمونه‌های قابل توجه شامل \english{MoS₂/Ti₃C₂Tₓ MXenes}, \english{Fe₃O₄/graphene} و \english{Co₃S₄/CoP} هستند که رسانایی الکتریکی بالا و پیوند قوی پلی‌سولفید را از خود نشان می‌دهند. این ساختارها نفوذ یون لیتیوم را بهینه کرده و واکنش‌های جانبی را سرکوب می‌کنند که منجر به بهبود حفظ ظرفیت و عملکرد نرخ می‌شود. با بهره‌گیری از کاتالیزورهای هتروساختار، محققان قصد دارند باتری‌های \english{Li-S} را با طول عمر طولانی‌تر و چگالی انرژی بالاتر توسعه دهند و قابلیت آنها را برای کاربردهای ذخیره‌سازی انرژی نسل بعد افزایش دهند. پیشرفت‌های مداوم در مهندسی مواد و تکنیک‌های سنتز انتظار می‌رود عملکرد باتری و امکان‌سنجی تجاری را بیشتر بهبود بخشد.
}
\persian{
یک مطالعه اخیر سنتز هتروساختارهای \english{CoS-FeS} پشتیبانی‌شده توسط نانو قفس‌های کربنی دوپ‌شده با نیتروژن (\english{CoS-FeS@NC}) را به عنوان یک استراتژی برای مقابله با سینتیک ردوکس کند و اثر شاتل پلی‌سولفید در باتری‌های \english{Li-S} گزارش کرد.$^{۱۴۸}$ کامپوزیت \english{CoS-FeS@NC} دارای یک ساختار توخالی است که بهره‌برداری از گوگرد و پایداری چرخه‌ای را افزایش می‌دهد. دوپینگ نیتروژن رسانایی نانو قفس‌های کربنی را افزایش می‌دهد، در حالی که هتروپیوست \english{CoS-FeS} جایگاه‌های فعال فراوانی را فراهم کرده و تبدیل پلی‌سولفید لیتیوم را تسریع می‌کند. این کامپوزیت...
}

% --- End of content for page 15 ---
% --- SECTION 16: Page 16 of Translation ---
% --- Notes: Includes the critical Table 3 and continues with heterostructure examples.
% --------------------------------------------------------------------------

\persian{
ظرفیت برگشت‌پذیر \english{638 mAh g⁻¹} را در \english{0.2C} پس از ۲۰۰ چرخه به دست آورد و ظرفیت \english{542 mAh g⁻¹} را در \english{1C} پس از ۱۰۰۰ چرخه حفظ کرد، با نرخ تخریب ظرفیت حداقل \english{0.05\%} در هر چرخه.$^{۱۴۸}$
}
\persian{
تنگ و همکاران هتروپیوست‌های \english{MnFe₂O₄/MnO} ترکیب‌شده با شبکه‌های نانولوله کربنی (\english{CNT}) را برای افزایش عملکرد باتری‌های \english{Li-S} معرفی کردند.$^{۱۴۹}$ این هتروپیوست‌ها از یک مکانیزم کاتالیزوری دوگانه استفاده می‌کنند که در آن \english{MnO} تبدیل گوگرد (\english{S₈}) به پلی‌سولفیدهای لیتیوم (\english{Li₂S₆}) را کاتالیز می‌کند، در حالی که \english{MnFe₂O₄} تبدیل بیشتر به \english{Li₂S₄} و \english{Li₂S} را تسهیل می‌کند. یک میدان الکتریکی داخلی (\english{BEF}) در فصل مشترک هتروپیوست، مهاجرت جهت‌دار و تبدیل افزایش‌یافته پلی‌سولفید لیتیوم را امکان‌پذیر می‌سازد. شبکه‌های \english{CNT} رسانایی الکترونیکی و پایداری ساختاری را بهبود می‌بخشند. این کامپوزیت به ظرفیت ویژه بالای \english{1358 mAh g⁻¹} در \english{0.1C} دست یافت و ظرفیت \english{452 mAh g⁻¹} را طی ۲۰۰۰ چرخه در \english{2C} حفظ کرد، که پایداری چرخه‌ای بلندمدت عالی را نشان می‌دهد.$^{۱۴۹}$
}
\persian{
مطالعه دیگری ورقه‌های فوق‌نازک دی‌سولفید مولیبدن دوپ‌شده با سلنیوم (\english{Se-MoS₂}) پشتیبانی‌شده بر روی اکسید گرافن کاهش‌یافته (\english{rGO}) را برای بهبود عملکرد باتری \english{Li-S} توسعه داد.$^{۱۵۰}$ دوپینگ سلنیوم، جاهای خالی آنیونی و نقص‌هایی را ایجاد می‌کند و به طور قابل توجهی رسانایی و فعالیت کاتالیزوری را افزایش می‌دهد. اعوجاج‌های شبکه، جایگاه‌های فعال برای جذب و تبدیل پلی‌سولفید لیتیوم را افزایش می‌دهند که منجر به ظرفیت اولیه \english{1084.9 mAh g⁻¹} در \english{1C} و حفظ \english{620.7 mAh g⁻¹} پس از ۵۰۰ چرخه، با زوال ظرفیت تنها \english{0.077\%} در هر چرخه می‌شود.$^{۱۵۰}$
}
\persian{
یک کامپوزیت \english{MoO₂@rGO} که از طریق یک فرآیند هیدروترمال سنتز شده بود، جذب برتر پلی‌سولفیدهای لیتیوم و سینتیک ردوکس بهبودیافته را نشان داد.$^{۱۵۱}$ کامپوزیت \english{MoO₂@rGO} به ظرفیت ویژه اولیه \english{1136 mAh g⁻¹} در \english{0.3C} دست یافت و \english{75.4\%} از ظرفیت خود را پس از ۱۰۰ چرخه حفظ کرد. سلول‌های کیسه‌ای مونتاژ شده با این ماده، بازده کولمبی پایداری (حدود \english{99.2\%}) و حفظ ظرفیت \english{65.4\%} را پس از ۱۱۰ چرخه در \english{0.2C} نشان دادند.$^{۱۵۱}$
}
\persian{
کاتالیزورهای تک‌اتمی خاکی کمیاب (\english{SACs}) نیز برای باتری‌های \english{Li-S} مورد بررسی قرار گرفته‌اند. یک مطالعه، اتم‌های لوتتیوم (\english{Lu}) تعبیه‌شده در کربن متخلخل دوپ‌شده با نیتروژن (\english{Lu SACs/NC}) را به عنوان یک الکتروکاتالیست کاتد معرفی کرد که جذب قوی پلی‌سولفیدها و سینتیک ردوکس گوگرد بهبودیافته را از طریق هیبریداسیون اوربیتال \english{f-d-p} نشان می‌دهد.$^{۱۵۲}$ کاتد \english{Lu SAs/NC} ظرفیت ویژه \english{1391.8 mAh g⁻¹} را در \english{0.1C} با نرخ زوال ظرفیت پایین \english{0.049\%} در هر چرخه طی ۱۰۰۰ چرخه به نمایش گذاشت. طیف‌سنجی رامان درجا و محاسبات \english{DFT} توانایی آن را در سرکوب اثر شاتل و تسریع هسته‌زایی و انحلال \english{Li₂S} تأیید کرد.$^{۱۵۲}$
}
\persian{
یانگ و همکاران نانوکره‌های توخالی \english{FeS₂@C} تزئین‌شده با پلی(۳,۴-اتیلن‌دی‌اکسی‌تیوفن) (\english{PEDOT}) را توسعه دادند و از خواص کاتالیزوری \english{FeS₂} و رسانایی \english{PEDOT} برای افزایش انتقال بار و پایداری چرخه‌ای بهره بردند.$^{۱۵۳}$ کامپوزیت بهینه‌شده (\english{4 wt\% PEDOT}) ظرفیت اولیه \english{1435 mAh g⁻¹} را در \english{0.1C} نشان داد و ظرفیت \english{595 mAh g⁻¹} را پس از ۵۰۰ چرخه در \english{1C} با زوال \english{0.07\%} در هر چرخه حفظ کرد.$^{۱۵۳}$ یک کاتالیزور نوین \english{Cu-CeO₂-x@N/C} که شامل نانوذرات مس و اکسید سریم غنی از جای خالی اکسیژن است نیز گزارش شد.$^{۱۵۴}$ این کاتالیزور ظرفیت دشارژ \english{793.2 mAh g⁻¹} را در \english{5C} طی ۵۰۰ چرخه نشان داد و ظرفیت \english{719.9 mAh g⁻¹} را در \english{0.3C} با بارگذاری بالای گوگرد \english{5.4 mg cm⁻²} حفظ کرد.$^{۱۵۴}$ یک کاتالیزور دو فلزی \english{CuCo-NC}...
}

% Using `sidewaystable` for better layout of the wide table. Requires `rotating` package.
\usepackage{rotating}
\begin{sidewaystable*}
    \centering
    \caption{\persian{\farsibold{جدول ۳.} عملکرد الکتروشیمیایی مقایسه‌ای مواد اخیر مبتنی بر \english{MOF}, \english{COF} و \english{MXene} در باتری‌های \english{Li-S}}}
    \label{tab:comparison_materials}
    \tiny % Smaller font size for the table
    \begin{tabular}{rllllllll}
        \toprule
        \farsibold{مرجع} & \farsibold{نسبت E/S} & \farsibold{پایداری چرخه‌ای} & \farsibold{قابلیت نرخ} & \farsibold{بارگذاری S (\english{mg cm⁻²})} & \farsibold{ظرفیت (\english{mAh g⁻¹})} & \farsibold{نقش عملکردی کلیدی} & \farsibold{نوع ماده} \\
        \midrule
        \persian{چیوچان و همکاران$^{147}$} & \persian{نامشخص} & \persian{۸۴٪ پس از ۲۵۰ چرخه} & \persian{خوب، در دمای اتاق} & \persian{حدود ۴.۵} & \persian{حدود ۱۲۰۰} & \persian{الکترولیت حالت جامد با گروه‌های \english{-SO₃Li}} & \english{MOF (UIOSLI)} \\
        \persian{چیوچان و همکاران$^{147}$} & \persian{۳.۵} & \persian{۰.۰۶٪ افت در هر چرخه} & \persian{بالا، در \english{3C} حفظ شد} & \persian{حدود ۵.۰} & \persian{پایدار در حدود ۱۱۵۰} & \persian{الکترولیت هیبریدی جامد با مایع یونی} & \english{MOF (Li-IL/UIOSLI)} \\
        \persian{مطالعه \english{EDOT-Por-COF}$^{123}$} & \persian{۳} & \persian{۸۰٪ پس از ۵۰۰ چرخه} & \persian{\english{720} در \english{3C}} & \persian{۴.۵} & \persian{۱۲۷۸} & \persian{میزبان کاتد الکتروکاتالیزوری} & \english{COF (EDOT-Por-COF)} \\
        \persian{مطالعه \english{PI-CONS}$^{113}$} & \persian{۳.۲} & \persian{۰.۰۱۴٪ افت در هر چرخه} & \persian{\english{650} در \english{2C}} & \persian{۴.۸} & \persian{۱۱۷۰} & \persian{میزبان گوگرد با \english{COF}های مبتنی بر پلی‌ایمید} & \english{COF (PI-CONS)} \\
        \persian{مطالعه \english{3D-scu-COF}$^{126}$} & \persian{۲.۹} & \persian{۹۲٪ پس از ۴۰۰ چرخه} & \persian{\english{800} در \english{2C}} & \persian{۵.۲} & \persian{۱۲۳۰} & \persian{میزبان گوگرد سه‌بعدی با فعالیت ردوکس بالا} & \english{COF (3D-scu-COF)} \\
        \persian{مطالعه \english{Oxi-d-MXene}$^{143}$} & \persian{۲.۸} & \persian{۳۰۰ چرخه در \english{1C} با \english{CE} ۹۹٪} & \persian{پایدار از \english{0.2C} تا \english{5C}} & \persian{۵} & \persian{۹۰۰} & \persian{کاتد + جداکننده اصلاح‌شده برای سرکوب شاتل} & \english{MXene (Oxi-d-MXene900)} \\
        \persian{لیو و همکاران (۲۰۲۲)$^{136}$} & \persian{نامشخص} & \persian{۱۰۰۰ چرخه در \english{0.5C}} & \persian{حفظ ظرفیت بالا در نرخ‌ها} & \persian{۶} & \persian{>۳۰۰ پس از ۱۰۰۰ چرخه} & \persian{میزبان گوگرد کاتالیزوری با \english{CoSe₂} برای سینتیک بهبودیافته} & \english{MXene S@MXene-CoSe₂} \\
        \persian{ژو و همکاران (۲۰۲۳)$^{128}$} & \persian{۲.۷} & \persian{۳۰۰ چرخه در \english{1C} با حفظ >۹۰٪} & \persian{\english{950} در \english{2C}} & \persian{نامشخص} & \persian{حدود ۱۲۵۰} & \persian{سطح عامل‌دار شده برای جذب و کاتالیز پلی‌سولفید} & \english{MXene (O/F-functionalized...)} \\
        \persian{مطالعه \english{FeN₄-O-NC-VN}$^{127}$} & \persian{نامشخص} & \persian{۵۰۰ چرخه، پایداری افزایش‌یافته} & \persian{انتقال بار سریع} & \persian{نامشخص} & \persian{نامشخص} & \persian{کاتالیزور تک‌اتمی روی \english{MXene} برای کاهش شاتل} & \english{MXene (FeN₄-O-NC-VN)} \\
        \bottomrule
    \end{tabular}
\end{sidewaystable*}


% --- End of content for page 16 ---
% --- SECTION 17: Page 17 of Translation ---
% --- Notes: Concludes heterostructures, introduces solid electrolytes, and includes Figure 12.
% --------------------------------------------------------------------------

\persian{
که از طریق پیرولیز یک چارچوب ایمیدازولی زئولیتی (\english{ZIF}) بر روی نانوشیت‌های \english{g-C₃N₄} سنتز شده بود، جذب قوی \english{LiPS} و تبدیل کاتالیزوری را با نرخ حفظ \english{0.03\%} در هر چرخه در دمای ۰ درجه سانتی‌گراد و \english{0.07\%} در هر چرخه در دمای ۶۰ درجه سانتی‌گراد نشان داد.$^{۱۵۵}$
}
\persian{
سایر پیشرفت‌های اخیر شامل نانوسیم‌های فرومغناطیسی \english{Fe₃O₄} بر روی پارچه کربنی، الکتروکاتالیست‌های \english{CoS₂/CoS₁.₀₉₇@NC/CNT}، هتروساختارهای بلوری/آمورف \english{Ni₂P/CeOₓ}، تلوریم دوپ‌شده \english{Bi₂Se₃}، کاتالیزورهای تاندم \english{Co/Zn} و کاتالیزورهای دو اتمی \english{Fe-Co} می‌باشند.$^{۱۵۶-۱۶۱}$ این مواد فعالیت کاتالیزوری بالا، جذب افزایش‌یافته پلی‌سولفید و پایداری بلندمدت برتری را از خود نشان می‌دهند.
}
\persian{
به طور خلاصه، مواد هتروساختار به طور قابل توجهی عملکرد الکتروشیمیایی باتری‌های \english{Li-S} را با افزایش سینتیک ردوکس گوگرد، سرکوب اثر شاتل پلی‌سولفید و بهبود پایداری چرخه‌ای بهبود بخشیده‌اند. ادغام پایه‌های رسانا، مکانیزم‌های کاتالیزوری دوگانه و مهندسی نقص، پیشرفت‌های قابل توجهی در عملکرد باتری \english{Li-S} را ممکن ساخته است. تحقیقات آینده باید بر بهینه‌سازی فصل مشترک‌های هتروساختار، کاوش استراتژی‌های دوپینگ نوین و مقیاس‌بندی تکنیک‌های سنتز برای تسهیل تجاری‌سازی متمرکز شود. توسعه مداوم کاتالیزورهای هتروساختار در تحقق باتری‌های \english{Li-S} با کارایی بالا، طولانی‌مدت و تجاری‌شدنی حیاتی خواهد بود.
}

\section*{\persian{۳. الکترولیت‌های جامد پیشرفته برای باتری‌های \english{Li-S}}}
\persian{
چیوچان و همکاران یک راه‌حل نوین ارائه دادند: یک الکترولیت حالت جامد مشتق از \english{MOF}ها که به طور خاص برای پرداختن به این مسائل طراحی شده است.$^{۱۴۷}$ محققان یک \english{MOF} مبتنی بر \english{UIO} را که با گروه‌های سولفونات لیتیوم (\english{-SO₃Li}) عامل‌دار شده بود و \english{UIOSLi} نامیده می‌شد، توسعه دادند. این عامل‌دار کردن نه تنها رسانایی یونی را بهبود بخشید، بلکه سرکوب مهاجرت پلی‌سولفید را نیز ممکن ساخت. برای افزایش بیشتر عملکرد، یک مایع یونی لیتیوم (\english{Li-IL}) در ماتریس \english{MOF} گنجانده شد و الکترولیت جامد هیبریدی \english{Li-IL/UIOSLi} را ایجاد کرد. ماده حاصل به رسانایی یونی چشمگیر \english{3.3 × 10⁻⁴ S cm⁻¹} در دمای اتاق دست یافت که یک بهبود قابل توجه برای الکترولیت‌های مبتنی بر \english{MOF} است. آزمون‌های الکتروشیمیایی نشان داد که سلول‌های \english{Li-S} با استفاده از \english{Li-IL/UIOSLi} عملکرد چرخه‌ای پایداری را نشان می‌دهند و ۸۴٪ از ظرفیت اولیه خود را پس از ۲۵۰ چرخه با افت ظرفیت حداقل \english{0.06\%} در هر چرخه حفظ می‌کنند. این عملکرد افزایش‌یافته به دوگانگی عملکردی منحصربه‌فرد \english{MOF} نسبت داده می‌شود: ساختار متخلخل آن از نفوذ پلی‌سولفید جلوگیری می‌کند، در حالی که گروه‌های پیوندی \english{-SO₃Li} انتقال کارآمد یون لیتیوم را تسهیل می‌کنند. این مطالعه همچنین توانایی الکترولیت در کاهش رشد دندریت لیتیوم را برجسته می‌کند و ایمنی را بیشتر بهبود می‌بخشد. برخلاف الکترولیت‌های مایع متداول، طراحی حالت جامد پایداری بلندمدت را تضمین کرده و واکنش‌های پارازیتی با فلز لیتیوم را سرکوب می‌کند.
}
\persian{
شکل ۱۲ طراحی ساختاری، عملکرد الکتروشیمیایی و قابلیت مسدودکنندگی پلی‌سولفید الکترولیت جامد مبتنی بر \english{MOF} را که برای باتری‌های لیتیوم-گوگرد توسعه یافته است، نشان می‌دهد. شکل‌های ۱۳(الف-ج) ساختارهای مولکولی شماتیک مواد \english{MOF} را به تصویر می‌کشند. چارچوب پایه در شکل ۱۲(الف) \english{UIO-66} است که به خاطر خوشه‌های زیرکونیوم مستحکم و تخلخلش شناخته شده است. در شکل ۱۲(ب)، گروه‌های اسید سولفونیک (\english{-SO₃H}) به لیگاند پیوند داده شده و \english{UIO-SO₃H} (\english{UIOS}) را تشکیل می‌دهند، و در شکل ۱۲(ج)، این گروه‌ها برای تولید \english{UIOSLi} بیشتر لیتیومی شده و انتقال یون لیتیوم و سازگاری الکتروشیمیایی را افزایش می‌دهند. شکل‌های ۱۳(د) و (ه) رسانایی یونی کامپوزیت‌های \english{Li-IL/UIOSLi} را با نسبت‌های وزنی مختلف مایع یونی لیتیوم (\english{Li-IL}) به \english{MOF} نشان می‌دهند. نمودارهای نایکوئیست \english{EIS} در شکل ۱۲(د) مقاومت کاهش‌یافته را با افزایش محتوای \english{Li-IL} نشان می‌دهند و شکل ۱۲(ه) رسانایی را کمی‌سازی می‌کند که در نسبت ۱:۱ به اوج \english{3.3 × 10⁻⁴ S cm⁻¹} می‌رسد. فراتر رفتن از این نسبت منجر به رفتار ژل-مانند شده و پایداری مکانیکی را به خطر می‌اندازد. شکل‌های ۱۳(و) و (ز) تأثیر درصد پیوند \english{-SO₃Li} را نشان می‌دهند. رسانایی یونی با عامل‌دار کردن بیشتر سولفونات افزایش می‌یابد و تأیید می‌کند که گروه‌های \english{-SO₃Li} به عنوان کانال‌های انتقال کارآمد یون لیتیوم عمل می‌کنند. شکل‌های ۱۳(ح) و (ط) توانایی مسدودکنندگی پلی‌سولفید را ارزیابی می‌کنند. با استفاده از یک سلول نفوذ \english{Li₂S₆}، غشای متداول \english{Celgard} در شکل ۱۲(ح) نشتی قابل توجهی را در عرض ۱ ساعت نشان می‌دهد. در مقابل، غشای \english{Li-IL/UIOSLi} در شکل ۱۲(ط) پلی‌سولفیدها را حتی پس از ۷ روز حفظ می‌کند و قابلیت محبوس‌سازی شیمیایی برتر خود را برجسته می‌نماید. در مجموع، این نتایج تأیید می‌کنند که \english{Li-IL/UIOSLi} یک پلتفرم الکترولیت جامد با رسانایی بالا، سرکوب‌کننده دندریت و مسدودکننده پلی‌سولفید مناسب برای باتری‌های لیتیوم-گوگرد با کارایی بالا ارائه می‌دهد.
}

\begin{figure*}[t]
    \centering
    \includegraphics[width=0.95\textwidth]{example-image-b} % Placeholder for Figure 12
    \caption{\persian{
    \farsibold{شکل ۱۲.} (الف)-(ج) ساختارهای مولکولی مواد چارچوب فلزی-آلی (\english{MOF}): (الف) \english{UIO-66} بکر، (ب) \english{UIO} عامل‌دار شده با اسید سولفونیک (\english{UIO-SO₃H})، و (ج) \english{UIO} لیتیومی شده با گروه‌های سولفونات (\english{UIO-SO₃Li}). (د) نمودارهای نایکوئیست طیف‌سنجی امپدانس الکتروشیمیایی (\english{EIS}) و (ه) رسانایی یونی متناظر الکترولیت‌های جامد \english{Li-IL/UIOSLi} با نسبت‌های وزنی مختلف \english{Li-IL} به \english{UIOSLi}. (و) نمودارهای نایکوئیست \english{EIS} و (ز) رسانایی یونی \english{Li-IL/UIOSLi} با درجات مختلف عامل‌دار کردن \english{-SO₃Li}. (ح) آزمون نفوذ پلی‌سولفید (\english{Li₂S₆}) با استفاده از یک غشای متداول \english{Celgard} که نفوذ قابل توجهی را در عرض ۱ ساعت نشان می‌دهد. (ط) آزمون نفوذ پلی‌سولفید (\english{Li₂S₆}) با غشای \english{Li-IL/UIOSLi} که حفظ عالی پلی‌سولفیدها را حتی پس از ۷ روز نشان می‌دهد. بازتولید شده از مرجع ۱۴۷ با مجوز از \english{AAAS}، کپی‌رایت ۲۰۲۴.
    }}
    \label{fig:12}
\end{figure*}

\persian{
باتری‌های تمام-جامد لیتیوم-گوگرد (\english{ASSLSBs}) به دلیل ظرفیت نظری بالا، هزینه کم و ایمنی بهبودیافته در مقایسه با باتری‌های لیتیوم-یون سنتی، از جمله امیدوارکننده‌ترین سیستم‌های ذخیره‌سازی انرژی نسل بعد هستند.$^{۱۶۲}$ با این حال، دو چالش حیاتی مانع از اجرای عملی آنها می‌شود: (۱) رسانایی یونی پایین در الکترولیت‌های پلیمری جامد (\english{SPEs}) و (۲) اثر شاتل ناشی از مهاجرت \english{LiPSs} که منجر به تخریب ظرفیت می‌شود. در یک مطالعه اخیر، لی و همکاران به این موارد پرداختند.
}
% --- End of content for page 17 ---
% --- SECTION 18: Page 18 of Translation ---
% --- Notes: Includes detailed analysis of Figure 13 and introduces self-healing SPEs.
% --------------------------------------------------------------------------

\persian{
این دو چالش دوگانه را از طریق طراحی یک الکترولیت جامد کامپوزیتی نوین (\english{CSE}) به نام \english{3D MPPL CSE}، متشکل از پلی(اتیلن اکسید) (\english{PEO}) و پرکننده‌های \english{MOF} \english{MIL-125-NH₂} منظم، برطرف کردند.$^{۱۶۲}$ با استفاده از تکنیک‌های الکتروریسی و کلندرینگ، نویسندگان به توزیع یکنواخت سه‌بعدی ذرات \english{MOF} در ماتریس پلیمری دست یافتند، ساختاری که به طور قابل توجهی رسانایی یون لیتیوم را افزایش داده و انتقال \english{LiPS} را مسدود می‌کند. معماری منظم \english{MOF} مسیرهای یون لیتیوم بسیار پیوسته و جایگاه‌های باز فراوان تیتانیوم را فراهم می‌کند که تفکیک نمک \english{LiTFSI} را تسهیل کرده و غلظت یون‌های متحرک \english{Li⁺} را افزایش می‌دهد. الکترولیت حاصل، رسانایی یونی بالای \english{8.3 × 10⁻⁴ S cm⁻¹} را در دمای ۶۰ درجه سانتی‌گراد، با عدد انتقال \english{Li⁺} \english{0.81} نشان می‌دهد—هر دو برتر از نمونه‌های کنترلی با \english{MOF}های نامنظم یا بدون \english{MOF} هستند. مهم‌تر از آن، ساختار نانو متخلخل (حدود ۴.۵ آنگستروم) \english{MOF}ها به طور مؤثری از نفوذ پلی‌سولفید جلوگیری می‌کند، همانطور که توسط طیف‌سنجی رامان درجا و آزمایش‌های نفوذ تأیید شده است. علاوه بر این، فصل مشترک بین \english{CSE} و فلز لیتیوم با تشکیل \english{Li₃N} و \english{LiF} پایدار می‌شود و رشد دندریت را سرکوب کرده و عمر چرخه‌ای را افزایش می‌دهد. \english{3D MPPL CSE} چرخه‌زنی پایدار بیش از ۴۰۰ چرخه را در سلول‌های کیسه‌ای \english{ASSLSBs} در \english{0.5C} و ۶۰ درجه سانتی‌گراد امکان‌پذیر می‌سازد. این کار یک پیشرفت قابل توجه را با نشان دادن یک طراحی الکترولیت جامد چندکاره و مقیاس‌پذیر که همزمان انتقال یون را افزایش داده و شاتل پلی‌سولفید را مهار می‌کند—گام‌های کلیدی به سوی تجاری‌سازی \english{ASSLSBs} با چگالی انرژی بالا—ارائه می‌دهد.
}
\persian{
شکل ۱۳ یک تصویرسازی جامع از سنتز، مشخصه‌یابی ساختاری، تحلیل درجا و مقایسه عملکردی \english{CSE}های مختلف مورد استفاده در \english{ASSLSBs} ارائه می‌دهد. نوآوری کلیدی نشان داده شده، استفاده از یک الکترولیت مبتنی بر \english{MOF} منظم (\english{3D MPPL CSE}) است که به طور قابل توجهی انتقال یون را بهبود می‌بخشد در حالی که شاتل \english{LiPS} را سرکوب می‌کند—دو چالش حیاتی در توسعه باتری \english{Li-S}. شکل ۱۳(الف) فرآیند سنتز \english{3D MPPL CSE} را نشان می‌دهد که شامل الکتروریسی مخلوط \english{PAN/MIL-125-NH₂} به یک غشای فیبری (\english{3D MP})، و به دنبال آن کلندرینگ برای متراکم کردن ساختار و نفوذ با محلول \english{PEO/LiTFSI} برای تشکیل الکترولیت کامپوزیتی نهایی است. تصاویر \english{SEM} در شکل‌های ۱۳(ب-د) مورفولوژی ذرات \english{MOF} و معماری فیبری غشای \english{3D MP} را تأیید می‌کنند که یک چارچوب متخلخل و در عین حال مکانیکی مستحکم را حفظ می‌کند. شکل‌های ۱۳(ه) و (و) به صورت مفهومی مسیرهای انتقال \english{Li⁺} را در توزیع‌های \english{MOF} نامنظم (\english{MPL CSE}) در مقابل منظم (\english{3D MPPL CSE}) مقایسه می‌کنند. ساختار نامنظم منجر به مسیرهای رسانش ناپیوسته \english{Li⁺} و شار یونی ناهموار می‌شود، در حالی که ساختار منظم کانال‌های انتقال پیوسته، هم‌تراز و تحرک یونی بهبودیافته را ارائه می‌دهد. تحلیل رامان درجا در شکل‌های ۱۳(ز-ن) بینش مستقیمی در مورد رفتار \english{LiPS} در حین دشارژ ارائه می‌دهد. یک چیدمان سلول نوری سفارشی در شکل‌های ۱۳(ز-ح) امکان نظارت بی‌درنگ بر تحول \english{LiPS} در نزدیکی آند لیتیوم را فراهم می‌کند. نقشه‌های رامان در شکل‌های ۱۳(ط-ک) و طیف‌ها در شکل‌های ۱۳(ل-ن) سیگنال‌های پلی‌سولفید به طور قابل توجهی ضعیف‌تری را در \english{3D MPPL CSE} در مقایسه با \english{PL} و \english{MPL CSEs} نشان می‌دهند که بیانگر سرکوب مؤثر اثر شاتل است. در نهایت، شکل‌های ۱۳(س-ع) به صورت شماتیک رفتار سه الکترولیت را در \english{ASSLSBs} مقایسه می‌کنند. تنها \english{3D MPPL CSE} در شکل ۱۳(س) به طور موفقیت‌آمیزی رسانایی یونی بالا، شیمی فصل مشترک پایدار و محبوس‌سازی \english{LiPS} را به لطف ساختار منظم \english{MOF} خود ترکیب می‌کند. در مقابل، \english{PL CSE} در شکل ۱۳(ص) و \english{MPL CSE} در شکل ۱۳(ع) از نفوذ کنترل‌نشده \english{LiPS} و انتقال ضعیف یون به دلیل عدم سازماندهی ساختاری رنج می‌برند.
}
\persian{
پی و همکاران یک الکترولیت پلیمری خود-ترمیم‌شونده مبتنی بر پلی(اتر-یورتان) (\english{SPE}) را معرفی کرده‌اند که یک راه‌حل امیدوارکننده برای باتری‌های \english{Li-S} حالت جامد با عمر چرخه‌ای طولانی ارائه می‌دهد.$^{۱۶۳}$ \english{SPE} پیشنهادی، پیوندهای دی‌سولفیدی کووالانسی دینامیک و پیوندهای هیدروژنی را ادغام می‌کند و خواص خود-ترمیم‌شوندگی فصل مشترکی استثنایی را امکان‌پذیر می‌سازد. این طراحی مقاومت فصل مشترکی را به حداقل می‌رساند، انتقال یکپارچه یون را ترویج می‌دهد و از تخریب ساختاری در طول چرخه‌زنی جلوگیری می‌کند. \english{SPE} عملکرد الکتروشیمیایی فوق‌العاده‌ای را نشان می‌دهد، از جمله رسانایی یونی \english{2.4 × 10⁻⁴ S cm⁻¹} در دمای ۳۰ درجه سانتی‌گراد که بسیار فراتر از مواد سنتی مانند الکترولیت‌های مبتنی بر پلی(اتیلن اکسید) (\english{PEO}) است. خواص خود-ترمیم‌شوندگی به فصل مشترک اجازه می‌دهد تا به طور خود به خود در دمای اتاق ترمیم شود و پایداری چرخه‌ای طولانی‌مدت و تماس بهتر بین الکترولیت و الکترودها را تضمین کند. این مطالعه استراتژی مونتاژ دوگانه-یکپارچه نوآورانه را برجسته می‌کند که تماس یکنواخت و صمیمی بین \english{SPE} و الکترودها را در مقایسه با طراحی‌های لمینت شده متداول تضمین می‌کند. این رویکرد به طور قابل توجهی کاهش می‌دهد...
}


\begin{figure*}[t]
    \centering
    \includegraphics[width=0.95\textwidth]{example-image-c} % Placeholder for Figure 13
    \caption{\persian{
    \farsibold{شکل ۱۳.} شماتیک یکپارچه و مشخصه‌یابی الکترولیت جامد کامپوزیتی \english{3D MPPL} (\english{CSE}) برای باتری‌های تمام-جامد \english{Li-S} (\english{ASSLSBs}).$^{162}$ (الف) شماتیک فرآیند سنتز \english{3D MPPL CSE} از طریق الکتروریسی، کلندرینگ و نفوذ با \english{PEO/LiTFSI}. (ب) تصویر \english{SEM} از ذرات \english{MIL-125-NH₂}. (ج) و (د) تصاویر \english{SEM} از غشای \english{3D MP} در بزرگنمایی‌های مختلف که شبکه فیبری و ساختار \english{MOF} تعبیه‌شده را نشان می‌دهد. (ه) و (و) مقایسه شماتیک مسیرهای انتقال \english{Li⁺} در الکترولیت مبتنی بر \english{MOF} نامنظم (\english{MPL CSE}) (ه) و الکترولیت مبتنی بر \english{MOF} منظم (\english{3D MPPL CSE}) (و) که بهبود پیوستگی انتقال یون را برجسته می‌کند. (ز) شماتیک چیدمان تجربی رامان درجا با یک شکاف مشاهده نازک (۱-۲ میلی‌متر) در نزدیکی آند لیتیوم برای نظارت بر تحول \english{LiPS}. (ح) پیکربندی سلول الکتروشیمیایی نوری با کاتد \english{KB@S}، الکترولیت جامد و آند فلز لیتیوم که در دمای ۶۰ درجه سانتی‌گراد گرم شده است. (ط)-(ک) نقشه‌برداری رامان درجا با تفکیک زمانی در حین دشارژ باتری برای: (ط) \english{PL CSE}، (ی) \english{MPL CSE}، و (ک) \english{3D MPPL CSE}. (ل)-(ن) طیف‌های رامان متناظر در ولتاژهای انتخاب‌شده برای: (ل) \english{PL CSE}، (م) \english{MPL CSE}، و (ن) \english{3D MPPL CSE} که سیگنال‌های پلی‌سولفید کاهش‌یافته را با ساختار \english{MOF} منظم نشان می‌دهد. (س)-(ع) تصویر شماتیک از عملکرد الکترولیت در \english{ASSLSBs}: (س) \english{3D MPPL CSE} نفوذ پلی‌سولفید سرکوب‌شده و رسانش \english{Li⁺} افزایش‌یافته را نشان می‌دهد؛ (ص) \english{PL CSE} از تفکیک ضعیف \english{Li⁺} و شاتل پلی‌سولفید رنج می‌برد؛ (ع) \english{MPL CSE} سرکوب جزئی را با مسیرهای یونی محدود به دلیل توزیع نامنظم \english{MOF} ارائه می‌دهد. بازتولید شده از مرجع ۱۶۲ با مجوز از \english{Wiley-VCH}، کپی‌رایت ۲۰۲۰.
    }}
    \label{fig:13}
\end{figure*}

% --- End of content for page 18 ---
% --- SECTION 19: Page 19 of Translation ---
% --- Notes: Introduces lithium anode protection strategies.
% --------------------------------------------------------------------------

\persian{
مقاومت تماس فصل مشترکی، همانطور که توسط تصویربرداری فراصوت پیشرفته تأیید شده است. پیکربندی باتری یکپارچه عملکرد قابل توجهی را به دست آورد، با سلول‌های متقارن لیتیومی که برای بیش از ۶۰۰۰ ساعت چرخه‌زنی کردند و سلول‌های کامل \english{Li-S} که ۹۳٪ ظرفیت را پس از ۷۰۰ چرخه در \english{0.3C} حفظ کردند. ظرفیت ویژه بالای \english{812 mAh g⁻¹} برای کاتدهای مبتنی بر گوگرد به دست آمد که نشان‌دهنده افزایش کارایی و طول عمر انرژی است.$^{۱۶۳}$ \english{SPE} همچنین خواص مکانیکی عالی، از جمله استحکام شکست \english{88.3 MPa} و ازدیاد طول ۲۰۰۰٪ را نشان می‌دهد که به عنوان یک مانع مکانیکی برای سرکوب رشد دندریت لیتیوم عمل می‌کند.$^{۱۶۳}$ علاوه بر این، طراحی دوگانه-یکپارچه فرآیند مونتاژ باتری را ساده می‌کند و یک رویکرد مقیاس‌پذیر برای باتری‌های حالت جامد با کارایی بالا ارائه می‌دهد.
}

\section*{\persian{۴. محافظت از آند لیتیوم برای باتری‌های گوگردی پیشرفته}}
\persian{
باتری‌های \english{Li-S} به دلیل چگالی انرژی نظری بالا و صرفه اقتصادی، دستگاه‌های ذخیره‌سازی انرژی نسل بعد امیدوارکننده‌ای هستند.$^{۱۶۴-۱۷۹}$ با این حال، کاربرد عملی باتری‌های \english{Li-S} با چالش‌های شدیدی مرتبط با آندهای لیتیوم، از جمله تشکیل دندریت، واکنش‌های جانبی پارازیتی و فصل مشترک الکترولیت جامد (\english{SEI}) ناپایدار، مواجه است. محافظت مؤثر از آند لیتیوم برای بهبود عمر چرخه‌ای و ایمنی باتری‌های \english{Li-S} ضروری است.
}
\persian{
چندین استراتژی برای محافظت از آندهای لیتیوم مورد بررسی قرار گرفته است. یک رویکرد استفاده از لایه‌های \english{SEI} مصنوعی است که انتقال یون را افزایش می‌دهند در حالی که از واکنش‌های ناخواسته با الکترولیت جلوگیری می‌کنند. لایه‌های فصل مشترک پایدار، مانند پوشش‌های فلوراید لیتیوم (\english{LiF}) یا فیلم‌های محافظ پلیمری، می‌توانند رشد دندریت را سرکوب کرده و بازده کولمبی را بهبود بخشند. روش دیگر شامل مهندسی الکترولیت است که نمک‌های لیتیوم، حلال‌های فلوئورینه یا افزودنی‌هایی را که شار یون لیتیوم را تنظیم کرده و سطح لیتیوم را پایدار می‌کنند، در بر می‌گیرد. علاوه بر این، میزبان‌های لیتیوم ساختاریافته سه‌بعدی، مانند داربست‌های متخلخل تزریق‌شده با لیتیوم یا آندهای آلیاژی لیتیوم، به کاهش چگالی جریان محلی و کاهش نوسانات حجمی کمک می‌کنند. ترکیب این استراتژی‌ها طول عمر و کارایی باتری‌های \english{Li-S} را افزایش می‌دهد و آنها را برای کاربردهای واقعی در وسایل نقلیه الکتریکی و ذخیره‌سازی در مقیاس شبکه قابل‌قبول‌تر می‌سازد. تحقیقات مداوم در زمینه محافظت از آند لیتیوم کلید باز کردن پتانسیل کامل فناوری باتری \english{Li-S} خواهد بود.
}
\persian{
در یک مطالعه پیشین، استفاده از یک لایه محافظ \english{Li₃N} بر روی آند لیتیوم به عنوان یک استراتژی بسیار مؤثر برای افزایش پایداری و طول عمر باتری نشان داده شده است. تشکیل درجا لایه \english{Li₃N} با قرار دادن فلز لیتیوم در معرض گاز نیتروژن در دمای اتاق، به طور قابل توجهی عملکرد الکتروشیمیایی را با سرکوب واکنش‌های جانبی ناخواسته، کاهش پلاریزاسیون و مهار رشد دندریت لیتیوم بهبود می‌بخشد.$^{۱۷۹}$ لایه \english{Li₃N} نه تنها از تماس مستقیم بین پلی‌سولفیدهای لیتیوم و آند لیتیوم جلوگیری می‌کند و در نتیجه اثر شاتل را کاهش می‌دهد، بلکه به دلیل رسانایی یونی بالای خود (حدود \english{10⁻³ S cm⁻¹}) نفوذ سریع‌تر یون لیتیوم را نیز تسهیل می‌کند. آزمون‌های الکتروشیمیایی نشان می‌دهند که باتری‌های \english{Li-S} با آندهای محافظت‌شده با \english{Li₃N} ظرفیت دشارژ بالای \english{773 mAh g⁻¹} را پس از ۵۰۰ چرخه در \english{0.5C} با بازده کولمبی متوسط ۹۲.۳٪ در غیاب نیترات لیتیوم (\english{LiNO₃})، که معمولاً استفاده می‌شود اما به تدریج تجزیه شده و بر عملکرد بلندمدت تأثیر می‌گذارد، نشان می‌دهند. علاوه بر این، نتایج طیف‌سنجی امپدانس نشان می‌دهد که \english{SEI} اصلاح‌شده با \english{Li₃N} پایدار باقی می‌ماند و افزایش مقاومت در طول چرخه‌زنی طولانی‌مدت را کاهش می‌دهد. مطالعات مورفولوژیکی تأیید می‌کنند که ضخامت لایه عایق \english{Li₂S/Li₂S₂} بر روی یک آند محافظت‌شده تنها حدود ۱۰ میکرومتر است، در مقایسه با حدود ۱۰۰ میکرومتر بر روی یک آند لیتیوم محافظت‌نشده، که کارایی اصلاح \english{Li₃N} را بیشتر تأیید می‌کند. علاوه بر این، این رویکرد مسائل خود-دشارژی، یک نقص حیاتی در سیستم‌های \english{Li-S}، را کاهش می‌دهد. در مجموع، این مطالعه پتانسیل پوشش‌های \english{Li₃N} را به عنوان یک راه‌حل عملی و مقیاس‌پذیر برای باتری‌های \english{Li-S} با کارایی بالا برجسته می‌کند و پایداری چرخه‌ای بهبودیافته، ایمنی افزایش‌یافته و طول عمر طولانی‌تری را ارائه می‌دهد و آن را به یک جایگزین جذاب برای استراتژی‌های محافظت از لیتیوم مبتنی بر الکترولیت تبدیل می‌کند.$^{۱۷۹}$
}
\persian{
شو و همکاران دو استراتژی برای کاهش خود-دشارژی پیشنهاد کردند: غیرفعال‌سازی آند با استفاده از نیترات لیتیوم (\english{LiNO₃}) و سرکوب نفوذ پلی‌سولفید از طریق یک جداکننده با پوشش نفیون.$^{۱۷۸}$ افزودن \english{LiNO₃} یک فصل مشترک الکترولیت جامد پایدار (\english{SEI}) بر روی آند فلز لیتیوم تشکیل می‌دهد که واکنش پلی‌سولفید را کاهش داده و به ثابت شاتل فوق‌العاده پایین \english{0.017 h⁻¹} دست می‌یابد. این رویکرد حفظ چشمگیر ۹۷٪ از ظرفیت دشارژ اولیه (\english{961 mAh g⁻¹}) را پس از ۱۲۰ روز در مدار باز امکان‌پذیر می‌سازد. در همین حال، جداکننده نفیون که به عنوان یک غشای کاتیون-گزین عمل می‌کند، به طور مؤثری نفوذ پلی‌سولفید را مسدود کرده و در نتیجه بازده کولمبی را افزایش می‌دهد (حدود ۱۰۰٪) و اثرات شاتل را به حداقل می‌رساند، هرچند پلاریزاسیون جزئی را القا می‌کند. تحلیل مقایسه‌ای نشان می‌دهد که محافظت از آند از طریق \english{LiNO₃} از کنترل نفوذ بهتر عمل می‌کند، زیرا ثابت شاتل برای جداکننده نفیون در \english{0.040 h⁻¹} باقی می‌ماند، هرچند به طور قابل توجهی پایین‌تر از سلول‌های \english{Li-S} معمولی (\english{0.114-0.218 h⁻¹}) است.$^{۱۷۸}$ نتایج نشان می‌دهد که ترکیب هر دو استراتژی می‌تواند پایداری باتری را بیشتر افزایش دهد. این مطالعه بر اهمیت طراحی افزودنی‌های الکترولیت پیشرفته، غشاهای یون-گزین و معماری‌های سلولی بهینه‌شده برای سرکوب خود-دشارژی در باتری‌های \english{Li-S} تأکید می‌کند.
}
\persian{
این مطالعه نقش پلی‌سولفیدهای لیتیوم (\english{LiPS}) را در پایدارسازی \english{SEI} بر روی آند فلز لیتیوم بررسی کرد و یک ترکیب الکترولیت بهینه‌شده حاوی \english{0.020 M Li₂S₅} (\english{0.10 M} گوگرد) و \english{5.0 wt\% LiNO₃} را پیشنهاد داد.$^{۱۷۷}$ یافته‌ها نشان می‌دهند که این فرمولاسیون الکترولیت یک \english{SEI} غنی از \english{LiF-Li₂Sₓ} را درجا تشکیل می‌دهد که به طور قابل توجهی رشد دندریت لیتیوم را سرکوب می‌کند، خوردگی لیتیوم را کاهش می‌دهد و پایداری چرخه‌ای بلندمدت را افزایش می‌دهد. آزمون‌های الکتروشیمیایی در نیم-سلول‌های \english{Li-Cu} نشان می‌دهد که الکترولیت بهینه‌شده بازده کولمبی پایداری ۹۵٪ را طی ۲۳۳ چرخه با مورفولوژی لیتیوم بدون دندریت حفظ می‌کند. این مطالعه بیشتر تأیید می‌کند که در غلظت‌های پایین \english{LiPS} (کمتر از \english{0.05 M} گوگرد)، تشکیل \english{SEI} ناقص است و منجر به محافظت ضعیف از لیتیوم می‌شود، در حالی که غلظت‌های بالا (بیش از \english{0.50 M} گوگرد) اچینگ لیتیوم و تشکیل دندریت را تسریع کرده و در نهایت عملکرد سلول را تخریب می‌کند. این نتایج تعادل ظریف مورد نیاز در غلظت \english{LiPS} را برای افزایش پایداری فلز لیتیوم و در عین حال کاهش واکنش‌های جانبی برجسته می‌کنند. بینش‌های مکانیکی ارائه‌شده توسط این کار...
}

% --- End of content for page 19 ---
% --- SECTION 20: Page 20 of Translation ---
% --- Notes: Discusses advanced anode protection (CIL, DMCE, DCPs, 3D alloys).
% --------------------------------------------------------------------------

\persian{
نشان می‌دهد که تنظیم دقیق انحلال و نفوذ پلی‌سولفید—بالقوه از طریق چارچوب‌های رسانا و لایه‌های میانی نفوذ-گزین—می‌تواند کاربرد عملی سلول‌های \english{Li-S} با بارگذاری بالا را امکان‌پذیر سازد. با نشان دادن اینکه یک لایه \english{SEI} غنی از \english{LiF} که درجا تشکیل می‌شود، به طور مؤثری آند لیتیوم را پایدار می‌کند، این مطالعه یک پیشرفت قابل توجه در مهندسی الکترولیت برای باتری‌های \english{Li-S} ارائه می‌دهد. کارهای آینده باید بر بهینه‌سازی بیشتر فرمولاسیون‌های الکترولیت، توسعه جداکننده‌های پیشرفته و ادغام پوشش‌های فصل مشترکی نوین برای باز کردن کامل پتانسیل باتری‌های \english{Li-S} برای نسل بعدی سیستم‌های ذخیره‌سازی انرژی متمرکز شود.$^{۱۷۷}$
}
\persian{
در مطالعه یائو و همکاران، یک لایه غیرآلی فشرده (\english{CIL}) به عنوان یک استراتژی محافظتی نوین برای پرداختن به این مسائل معرفی شد.$^{۱۷۱}$ \english{CIL} که به صورت برون‌جا (\english{ex situ}) از طریق واکنش فلز لیتیوم با یک پیش‌ساز مایع یونی (\english{0.5 M LiFSI} در \english{[BMIM][NO₃]}) تشکیل می‌شود، یک \english{SEI} پایدار غنی از \english{LiF} ایجاد می‌کند که به طور مؤثری مهاجرت پلی‌سولفید را مسدود کرده و از خوردگی لیتیوم جلوگیری می‌کند. میکروسکوپ الکترونی روبشی (\english{SEM}) و طیف‌سنجی فوتوالکترون اشعه ایکس (\english{XPS}) تشکیل یک لایه ضخیم ۴۰۰ نانومتری، فشرده و یکنواخت را تأیید کردند که عمدتاً از \english{LiF}, \english{Li₂O} و \english{Li₃N} تشکیل شده است و نه تنها پایداری مکانیکی را افزایش می‌دهد، بلکه انتقال یون لیتیوم را نیز تسهیل می‌کند. آزمون‌های الکتروشیمیایی نشان دادند که باتری‌های \english{Li-S} با آندهای محافظت‌شده با \english{CIL} به بازده کولمبی ۹۶.۷٪ در مقایسه با ۸۲.۴٪ برای لیتیوم لخت دست یافتند، با حفظ ظرفیت ۸۰.۶٪ پس از ۱۰۰ چرخه، که به طور قابل توجهی از آندهای لیتیوم محافظت‌نشده بهتر عمل می‌کرد.$^{۱۷۱}$ علاوه بر این، آزمون‌های چرخه‌زنی بلندمدت در سلول‌های متقارن \english{Li-Li} نشان داد که \english{CIL} به طور مؤثری رشد دندریت لیتیوم را سرکوب می‌کند و چرخه‌زنی پایدار را برای ۴۷۰ ساعت در \english{5 mA cm⁻²} تضمین می‌کند، در حالی که لیتیوم لخت به دلیل پلاریزاسیون بیش از حد زود از کار افتاد. \english{CIL} همچنین جریان شاتل را شش برابر کاهش داد و توانایی خود را در سرکوب واکنش‌های پارازیتی بین لیتیوم و پلی‌سولفیدها تأیید کرد. این رویکرد مقیاس‌پذیر و ساده، یک جهت امیدوارکننده برای افزایش پایداری آند فلز لیتیوم ارائه می‌دهد و پیامدهای عملی برای تجاری‌سازی باتری \english{Li-S} و طراحی‌های باتری لیتیوم بدون آند دارد. با توجه به کارایی آن در سرکوب عبور \english{LiPS} و پایدارسازی رسوب لیتیوم، مفهوم \english{CIL} می‌تواند به سایر فناوری‌های باتری مبتنی بر آند فلزی، از جمله باتری‌های استوانه‌ای غنی از نیکل، گسترش یابد و به طور بالقوه راه‌حل‌های ذخیره‌سازی انرژی را برای وسایل نقلیه الکتریکی و کاربردهای در مقیاس شبکه متحول کند.
}
\persian{
کونگ و همکاران یک الکترولیت رقیق با غلظت متوسط نوین (\english{DMCE}) را با ادغام ۱,۱,۲,۲-تترافلوئورواتیل ۲,۲,۳,۳-تترافلوئوروپروپیل اتر (\english{TTE}) در یک سیستم الکترولیت دو-نمکی متشکل از لیتیوم بیس(تری‌فلوئورومتان‌سولفونیل)ایمید (\english{LiTFSI})، لیتیوم بیس(فلوئوروسولفونیل)ایمید (\english{LiFSI})، تتراهیدروفوران (\english{THF}) و دی‌پروپیل اتر (\english{DPE}) توسعه دادند.$^{۱۶۸}$ فرمولاسیون بهینه‌شده \english{DMCE-2.0m} به طور مؤثری انحلال پلی‌سولفید لیتیوم را سرکوب می‌کند، اثر شاتل را کاهش می‌دهد و محافظت از آند لیتیوم را با تسهیل تشکیل یک لایه \english{SEI} پایدار و غنی از فلوئور افزایش می‌دهد. این الکترولیت به بازده کولمبی چشمگیر ۹۹.۶٪ برای آبکاری/کنده‌کاری لیتیوم دست می‌یابد که به طور قابل توجهی از الکترولیت‌های مبتنی بر اتر متداول بهتر است. علاوه بر این، سلول‌های \english{Li-S} با استفاده از \english{DMCE-2.0m} ظرفیت دشارژ اولیه بالای \english{682 mAh g⁻¹} را نشان می‌دهند و ۹۲٪ حفظ ظرفیت را طی ۵۰۰ چرخه با افت ظرفیت حداقل \english{0.016\%} در هر چرخه حفظ می‌کنند.$^{۱۶۸}$ علاوه بر این، طراحی جدید الکترولیت با موفقیت اثرات خود-دشارژی را به حداقل می‌رساند و پایداری بلندمدت باتری را تضمین می‌کند. عملکرد الکتروشیمیایی برتر به ساختار حلال‌پوشی بهینه‌شده، خواص انتقال یون لیتیوم بهبودیافته و ترشوندگی افزایش‌یافته الکترولیت نسبت داده می‌شود. این یافته‌ها پتانسیل الکترولیت‌های رقیق مبتنی بر اتر فلوئورینه را به عنوان یک راه‌حل عملی و مقیاس‌پذیر برای نسل بعدی \english{LSBs} برجسته می‌کنند و به چالش‌های کلیدی مرتبط با عمر چرخه‌ای، عملکرد نرخ و پایداری الکترولیت می‌پردازند. استراتژی الکترولیت پیشنهادی ممکن است به سایر سیستم‌های باتری فلز لیتیوم نیز گسترش یابد و مسیرهای جدیدی برای توسعه فناوری‌های ذخیره‌سازی انرژی با کارایی بالا ارائه دهد.$^{۱۶۸}$
}
\persian{
دی‌کلروپیرازین‌های ایزومری (\english{DCPs}) به عنوان افزودنی‌های الکترولیت دوکاره برای افزایش همزمان پایداری آند لیتیوم و سینتیک ردوکس گوگرد معرفی شده‌اند.$^{۱۶۴}$ در میان ایزومرهای آزمایش‌شده (\english{2,3-DCP}, \english{2,5-DCP}, \english{2,6-DCP})، \english{2,5-DCP} عملکرد برتری را نشان داد که به فعالیت ردوکس قوی، شکاف انرژی اوربیتال مولکولی پایین‌تر و برهمکنش قوی‌تر با پلی‌سولفیدهای لیتیوم (\english{LiPSs})، همانطور که توسط محاسبات \english{DFT} تأیید شده است، نسبت داده می‌شود. وجود \english{DCPs} تشکیل یک \english{SEI} غنی از \english{LiCl} را بر روی آند لیتیوم ترویج می‌دهد، رشد دندریت را سرکوب کرده و رسوب یکنواخت لیتیوم را تضمین می‌کند که منجر به پایداری چرخه‌ای بلندمدت با حفظ ظرفیت ۸۶.۶٪ پس از ۸۰۰ چرخه در \english{1C} می‌شود. علاوه بر این، \english{DCPs} سینتیک تبدیل \english{LiPS} را از طریق تشکیل پیوند \english{Li-N} تسریع می‌کنند، مانع انرژی برای هسته‌زایی \english{Li₂S} را کاهش داده و بهره‌برداری از گوگرد را بهبود می‌بخشند. آزمون‌های الکتروشیمیایی تأیید می‌کنند که \english{LSBs} با \english{2,5-DCP} به ظرفیت دشارژ بالای \english{818 mAh g⁻¹} دست می‌یابند و اورپتانسیل پایینی (حدود ۲۸ میلی‌ولت) را پس از ۸۰۰ ساعت در \english{1 mA cm⁻²} نشان می‌دهند. علاوه بر این، باتری‌های با \english{2,5-DCP} از افزودنی‌های متداول مانند \english{LiNO₃} و \english{LiCl} در پایدارسازی هر دو الکترود بهتر عمل می‌کنند. این مطالعه اهمیت افزودنی‌های الکترولیت هیبریدی آلی-غیرآلی را در افزایش کارایی کلی \english{LSBs} برجسته می‌کند و راه را برای استراتژی‌های مهندسی الکترولیت ساده‌تر و در عین حال بسیار مؤثر هموار می‌سازد. یافته‌ها نشان می‌دهند که غربالگری مولکولی مبتنی بر \english{DFT} می‌تواند به طراحی فرمولاسیون‌های الکترولیت پیشرفته برای پرداختن به محدودیت‌های ذاتی \english{LSBs} کمک کند و آنها را برای کاربردهای تجاری در وسایل نقلیه الکتریکی و ذخیره‌سازی انرژی در مقیاس شبکه قابل‌قبول‌تر سازد.
}
\persian{
مطالعه انجام‌شده توسط دونگ و همکاران یک راه‌حل نوآورانه معرفی می‌کند: یک آند فویل گرافنی/آلیاژ \english{Li-Sn} سه‌بعدی (\english{3D Li-Sn alloy/GF}) که این تنگناهای حیاتی را برطرف می‌کند.$^{۱۸۰}$ این آند یک آلیاژ \english{Li-Sn} را با یک چارچوب گرافنی سه‌بعدی ترکیب می‌کند. گرافن یک داربست مکانیکی مستحکم و رسانا فراهم می‌کند و تماس پایدار با الکترولیت حالت جامد (\english{SSE}) را تضمین می‌کند. این آلیاژ جایگاه‌های هسته‌زایی \english{Li₂₂Sn₅} را در خود جای می‌دهد که رسوب یکنواخت لیتیوم را هدایت کرده، تشکیل دندریت را به حداقل رسانده و تغییرات حجمی را در طول چرخه‌زنی کاهش می‌دهد. چنین طراحی‌ای نفوذ یون لیتیوم را افزایش می‌دهد در حالی که یکپارچگی ساختاری را حفظ می‌کند، یک ویژگی حیاتی برای پایداری چرخه‌ای بلندمدت. معیارهای عملکرد، نویدبخش بودن این ماده را برجسته می‌کنند: ظرفیت ویژه \english{3282 mAh g⁻¹}، طول عمر طولانی ۱۲۰۰ ساعت و چگالی جریان بحرانی بالای \english{4 mA cm⁻²}. علاوه بر این، \english{ASSLSBs} مبتنی بر این آند، حفظ ظرفیت قابل توجه ۹۷.۸٪ را طی ۲۳۵ چرخه در دمای اتاق نشان می‌دهند.
}

% --- End of content for page 20 ---
% --- SECTION 21: Page 21 of Translation ---
% --- Notes: Discusses LiAl alloys and introduces Figure 14 for Li2S protective layers.
% --------------------------------------------------------------------------

\persian{
و به ظرفیت‌های مساحتی بالایی (بیش از \english{3 mAh cm⁻²}) با بارگذاری گوگرد \english{3.53 mg cm⁻²} دست می‌یابند.$^{۱۸۰}$
}
\persian{
یک مطالعه اخیر آند آلیاژی کامپوزیتی \english{β-LiAl + α-Al} را که از طریق آبکاری الکتریکی ساخته شده بود، به عنوان یک جایگزین ایمن‌تر و پایدارتر بررسی کرد.$^{۱۸۱}$ آند آلیاژ لیتیوم-آلومینیوم (\english{LiAl}) از مکانیزم‌های نفوذ لیتیوم برای افزایش ایمنی بهره می‌برد که با واکنش‌پذیری پایین آن در آزمون‌های غوطه‌وری در آب مشهود است. این آلیاژ لایه‌های سطحی محافظی تشکیل می‌دهد و خطرات اتصال کوتاه را کاهش داده و یکپارچگی ساختاری را حفظ می‌کند. فرآیند لیتیاسیون/دلیتیاسیون که توسط نفوذ لیتیوم کنترل می‌شود، رشد دندریت را سرکوب کرده و بهره‌برداری یکنواخت از سطح الکترود را تضمین می‌کند و پایداری الکترود را در طول چرخه‌زنی طولانی‌مدت به طور قابل توجهی بهبود می‌بخشد. در پیکربندی‌های نیم-سلولی، الکترود \english{LiAl} از آندهای لیتیومی سنتی بهتر عمل کرد و پایداری چرخه‌ای استثنایی را به ویژه در دانسیته‌های جریان بالاتر نشان داد. هنگامی که با یک کاتد گوگرد در چیدمان‌های سلول کامل جفت شد، باتری \english{LiAl || S} ولتاژ کاری جابجاشده‌ای را به دلیل پتانسیل الکترود بالاتر \english{LiAl} (\english{0.34 V} در مقابل \english{Li/Li⁺}) نشان داد. در حالی که ظرفیت‌های دشارژ اولیه پایین‌تر از آندهای فلز لیتیوم بود، باتری \english{LiAl || S} حفظ ظرفیت بلندمدت برتری را به نمایش گذاشت و آن را برای کاربردهای عملی بسیار امیدوارکننده ساخت. چالش‌ها باقی می‌مانند، از جمله به دام افتادن نفوذی لیتیوم و پودر شدن الکترود در طول چرخه‌زنی طولانی‌مدت که منجر به اتلاف ظرفیت و «لیتیوم-آلومینیوم مرده» می‌شود. بهینه‌سازی‌های آینده، مانند اصلاح خواص سطحی آلیاژ و کاهش تخریب ساختاری، برای تحقق کامل پتانسیل آلیاژ \english{LiAl} ضروری است.$^{۱۸۱}$
}
\persian{
این مطالعه یک لایه محافظ نوین \english{Li₂S} با رسانایی یونی بالا و ترکیب همگن را برای پایدارسازی آند فلز لیتیوم معرفی کرد.$^{۱۸۲}$ شکل ۱۴ طراحی، اصل کار و عملکرد الکتروشیمیایی الکترود \english{Li₂S@Li} را نشان می‌دهد که برای افزایش پایداری آندهای فلز لیتیوم (\english{LMAs}) با کاهش رشد دندریت و بهبود عملکرد چرخه‌ای توسعه یافته است.$^{۱۸۲}$ شکل ۱۴(الف) فرآیند ساخت الکترود \english{Li₂S@Li} را به تصویر می‌کشد، جایی که فلز لیتیوم نقره‌ای-رنگ در معرض اتمسفر بخار گوگرد قرار می‌گیرد. این واکنش گاز-جامد منجر به تشکیل یک لایه محافظ \english{Li₂S} یکنواخت و طلایی-رنگ می‌شود که به عنوان یک فصل مشترک الکترولیت جامد (\english{SEI}) مصنوعی عمل می‌کند. پوشش \english{Li₂S} پایداری آند را با افزایش رسانایی یونی، تضمین رسوب یکنواخت لیتیوم و جلوگیری از واکنش‌های جانبی نامطلوب با الکترولیت افزایش می‌دهد. شکل ۱۴(ب) رفتار رسوب لیتیوم را تحت شرایط مختلف \english{SEI} مقایسه می‌کند. یک \english{SEI} با رسانایی یونی بالا (مانند لایه \english{Li₂S}) رسوب یکنواخت لیتیوم را ترویج می‌دهد و منجر به هسته‌های لیتیومی کروی با اندازه هسته بزرگتر می‌شود. این امر منجر به آبکاری/کنده‌کاری یکنواخت لیتیوم، به حداقل رساندن رشد دندریت و جلوگیری از شکست \english{SEI} می‌شود. در مقابل، یک \english{SEI} با رسانایی یونی پایین، مانند کربنات لیتیوم یا \english{LiF}، منجر به رسوب غیریکنواخت لیتیوم می‌شود و ساختارهای لیتیومی سوزنی-شکل با اندازه هسته کوچکتر تشکیل می‌دهد. این رشته‌های لیتیومی می‌توانند به \english{SEI} نفوذ کرده، باعث شکستگی شده و منجر به اتصالات کوتاه داخلی شوند. علاوه بر این، هنگامی که یک \english{SEI} از چندین جزء با رسانایی‌های یونی متفاوت تشکیل شده باشد، تفاوت در حجم رسوب لیتیوم تنش‌های موضعی ایجاد کرده و شکست \english{SEI} را بیشتر تسریع می‌کند. شکل ۱۴(ج) عملکرد الکتروشیمیایی الکترود \english{Li₂S@Li} را در مقایسه با الکترودهای لیتیوم خالص و لیتیوم پیش‌کاشته با \english{Li₂S₈} در سلول‌های متقارن تحت دانسیته جریان \english{2 mA cm⁻²} و ظرفیت آبکاری/کنده‌کاری \english{5 mAh cm⁻²} ارائه می‌دهد. الکترود \english{Li₂S@Li} (منحنی قرمز) پایداری بلندمدت با حداقل نوسانات ولتاژ را نشان می‌دهد و کارایی خود را در سرکوب رشد دندریت تأیید می‌کند. الکترود لیتیوم پیش‌کاشته با \english{Li₂S₈} (منحنی بنفش) افزایش اورپتانسیل را در طول زمان نشان می‌دهد که بیانگر تخریب \english{SEI} و اتلاف تدریجی لیتیوم است. در همین حال، الکترود لیتیوم خالص (منحنی سیاه) افزایش سریع اورپتانسیل را تجربه می‌کند و در نهایت منجر به خرابی اتصال کوتاه می‌شود و ناپایداری خود را به دلیل رشد دندریتی لیتیوم و شکست \english{SEI} برجسته می‌کند. در مجموع، شکل ۱۴ بر اهمیت یک \english{SEI} یکنواخت و با رسانایی یونی بالا برای پایدارسازی آندهای فلز لیتیوم تأکید می‌کند. لایه محافظ \english{Li₂S@Li} به طور قابل توجهی یکنواختی آبکاری/کنده‌کاری لیتیوم را بهبود می‌بخشد، عمر چرخه‌ای را افزایش می‌دهد و خرابی‌های مرتبط با دندریت را کاهش می‌دهد و پتانسیل خود را برای نسل بعدی باتری‌های فلز لیتیوم نشان می‌دهد.$^{۱۸۲}$
}

\begin{figure*}[t]
    \centering
    \includegraphics[width=0.95\textwidth]{example-image-c} % Placeholder for Figure 14
    \caption{\persian{
    \farsibold{شکل ۱۴.} استراتژی طراحی برای الکترود \english{Li₂S@Li} و عملکرد الکتروشیمیایی آن. (الف) فرآیند ساخت الکترود \english{Li₂S@Li}: فلز لیتیوم نقره‌ای-رنگ در معرض اتمسفر بخار گوگرد قرار می‌گیرد و منجر به تشکیل یک لایه محافظ \english{Li₂S} یکنواخت و طلایی-رنگ از طریق یک واکنش گاز-جامد می‌شود. (ب) تأثیر رسانایی \english{SEI} بر رسوب لیتیوم: یک \english{SEI} با رسانایی یونی بالا، رسوب یکنواخت لیتیوم را با اندازه هسته بزرگتر (\english{r_high}) ترویج می‌دهد و خطر تشکیل دندریت را کاهش می‌دهد. در مقابل، یک \english{SEI} با رسانایی یونی پایین منجر به رشته‌های لیتیومی سوزنی-شکل با اندازه هسته کوچکتر (\english{r_low}) می‌شود که به دلیل حجم رسوب ناهموار می‌توانند به راحتی در \english{SEI} نفوذ کرده و آن را بشکنند. (ج) مقایسه عملکرد الکتروشیمیایی: پروفایل‌های ولتاژ سلول‌های متقارن با استفاده از الکترودهای \english{Li₂S@Li}، لیتیوم خالص و لیتیوم پیش‌کاشته با \english{Li₂S₈} تحت دانسیته جریان \english{2 mA cm⁻²} و ظرفیت \english{5 mAh cm⁻²}. الکترود \english{Li₂S@Li} چرخه‌زنی پایداری را در یک دوره طولانی نشان می‌دهد، در حالی که الکترودهای لیتیوم خالص و پیش‌کاشته با \english{Li₂S₈} افزایش اورپتانسیل و خرابی نهایی اتصال کوتاه را تجربه می‌کنند. بازتولید شده از مرجع ۱۸۲ با مجوز از \english{Wiley-VCH}، کپی‌رایت ۲۰۱۹.
    }}
    \label{fig:14}
\end{figure*}

\persian{
در یک مطالعه اخیر، یک هتروساختار نوین \english{ZnSe-CoSe₂} که در یک چارچوب کربنی دوپ‌شده با نیتروژن با ساختار پوسته-زرده (\english{ZnSe-CoSe₂@NC}) تعبیه شده بود، به عنوان یک میزبان دوکاره «دو در یک» برای هر دو کاتد گوگرد و آند فلز لیتیوم توسعه یافت.$^{۱۶۷}$ شکل ۱۵ طراحی، کارکرد...
}

% --- End of content for page 21 ---
% --- SECTION 22: Page 22 of Translation ---
% --- Notes: Includes Figure 15, the overall conclusion, and author contributions.
% --------------------------------------------------------------------------

\persian{
اصل و عملکرد الکتروشیمیایی باتری کامل \english{Li-S} با استفاده از هتروساختار \english{ZnSe-CoSe₂@NC} به عنوان یک میزبان دو در یک برای هر دو کاتد و آند را نشان می‌دهد.$^{۱۶۷}$ در شکل ۱۵(الف)، شماتیک فرآیند سنتز \english{ZnSe-CoSe₂@NC} را از یک پیش‌ساز \english{ZIF-8-ZIF-67} برجسته می‌کند و نقش آن را در محافظت از کاتد با تسریع واکنش‌های ردوکس پلی‌سولفید و مهار اثر شاتل نشان می‌دهد. همزمان، رسوب یکنواخت لیتیوم را در آند با هدایت هسته‌زایی لیتیوم و سرکوب رشد دندریت از طریق واکنش‌های شیمیایی درجا امکان‌پذیر می‌سازد، جایی که \english{ZnSe} و \english{CoSe₂} با لیتیوم واکنش داده و \english{Li₂Se} را به همراه \english{Zn} و \english{Co} فلزی تشکیل می‌دهند که به عنوان جایگاه‌های هسته‌زایی ترجیحی لیتیوم عمل می‌کنند. شکل‌های ۱۵(ب) و (ج) مشاهدات درجا از رسوب لیتیوم را ارائه می‌دهند. شکل ۱۵(ب) رشد دندریت لیتیوم را در طول زمان بر روی یک الکترود لیتیوم لخت نشان می‌دهد، جایی که رسوب کنترل‌نشده منجر به تشکیل ساختارهای سوزنی-شکل می‌شود. در مقابل، شکل ۱۵(ج) نشان می‌دهد که \english{Li/ZnSe-CoSe₂@NC} یک رسوب صاف و یکنواخت را حفظ می‌کند و توانایی خود را در تنظیم رشد لیتیوم و جلوگیری از تشکیل دندریت، که ایمنی و طول عمر باتری را افزایش می‌دهد، تأیید می‌کند. عملکرد الکتروشیمیایی سلول کامل \english{Li-S} مبتنی بر \english{ZnSe-CoSe₂@NC} در شکل‌های ۱۵(د)-(ط) ارائه شده است. شکل ۱۵(د) منحنی‌های ولتامتری چرخه‌ای (\english{CV}) را در نرخ‌های اسکن مختلف نشان می‌دهد و پیک‌های ردوکس متمایزی را که با واکنش‌های تبدیل گوگرد مرتبط هستند، آشکار می‌سازد که نشان‌دهنده جذب کارآمد پلی‌سولفید و تبدیل کاتالیزوری است. شکل ۱۵(ه) نمودارهای نایکوئیست را ارائه می‌دهد که سلول کامل بکر را با سایر پیکربندی‌ها مقایسه می‌کند، جایی که مقاومت انتقال بار پایین‌تر (\english{Rct}) \english{ZnSe-CoSe₂@NC} رسانایی الکترونیکی برتر و نفوذ سریع یون لیتیوم را تأیید می‌کند. شکل ۱۵(و) قابلیت نرخ سلول کامل را نمایش می‌دهد و ظرفیت‌های پایداری را در دانسیته‌های جریان فزاینده از \english{0.1} تا \english{5C} با حداقل تخریب هنگام بازگشت به نرخ‌های \english{C} پایین‌تر نشان می‌دهد. شکل ۱۵(ز) منحنی‌های شارژ-دشارژ گالوانواستاتیک را در دانسیته‌های جریان مختلف نشان می‌دهد و پلاتوهای به خوبی تعریف‌شده را حتی در نرخ‌های بالا حفظ می‌کند و سینتیک ردوکس سریع فعال‌شده توسط هتروساختار را بیشتر تأیید می‌کند. شکل‌های ۱۵(ح) و (ط) عملکرد چرخه‌ای بلندمدت را به ترتیب در \english{1C} و \english{2C} نشان می‌دهند. سلول کامل ظرفیت ویژه بالایی حدود \english{730 mAh g⁻¹} را در \english{1C} پس از ۴۰۰ چرخه و حدود \english{477 mAh g⁻¹} را در \english{2C} پس از ۱۰۰۰ چرخه، با بازده کولمبی عالی، حفظ می‌کند و پایداری فوق‌العاده و امکان‌سنجی عملی \english{ZnSe-CoSe₂@NC} را برای باتری‌های \english{Li-S} برجسته می‌کند.$^{۱۶۷}$
}

\begin{figure*}[t]
    \centering
    \includegraphics[width=0.95\textwidth]{example-image-b} % Placeholder for Figure 15
    \caption{\persian{
    \farsibold{شکل ۱۵.} (الف) اصل کار باتری کامل \english{Li-S} مبتنی بر میزبان‌های «دو در یک» \english{ZnSe-CoSe₂@NC}. (ب) مشاهدات درجا از سلول‌های متقارن با لیتیوم خالص در حال چرخه‌زنی با دانسیته جریان \english{3 mA cm⁻²}. (ج) مشاهدات درجا از سلول‌های متقارن با الکترود \english{Li/ZnSe-CoSe₂@NC} در حال چرخه‌زنی با دانسیته جریان \english{3 mA cm⁻²}. مقیاس‌ها: ۵۰۰ میکرومتر. (د) منحنی‌های \english{CV} سلول کامل \english{S/ZnSe-CoSe₂@NC || Li/ZnSe-CoSe₂@NC} در نرخ‌های اسکن مختلف. (ه) نمودارهای نایکوئیست سلول کامل بکر. (و) عملکرد نرخ سلول کامل از \english{0.1} تا \english{5C}. (ز) پروفایل‌های شارژ-دشارژ گالوانواستاتیک سلول کامل در دانسیته‌های جریان مختلف از \english{0.1} تا \english{5C}. (ح) عملکرد چرخه‌ای سلول کامل در \english{1C}. (ط) عملکرد چرخه‌ای سلول کامل در \english{2C}. بازتولید شده از مرجع ۱۶۷ با مجوز از \english{Elsevier Ltd.}، کپی‌رایت ۲۰۲۲.
    }}
    \label{fig:15}
\end{figure*}

\section*{\persian{۵. نتیجه‌گیری و چشم‌انداز آینده}}
\persian{
باتری‌های \english{Li-S} یک فناوری ذخیره‌سازی انرژی نسل بعد امیدوارکننده را نمایندگی می‌کنند که چگالی انرژی نظری به طور قابل توجهی بالاتر، هزینه‌های تولید پایین‌تر و پایداری افزایش‌یافته را به دلیل فراوانی طبیعی گوگرد ارائه می‌دهند. این ویژگی‌ها باتری‌های \english{Li-S} را به عنوان نامزدهای قوی برای پیشی گرفتن از باتری‌های لیتیوم-یون متداول، به ویژه برای کاربردهایی که نیازمند چگالی انرژی بالا و سازگاری زیست‌محیطی هستند، معرفی می‌کنند. علی‌رغم پتانسیل آنها، تجاری‌سازی گسترده باتری‌های \english{Li-S} همچنان با چندین چالش حیاتی مواجه است. اصلی‌ترین آنها اثر شاتل پلی‌سولفید، رسانایی الکتریکی ذاتاً پایین گوگرد و محصولات دشارژ آن، ناپایداری آند فلز لیتیوم و خطرات ایمنی مرتبط با رشد دندریت و فرار حرارتی است. پرداختن به این مسائل نیازمند یک رویکرد چند رشته‌ای است که نوآوری مواد، مهندسی الکتروشیمیایی و یکپارچه‌سازی عملی در سطح سیستم را ترکیب می‌کند. پیشرفت‌های اخیر در طراحی مواد—مانند کامپوزیت‌های گوگرد-کربن، \english{MOF}های عامل‌دار شده، \english{COF}ها، \english{MXene}ها و هتروساختارهای سلسله‌مراتبی—گام‌های مهمی در افزایش بهره‌برداری از گوگرد، کاهش انحلال پلی‌سولفید، پایدارسازی فصل مشترک‌های فلز لیتیوم و تسریع سینتیک ردوکس برداشته‌اند. به موازات آن، استراتژی‌های مهندسی از جمله اصلاحات جداکننده، لایه‌های میانی محافظ، سیستم‌های الکترولیت جامد و هیبریدی و معماری‌های پیشرفته کاتد بیشتر به بهبود عملکرد باتری، عمر چرخه‌ای و ایمنی کمک کرده‌اند. با نگاه به آینده، تحقیقات آتی باید اولویت را به ترجمه دستاوردهای مقیاس آزمایشگاهی به سیستم‌های تجاری‌شدنی بدهند. جهت‌های کلیدی شامل سنتز مقیاس‌پذیر و مقرون‌به‌صرفه مواد عملکردی، توسعه کاتدهای با بارگذاری بالای گوگرد با ظرفیت‌های مساحتی عملی و نسبت‌های پایین الکترولیت به گوگرد و بهینه‌سازی فرمولاسیون‌های الکترولیت با عمر طولانی و ایمن است. علاوه بر این، ادغام معماری‌های الکترولیت جامد و هیبریدی یک مسیر امیدوارکننده به سوی دستیابی به چگالی انرژی بالاتر و پایداری عملیاتی افزایش‌یافته ارائه می‌دهد. با غلبه بر این چالش‌های علمی و مهندسی، باتری‌های \english{Li-S} می‌توانند نقش تحول‌آفرینی در برقی‌سازی حمل‌ونقل، پیشرفت دستگاه‌های الکترونیکی قابل حمل و استقرار سیستم‌های ذخیره‌سازی انرژی تجدیدپذیر در مقیاс شبکه ایفا کنند—و به طور قابل توجهی به آینده‌ای پایدار کمک کنند.
}

\section*{\persian{مشارکت نویسندگان}}
\persian{
نویسنده در مفهوم‌پردازی، نگارش و ویرایش این مقاله مروری مشارکت داشته است.
}

% --- End of content for page 22 ---
% --- SECTION 23: Page 23 of Translation ---
% --- Notes: Includes acknowledgements and the start of the bibliography.
% --------------------------------------------------------------------------

\section*{\persian{دسترسی به داده‌ها}}
\persian{
هیچ نتیجه تحقیق اولیه، نرم‌افزار یا کدی در این مقاله مروری گنجانده نشده و هیچ داده جدیدی به عنوان بخشی از این مرور تولید یا تحلیل نشده است.
}

\vspace{1em}

\section*{\persian{تضاد منافع}}
\persian{
هیچ تضاد منافعی برای اعلام وجود ندارد.
}

\vspace{1em}

\section*{\persian{سپاس‌گزاری}}
\persian{
این کار از حمایت مالی واحد مدیریت برنامه برای افزایش رقابت‌پذیری ملی (\english{PMU-C}) توسط دفتر شورای سیاست‌گذاری ملی آموزش عالی، تحقیقات علمی و نوآوری (\english{NXPO})، شرکت سهامی عام \english{PTT}، و شرکت سهامی عام \english{IRPC}، تحقیقات و نوآوری علمی تایلند (\english{TSRI}) تحت صندوق بنیادین توسط \english{TSRI (FRB680014/0457)}، مؤسسه علم و فناوری ویدیاسیریمدی (\english{VISTEC})، و دفتر سیاست‌گذاری و برنامه‌ریزی انرژی (\english{EPPO})، وزارت انرژی، تایلند برخوردار بوده است. علاوه بر این، مرکز تحقیقات مرزی (\english{FRC}) از طریق \english{VISTEC} از این کار حمایت کرد.
}

\vspace{2em}

\rule{\columnwidth}{0.4pt}
\vspace{1em}


\section*{\persian{یادداشت‌ها و مراجع}}

\begin{thebibliography}{99}
\setlength{\itemsep}{0pt} % Reduces space between items

\bibitem{ref1} \english{A. Manthiram, S. H. Chung and C. Zu, Adv. Mater., 2015, 27, 1980-2006.}
\bibitem{ref2} \english{A. Manthiram, Y. Fu, S. H. Chung, C. Zu and Y. S. Su, Chem. Rev., 2014, 114, 11751-11787.}
\bibitem{ref3} \english{A. Manthiram, Y. Fu and Y. S. Su, Acc. Chem. Res., 2013, 46, 1125-1134.}
\bibitem{ref4} \english{Y. S. Su and A. Manthiram, Nat. Commun., 2012, 3, 1166.}
\bibitem{ref5} \english{X. Ji, K. T. Lee and L. F. Nazar, Nat. Mater., 2009, 8, 500-506.}
\bibitem{ref6} \english{N. Jayaprakash, J. Shen, S. S. Moganty, A. Corona and L. A. Archer, Angew. Chem., Int. Ed., 2011, 50, 5904-5908.}
\bibitem{ref7} \english{X. Wu, H. Pan, M. Zhang, H. Zhong, Z. Zhang, W. Li, X. Sun, X. Mu, S. Tang, P. He and H. Zhou, Adv. Sci., 2024, 11, 2308604.}
\bibitem{ref8} \english{L. Zhou, D. L. Danilov, F. Qiao, J. Wang, H. Li, R.-A. Eichel and P. H. L. Notten, Adv. Energy Mater., 2022, 12, 2202094.}
\bibitem{ref9} \english{B. Wang, L. Wang, B. Zhang, S. Zeng, F. Tian, J. Dou, Y. Qian and L. Xu, ACS Nano, 2022, 16, 4947-4960.}
\bibitem{ref10} \english{L. Luo, J. Li, H. Yaghoobnejad Asl and A. Manthiram, ACS Energy Lett., 2020, 5, 1177-1185.}
\bibitem{ref11} \english{C. Li, W. Ge, S. Qi, L. Zhu, R. Huang, M. Zhao, Y. Qian and L. Xu, Adv. Energy Mater., 2022, 12, 2103915.}
\bibitem{ref12} \english{J. Kang, D.-Y. Han, S. Kim, J. Ryu and S. Park, Adv. Mater., 2023, 35, 2203194.}
\bibitem{ref13} \english{Z. Ye, Y. Jiang, L. Li, F. Wu and R. Chen, eScience, 2023, 3, 100107.}
\bibitem{ref14} \english{C. Julen, Q. Lixin, S. Alexander, J. Xabier, B. Amaia Sáenz de, J.-M. Gonzalo, A. Michel, Z. Heng and L. Chunmei, Energy Mater., 2022, 2, 200003.}
\bibitem{ref15} \english{M. Hagen, D. Hanselmann, K. Ahlbrecht, R. Maça, D. Gerber and J. Tübke, Adv. Energy Mater., 2015, 5, 1401986.}
\bibitem{ref16} \english{S. Kaenket, S. Duangdangchote, K. Homlamai, N. Joraleechanchai, T. Sangsanit, W. Tejangkura and M. Sawangphruk, Chem. Commun., 2023, 59, 10376-10379.}
\bibitem{ref17} \english{L. Huang, T. Lu, G. Xu, X. Zhang, Z. Jiang, Z. Zhang, Y. Wang, P. Han, G. Cui and L. Chen, Joule, 2022, 6, 906-922.}
\bibitem{ref18} \english{R. Fang, S. Zhao, S. Pei, X. Qian, P.-X. Hou, H.-M. Cheng, C. Liu and F. Li, ACS Nano, 2016, 10, 8676-8682.}
\bibitem{ref19} \english{Y.-C. Ko, C.-H. Hsu, C.-A. Lo, C.-M. Wu, H.-L. Yu, C.-H. Hsu, H.-P. Lin, C.-Y. Mou and H.-L. Wu, ACS Sustainable Chem. Eng., 2022, 10, 4462-4472.}
\bibitem{ref20} \english{C. Kensy, D. Leistenschneider, S. Wang, H. Tanaka, S. Dörfler, K. Kaneko and S. Kaskel, Batteries Supercaps, 2021, 4, 612-622.}
\bibitem{ref21} \english{C. Dong, H. Shi, H. Cui, S. Yu, Y. Li, Y. Ma, Y. Guo, Y. Dong, L. Zhang, C. Li, Y. Yu and Z. S. Wu, Energy Storage Mater., 2025, 75, 103987.}
\bibitem{ref22} \english{J. P. Grace and S. K. Martha, J. Energy Storage, 2024, 88, 111585.}
\bibitem{ref23} \english{Y. Cheng, B. Liu, X. Li, X. He, Z. Sun, W. Zhang, Z. Gao, L. Zhang, X. Chen, Z. Chen, Z. Chen, L. Peng and X. Duan, Carbon Energy, 2024, 6, e599.}
\bibitem{ref24} \english{T. Wang, Q. Zhang, J. Zhong, M. Chen, H. Deng, J. Cao, L. Wang, L. Peng, J. Zhu and B. Lu, Adv. Energy Mater., 2021, 11, 2100448.}
\bibitem{ref25} \english{J. Liu, X. Liu, Q. Zhang, X. Liang, J. Yan, H. H. Tan, Y. Yu and Y. Wu, Electrochim. Acta, 2021, 382, 138267.}
\bibitem{ref26} \english{Y. Hou, Y. Ren, S. Zhang, K. Wang, F. Yu and T. Zhu, J. Alloys Compd., 2021, 852, 157011.}
\bibitem{ref27} \english{X. Wen, K. Xiang, Y. Zhu, L. Xiao, H. Liao, W. Chen, X. Chen and H. Chen, J. Alloys Compd., 2020, 815, 152350.}
\bibitem{ref28} \english{J. Tan, D. Li, Y. Liu, P. Zhang, Z. Qu, Y. Yan, H. Hu, H. Cheng, J. Zhang, M. Dong, C. Wang, J. Fan, Z. Li, Z. Guo and M. Liu, J. Mater. Chem. A, 2020, 8, 7980-7990.}
\bibitem{ref29} \english{N. Li, Z. Xu, P. Wang, Z. Zhang, B. Hong, J. Li and Y. Lai, Chem. Eng. J., 2020, 398, 125432.}
\bibitem{ref30} \english{P. Chiochan, S. Kosasang, N. Ma, S. Duangdangchote, P. Suktha and M. Sawangphruk, Carbon, 2020, 158, 244-255.}
\bibitem{ref31} \english{X. Zhu, J. Ye, Y. Lu and X. Jia, ACS Sustainable Chem. Eng., 2019, 7, 11241-11249.}
\bibitem{ref32} \english{H. Shi, X. Zhao, Z. S. Wu, Y. Dong, P. Lu, J. Chen, W. Ren, H. M. Cheng and X. Bao, Nano Energy, 2019, 60, 743-751.}
\bibitem{ref33} \english{S. Choi, et al., J. Mater. Chem. A, 2019, 7, 4596-4603.}
\bibitem{ref34} \english{D. Cheng, P. Wu, J. Wang, X. Tang, T. An, H. Zhou, D. Zhang and T. Fan, Carbon, 2019, 143, 869-877.}
\bibitem{ref35} \english{J. Cai, Z. Zhang, S. Yang, Y. Min, G. Yang and K. Zhang, Electrochim. Acta, 2019, 295, 900-909.}
\bibitem{ref36} \english{K. Wu, et al., Electrochim. Acta, 2018, 291, 24-30.}
\bibitem{ref37} \english{Y. Wang, J. Huang, X. Chen, L. Wang and Z. Ye, Carbon, 2018, 137, 368-378.}
\bibitem{ref38} \english{J. L. Shi, C. Tang, J. Q. Huang, W. Zhu and Q. Zhang, J. Energy Chem., 2018, 27, 167-175.}
\bibitem{ref39} \english{D. Liu, Q. Li, J. Hou and H. Zhao, Sustainable Energy Fuels, 2018, 2, 2197-2205.}
\bibitem{ref40} \english{Z. Li, R. Xu, S. Deng, X. Su, W. Wu, S. Liu and M. Wu, Appl. Surf. Sci., 2018, 433, 10-15.}
\bibitem{ref41} \english{Z. Li, et al., Adv. Mater., 2018, 30, 1804089.}
\bibitem{ref42} \english{N. Li, F. Gan, P. Wang, K. Chen, S. Chen and X. He, J. Alloys Compd., 2018, 754, 64-71.}
\bibitem{ref43} \english{J. He, Y. Chen and A. Manthiram, iScience, 2018, 4, 36-43.}
\bibitem{ref44} \english{Z. Guo, et al., Adv. Sci., 2018, 5, 1800026.}
\bibitem{ref45} \english{Y. Chen, S. Choi, D. Su, X. Gao and G. Wang, Nano Energy, 2018, 47, 331-339.}
\bibitem{ref46} \english{A. Benítez, et al., J. Power Sources, 2018, 397, 102-112.}
\bibitem{ref47} \english{Z. Zhang, L. L. Kong, S. Liu, G. R. Li and X. P. Gao, Adv. Energy Mater., 2017, 7, 1602543.}
\bibitem{ref48} \english{C. Li, X. L. Sui, Z. B. Wang, Q. Wang and D. M. Gu, Chem. Eng. J., 2017, 326, 265-272.}
\bibitem{ref49} \english{Y. Chen, S. Lu, J. Zhou, X. Wu, W. Qin, O. Ogoke and G. Wu, J. Mater. Chem. A, 2017, 5, 102-112.}
\bibitem{ref50} \english{G. Zhou, E. Paek, G. S. Hwang and A. Manthiram, Adv. Energy Mater., 2016, 6, 1501355.}
\bibitem{ref51} \english{D. H. Wang, D. Xie, T. Yang, Y. Zhong, X. L. Wang, X. H. Xia, C. D. Gu and J. P. Tu, J. Power Sources, 2016, 313, 233-239.}
\bibitem{ref52} \english{Y. Hou, J. Li, X. Gao, Z. Wen, C. Yuan and J. Chen, Nanoscale, 2016, 8, 8228-8235.}
\bibitem{ref53} \english{J. He, et al., J. Power Sources, 2016, 327, 474-480.}
\bibitem{ref54} \english{Y. L. Ding, P. Kopold, K. Hahn, P. A. Van Aken, J. Maier and Y. Yu, Adv. Funct. Mater., 2016, 26, 1112-1119.}
\bibitem{ref55} \english{S. Liu, K. Xie, Z. Chen, Y. Li, X. Hong, J. Xu, L. Zhou, J. Yuan and C. Zheng, J. Mater. Chem. A, 2015, 3, 11395-11402.}

\end{thebibliography}

% --- End of content for page 23 ---
% --- SECTION 24: Page 24 of Translation ---
% --- Notes: Continues the bibliography (References 56-121).
% --------------------------------------------------------------------------

\bibitem{ref56} \english{J. Zhang, C. Wang, K. Wan, H. Guo, H. Zhou, J. Chen, X. Chen and Z. Li, J. Mater. Chem. A, 2015, 3, 19258-19266.}
\bibitem{ref57} \english{Y. M. Li, K. R. Kaiser, J. Ma, Y. Hou, T. Zhou, J. Chen, W. Lai, J. Chen, H. Zhou and L. Mai, ACS Appl. Mater. Interfaces, 2020, 12, 958-967.}
\bibitem{ref58} \english{R. Jiang, C. Chen, J. Zhang, J. Zhong, M. Xu and L. Fei, Angew. Chem., Int. Ed., 2024, 63, e202414771.}
\bibitem{ref59} \english{X. Liu, M. Lu, W. Qu, J. Zhou, M. Lu, W. Ou, L. H. Chung and J. He, J. Power Sources, 2025, 618, 235005.}
\bibitem{ref60} \english{D. Zhang, R. Zheng, H. Lu, J. Yang, J. Rong, J. Weng, B. Zhang and J. Xue, Energy Storage Energy, 2024, 18, 1343-1353.}
\bibitem{ref61} \english{Y. Yao, M. Zhao, Q. Zhao and S. Yang, Carbon, 2024, 223, 119018.}
\bibitem{ref62} \english{C. Xie, Q. Zeng, Y. Yang, F. Yang, X. Zhang, X. You, D. Long, H. Yin, L. Zhang and Z. Sun, Carbon, 2024, 18, 12821-12829.}
\bibitem{ref63} \english{H. Gao, G. Wu, J. Yang, R. Wang, J. Yang, L. Lyu, G. Diao, W. Zhang and Z. Pang, Adv. Mater., 2024, 36, 2415133.}
\bibitem{ref64} \english{H. Lv, Q. Yang, C. Li, C. Li, Q. Zhang and L. Wang, Adv. Energy Mater., 2024, 15, 2403223.}
\bibitem{ref65} \english{H. Liu, Q. Zhang, X. Yang and J. Yu, Adv. Energy Mater., 2024, 14, 2311859.}
\bibitem{ref66} \english{X. Li, et al., Angew. Chem., Int. Ed., 2024, 63, e202409618.}
\bibitem{ref67} \english{M. Wang, M. K. Shehab, H. Yang, Y. Tan, A. I. Ahmed, S. A. El-Hakam, Q. S. Ibrahim, P. Jena and H. M. El-Kaderi, ACS Appl. Mater. Interfaces, 2024, 16, 2692-2702.}
\bibitem{ref68} \english{Y. Fei, Z. Li, Z. Liu, Q. Zhang, H. Deng, L. Zhou, L. Li and G. Yang, ACS Appl. Mater. Interfaces, 2024, 16, 8435-8443.}
\bibitem{ref69} \english{S. Wang, et al., Adv. Funct. Mater., 2023, 33, 2306710.}
\bibitem{ref70} \english{Y. Xiao, S. Guo, Q. Yang, D. Li, S. Huang, Y. Ouyang, A. Cherevan, L. Deng, S. Eder, Q. Wang and S. Huang, ACS Energy Lett., 2023, 8, 5107-5115.}
\bibitem{ref71} \english{W. Jin, J. Wang, S. Zhang, L. Yang, T. Gong, B. He, H. Wang and W. Wang, ACS Appl. Mater. Interfaces, 2023, 15, 43962-43970.}
\bibitem{ref72} \english{X. Wang, et al., Angew. Chem., Int. Ed., 2023, 62, e202306901.}
\bibitem{ref73} \english{Q. Zhao, et al., Nano Energy, 2023, 116, 108813.}
\bibitem{ref74} \english{L. Zhou, H. Liang, Y. Yang, D. Wei, W. Pan, C. Liu, L. Jin, J. Sun and L. Huang, Adv. Funct. Mater., 2023, 33, 2303538.}
\bibitem{ref75} \english{Y. Guo, et al., Adv. Mater., 2023, 35, 2302787.}
\bibitem{ref76} \english{D. Yang, J. Li, R. Morante, S. S. Chou and A. Cabot, Adv. Mater., 2022, 34, 2208355.}
\bibitem{ref77} \english{Y. Xiao, S. Guo, Q. Yang, Y. Ouyang, D. Li, Q. Zeng, W. Gong, L. Gan, Q. Zhang and S. Huang, Energy Storage Mater., 2022, 53, 212-221.}
\bibitem{ref78} \english{X. Xu, B. Zhao, S. Xu, B. Li, J. Yin, Y. Zhang, Q. Zhou, Z. Chen and X. Zheng, ACS Appl. Mater. Interfaces, 2022, 14, 50809-50820.}
\bibitem{ref79} \english{S. Wang, et al., ACS Appl. Mater. Interfaces, 2022, 14, 50815-50826.}
\bibitem{ref80} \english{W. Zhang, et al., J. Mater. Chem. A, 2022, 10, 14761-14767.}
\bibitem{ref81} \english{Y. Zhang, et al., Nano, 2021, 15, 13893-13903.}
\bibitem{ref82} \english{S. Wang, F. Huang, Z. Zhang, W. Cai, X. Jie, S. Wang, P. Yan, S. Jiao and J. Yang, J. Mater. Chem. A, 2021, 9, 13400-13408.}
\bibitem{ref83} \english{Z. Zhang, F. Li, Z. Dai, S. Gao and M. Zhao, ACS Appl. Mater. Interfaces, 2021, 13, 61007-61014.}
\bibitem{ref84} \english{F. Qi, Z. Sun, J. Zhang, W. Zhang, T. Shi, G. Hu and F. Li, Adv. Energy Mater., 2021, 11, 2100387.}
\bibitem{ref85} \english{S. E. Jeon and S. L. Cui, ACS Nano, 2021, 15, 13893-13903.}
\bibitem{ref86} \english{L. Jeon and S. L. Chang, J. Mater. Chem. A, 2021, 9, 23923-23940.}
\bibitem{ref87} \english{Z. Zhu, X. Gao, C. Wang, W. Hou, Q. Zhang, J. Mater. Chem. A, 2021, 9, 21918-21931.}
\bibitem{ref88} \english{S. H. Yu, Q. Jiao, J. Wang, H. Deng and H. Zhou, J. Mater. Chem. A, 2021, 9, 21462-21473.}
\bibitem{ref89} \english{Y. B. Park, J. S. Park and D. W. Kim, Small, 2020, 16, 2004806.}
\bibitem{ref90} \english{S. Bai, et al., Nat. Energy, 2016, 1, 16094.}
\bibitem{ref91} \english{S. Xiang, et al., Energy Storage Mater., 2023, 63, 103094.}
\bibitem{ref92} \english{S. Razaq, et al., Adv. Energy Mater., 2024, 14, 2400924.}
\bibitem{ref93} \english{S. Janakiram, et al., Adv. Mater., 2024, 36, 2401667.}
\bibitem{ref94} \english{S. Kim, J. Jeon, S. Cho and S. Chou, Adv. Funct. Mater., 2024, 34, 2403693.}
\bibitem{ref95} \english{J. Zhang, et al., Adv. Energy Mater., 2023, 13, 2302302.}
\bibitem{ref96} \english{D. Han, W. Qin, M. Qiu, Z. Zhu, L. Zhang, H. Li, Y. Wang, Y. Zhang and L. Zhai, Nano Energy, 2025, 134, 110585.}
\bibitem{ref97} \english{C. Sun and J. Liu, Adv. Mater., 2024, 36, 2307211.}
\bibitem{ref98} \english{Y. Yu, et al., Adv. Mater., 2023, 35, 2303248.}
\bibitem{ref99} \english{W. Jia, M. Jiang, D. Gao, X. Chen, C. Chen, C. Li, C. Sun, W. Ren, X. Tian, S. Li and H. Zhang, J. Mater. Chem. A, 2024, 12, 13629-13639.}
\bibitem{ref100} \english{C. Wang, et al., J. Mater. Chem. A, 2024, 36, 13540-13548.}
\bibitem{ref101} \english{S. Wang, J. Zhang, W. Wang, W. Wu and Z. Wang, ACS Appl. Mater. Interfaces, 2024, 16, 3774-3781.}
\bibitem{ref102} \english{D. Han, et al., Energy Storage Mater., 2024, 66, 103222.}
\bibitem{ref103} \english{X. Zhu, Y. Wu, S. Y. Lee and L. Mi, Energy Storage Mater., 2024, 66, 103222.}
\bibitem{ref104} \english{X. Zhu, et al., Energy Storage Mater., 2024, 16, 1559-1567.}
\bibitem{ref105} \english{J. Liu, et al., J. Mater. Chem. A, 2023, 46, 25146-25155.}
\bibitem{ref106} \english{S. Liu, et al., Adv. Mater., 2023, 35, 2304394.}
\bibitem{ref107} \english{L. Jin, X. Yu, J. Dong, T. Zhang, J. Jin, J. Luo and K. Wang, H. Yu and J. H. Xu, Chem. Eng. J., 2023, 462, 142103.}
\bibitem{ref108} \english{J. Hu, et al., J. Mater. Chem. A, 2023, 11, 10992-11000.}
\bibitem{ref109} \english{J. Shin, et al., J. Mater. Chem. A, 2023, 11, 9921-9930.}
\bibitem{ref110} \english{Y. Ge, et al., Y. Yao, Y. Li, Y. Wang, Y. Li, W. Jin and J. Ge, J. Mater. Chem. A, 2023, 11, 4138-4146.}
\bibitem{ref111} \english{Y. Ge, J. Li, Y. Meng and D. Xiao, Nano Energy, 2023, 109, 108297.}
\bibitem{ref112} \english{Y. Cao, C. Liu, M. Wang, H. Yang, S. Liu, H. Wang, Z. Yang, F. Pan and J. Sun, ACS Nano, 2023, 17, 22632-22641.}
\bibitem{ref113} \english{H. Duan, K. Li, M. Xie, L. Li, M. Cao, H. G. Wang, W. Gao, H. Ning, Y. Chen, Y. Liu and G. Li, Chem. Eng. J., 2021, 420, 129753.}
\bibitem{ref114} \english{V. A. D. et al., J. Mater. Chem. A, 2021, 9, 3897-3906.}
\bibitem{ref115} \english{S. Chen, et al., Adv. Mater., 2018, 30, 1700548.}
\bibitem{ref116} \english{Z. Chen, et al., Energy Storage Mater., 2018, 14, 74-82.}
\bibitem{ref117} \english{Z. Chen, H. Liu, B. Zhang, H. Meng, X. Xu, R. Wang, X. Ai and L. Zhang, J. Mater. Chem. A, 2016, 4, 7416-7421.}
\bibitem{ref118} \english{Z. Liang, et al., J. Mater. Chem. A, 2016, 4, 8854-8859.}
\bibitem{ref119} \english{Y. Li, H. Ding, X. Ai and C. Wang, Mater. Chem. A, 2014, 2, 8854-8859.}
\bibitem{ref120} \english{H. Jin, et al., Commun. Chem., 2023, 60, 12762-12765.}
\bibitem{ref121} \english{S. Haldar, A. Schneemann and S. Kaskel, J. Am. Chem. Soc., 2023, 145, 13934-13941.}

\end{thebibliography}

% --- End of content for page 24 ---
% --- SECTION 25: Page 25 of Translation (Final) ---
% --- Notes: Completes the bibliography (References 122-182).
% --------------------------------------------------------------------------

\bibitem{ref122} \english{O. Buyukcakir, et al., Adv. Funct. Mater., 2020, 30, 2003761.}
\bibitem{ref123} \english{D. Han, et al., Nano Energy, 2025, 134, 110585.}
\bibitem{ref124} \english{Y. Cao, et al., Energy Storage Mater., 2020, 29, 207-215.}
\bibitem{ref125} \english{L. Sun, et al., Energy Storage Mater., 2024, 66, 103222.}
\bibitem{ref126} \english{W. Liu, et al., J. Am. Chem. Soc., 2022, 144, 17209-17218.}
\bibitem{ref127} \english{X. Zhang, et al., Energy Environ. Sci., 2024, 17, 7403-7415.}
\bibitem{ref128} \english{X. Zhu, M. Ge, T. Sun, X. Yuan and Y. Li, J. Phys. Chem. Lett., 2023, 14, 2215-2221.}
\bibitem{ref129} \english{W. Y. Lieu, et al., Nano Lett., 2023, 23, 5762-5769.}
\bibitem{ref130} \english{M. Fang, et al., J. Am. Chem. Soc., 2023, 145, 12601-12608.}
\bibitem{ref131} \english{R. Cheng, et al., Phys. Chem. Chem. Phys., 2023, 25, 10635-10646.}
\bibitem{ref132} \english{T. Zhang, L. Zhang and Y. Hou, eScience, 2022, 2, 164-182.}
\bibitem{ref133} \english{L. Zhang, et al., Phys. Chem. Chem. Phys., 2022, 24, 8913-8922.}
\bibitem{ref134} \english{A. J. Y. Wong, W. Y. Lieu, H. Y. Yang and Z. W. Seh, J. Mater. Res., 2022, 37, 3890-3905.}
\bibitem{ref135} \english{C. Ling, et al., Micro Nanostruct., 2022, 168, 207303.}
\bibitem{ref136} \english{W. Y. Lieu, et al., Nano Lett., 2022, 22, 8679-8687.}
\bibitem{ref137} \english{J. Y. Hou, et al., Sci. China: Technol. Sci., 2022, 65, 2259-2273.}
\bibitem{ref138} \english{Z. Chen, Z. Chang, Z. Liu and N. Zhou, Appl. Surf. Sci., 2022, 602, 154375.}
\bibitem{ref139} \english{L. Giebeler and J. Balach, Mater. Today Commun., 2021, 27, 102323.}
\bibitem{ref140} \english{Q. Zhang, et al., ACS Omega, 2020, 5, 29272-29283.}
\bibitem{ref141} \english{C. Zhang, L. Cui, S. Abdolhosseinzadeh and J. Heier, InfoMat, 2020, 2, 613-638.}
\bibitem{ref142} \english{X. Wang, Y. Cai, S. Wu and B. Li, Appl. Surf. Sci., 2020, 525, 146501.}
\bibitem{ref143} \english{D. K. Lee, Y. Chae, H. Yun, C. W. Ahn and J. W. Lee, ACS Nano, 2020, 14, 9744-9754.}
\bibitem{ref144} \english{H. Lin, D. D. Yang, N. Lou, S. G. Zhu and H. Z. Li, Ceram. Int., 2019, 45, 1588-1594.}
\bibitem{ref145} \english{N. Li, et al., Nanoscale, 2019, 11, 8485-8493.}
\bibitem{ref146} \english{Y. Zhao and J. Zhao, Appl. Surf. Sci., 2017, 412, 591-598.}
\bibitem{ref147} \english{P. Chiochan, X. Yu, M. Sawangphruk and A. Manthiram, Adv. Energy Mater., 2020, 10, 2001285.}
\bibitem{ref148} \english{S. Jin, et al., J. Power Sources, 2025, 629, 236055.}
\bibitem{ref149} \english{F. Teng, et al., Chem. Eng. J., 2025, 505, 159216.}
\bibitem{ref150} \english{M. Zhu, et al., Appl. Surf. Sci., 2025, 682, 161718.}
\bibitem{ref151} \english{C. Cheng, et al., Carbon, 2025, 233, 119897.}
\bibitem{ref152} \english{Z. Lian, et al., Appl. Catal., B, 2025, 361, 124661.}
\bibitem{ref153} \english{H. Xing, et al., J. Colloid Interface Sci., 2025, 677, 181-193.}
\bibitem{ref154} \english{M. Shu, et al., ACS Appl. Mater. Interfaces, 2025, 17, 4961-4971.}
\bibitem{ref155} \english{Y. Luo, et al., Chem. Eng. J., 2025, 505, 159158.}
\bibitem{ref156} \english{N. B. Songwe Selabi, et al., J. Power Sources, 2025, 626, 235785.}
\bibitem{ref157} \english{H. Xing, P. Yang, Y. Niu, Z. Wen and Y. Xu, Chem. Eng. J., 2025, 503, 158265.}
\bibitem{ref158} \english{Y. Liu, G. Qin, M. Song, Y. Huang and X. Huang, Nano Energy, 2025, 133, 110508.}
\bibitem{ref159} \english{C. Huang, et al., Angew. Chem., Int. Ed., 2025, 64, e202420488.}
\bibitem{ref160} \english{Z. Wang, et al., Small, 2025, 21, 2409867.}
\bibitem{ref161} \english{X. Sun, et al., Nat. Commun., 2023, 14, 291.}
\bibitem{ref162} \english{J. Li, F. Xie, W. Pang, Q. Liang, X. Yang and L. Zhang, Sci. Adv., 2024, 10, ead13925.}
\bibitem{ref163} \english{F. Pei, et al., Nat. Commun., 2024, 15, 351.}
\bibitem{ref164} \english{K. Nie, et al., J. Power Sources, 2024, 613, 234833.}
\bibitem{ref165} \english{R. Liu, et al., Mater. Today Energy, 2023, 38, 101452.}
\bibitem{ref166} \english{J. Castillo, et al., APL Mater., 2023, 11, 010901.}
\bibitem{ref167} \english{J. Xu, et al., Energy Storage Mater., 2022, 47, 223-234.}
\bibitem{ref168} \english{X. Kong, et al., Small, 2022, 18, 2205017.}
\bibitem{ref169} \english{X. Kong, Y. Kong, X. Liao, S. Liu and Y. Zhao, Sustainable Energy Fuels, 2022, 6, 3658-3668.}
\bibitem{ref170} \english{X. Chen, et al., Sci. China Mater., 2021, 64, 2127-2138.}
\bibitem{ref171} \english{Y. X. Yao, et al., InfoMat, 2020, 2, 379-388.}
\bibitem{ref172} \english{W. J. Chen, et al., Energy Environ. Mater., 2020, 3, 160-165.}
\bibitem{ref173} \english{X. Yan, H. Zhang, M. Huang, M. Qu and Z. Wei, ChemSusChem, 2019, 12, 2263-2270.}
\bibitem{ref174} \english{C. Yan, et al., Trends Chem., 2019, 1, 693-704.}
\bibitem{ref175} \english{Y. Wang, E. Sahadeo, G. Rubloff, C. F. Lin and S. B. Lee, J. Mater. Sci., 2019, 54, 3671-3693.}
\bibitem{ref176} \english{Q. J. Meisner, et al., J. Power Sources, 2019, 438, 226939.}
\bibitem{ref177} \english{C. Yan, et al., J. Power Sources, 2016, 327, 212-220.}
\bibitem{ref178} \english{W. T. Xu, et al., ChemSusChem, 2015, 8, 2892-2901.}
\bibitem{ref179} \english{G. Ma, et al., Chem. Commun., 2014, 50, 14209-14212.}
\bibitem{ref180} \english{C. Dong, et al., Energy Storage Mater., 2025, 75, 103987.}
\bibitem{ref181} \english{F. Huang, et al., J. Power Sources, 2025, 629, 236028.}
\bibitem{ref182} \english{H. Chen, et al., Adv. Energy Mater., 2019, 9, 1900858.}


% --- End of content for page 25 ---
\end{document}
