\documentclass[12pt,a4paper]{article}
\usepackage{polyglossia}
\usepackage{fontspec}
\usepackage{bidi}
\usepackage{geometry}
\usepackage{graphicx}
\usepackage{amsmath}

% --- Document Setup ---
\geometry{a4paper, margin=1in}
\setdefaultlanguage{english}
\setotherlanguage{farsi}

% --- Font Setup ---
% Note: Please ensure you have a suitable Persian font installed, like "XB Niloofar" or "B Nazanin".
\newfontfamily\farsifont[Script=Arabic, Scale=1.1]{XB Niloofar}
\newfontfamily\englishfont{Times New Roman}

% --- Custom Commands ---
\newcommand{\persian}[1]{\textfarsi{#1}}
\newcommand{\eng}[1]{\textenglish{#1}}
\renewcommand{\thefootnote}{\arabic{footnote}} % Standard footnote numbering

% ====================================================================
%                          DOCUMENT START
% ====================================================================
\begin{document}

\begin{center}
    \huge\bfseries\persian{ترجمه مقاله علمی: صفحه اول}
\end{center}
\vspace{1cm}

% --------------------------------------------------------------------
%                           ARTICLE TITLE
% --------------------------------------------------------------------
\section*{\persian{عنوان مقاله}}

\begin{quote}
    \eng{\textbf{Original Title:} New materials for lithium-sulfur batteries: challenges and future directions}
\end{quote}

\persian{\textbf{ترجمه عنوان:} مواد جدید برای باتری‌های لیتیوم-گوگرد: چالش‌ها و چشم‌اندازهای آینده}
\vspace{5mm}
\persian{\textbf{نویسنده:} مونتری ساوانگفروک (\eng{Montree Sawangphruk})}
\hrule
\vspace{5mm}

% --------------------------------------------------------------------
%                               ABSTRACT
% --------------------------------------------------------------------
\section*{\persian{چکیده}}

\begin{quote}
    \eng{\textbf{Original Abstract:} This review explores recent advances in lithium-sulfur (Li-S) batteries, promising next-generation energy storage devices known for their exceptionally high theoretical energy density ($\sim$2500 Wh kg⁻¹), cost-effectiveness, and environmental advantages.}
\end{quote}

\persian{
\textbf{ترجمه چکیده:} این مقاله مروری، پیشرفت‌های اخیر در باتری‌های لیتیوم-گوگرد (\eng{Li-S}) را بررسی می‌کند؛ دستگاه‌های ذخیره‌سازی انرژی نسل بعدی که به دلیل چگالی انرژی نظری فوق‌العاده بالا (حدود ۲۵۰۰ وات‌ساعت بر کیلوگرم)، مقرون‌به‌صرفه بودن و مزایای زیست‌محیطی، آینده‌ای درخشان دارند.
}

\section*{\persian{تست فونت‌ها}}

\persian{این متن برای تست فونت XB Niloofar نوشته شده است.}

\eng{This text is for testing English font (Times New Roman).}

\persian{اعداد فارسی: ۱۲۳۴۵۶۷۸۹۰}

\eng{English numbers: 1234567890}

\end{document}
% ====================================================================
%                            DOCUMENT END
% ====================================================================