% This document is designed to test Persian fonts with XeLaTeX.
% It incorporates best practices learned from the error log.
% Compile with: xelatex template.tex

\documentclass[12pt,a4paper]{article}

% IMPORTANT: Package Loading Order
% Load font packages first, followed by xepersian/polyglossia.
% This prevents conflicts with the bidi package (see Error 1).
\usepackage{fontspec}
\usepackage{geometry}

% --- CHOOSE ONE: xepersian or polyglossia ---

% OPTION 1: Using xepersian (recommended for Persian-focused documents)
% The xepersian package must come after fontspec and geometry (see Error 1).
\usepackage{xepersian}

% Use the correct command to set the font for xepersian (see Error 15).
% Use the correct font name. 'Vazir' is valid, but 'Vazirmatn' is the modern successor.
\settextfont{Vazirmatn}

% OPTION 2: Using polyglossia (if needed for multi-language documents)
% Note: Do NOT use xepersian and polyglossia together (see Error 21).
% \usepackage{polyglossia}
% \setdefaultlanguage{farsi}
% \setotherlanguage{english}
% Use the correct command name for polyglossia (see Error 6).
% \newfontfamily\arabicfont{Vazirmatn}


\geometry{margin=2cm}

\title{تست فونت‌های فارسی}
\author{تست}
\date{\today}

\begin{document}

\maketitle

\section{نمونه متن فارسی}

این یک نمونه متن فارسی است که برای تست فونت‌های مختلف استفاده می‌شود. در این متن حروف مختلف الفبای فارسی آمده است:

الف، ب، پ، ت، ث، ج، چ، ح، خ، د، ذ، ر، ز، ژ، س، ش، ص، ض، ط، ظ، ع، غ، ف، ق، ک، گ، ل، م، ن، و، ه، ی

\section{اعداد فارسی}
۰، ۱، ۲، ۳، ۴، ۵، ۶، ۷، ۸، ۹

\section{علائم نگارشی}
نقطه (.)، ویرگول (،)، نقطه‌ویرگول (؛)، دونقطه (:)، علامت سؤال (؟)، علامت تعجب (!)

\section{متن ترکیبی}
این متن شامل کلمات انگلیسی مثل \lr{LaTeX} و \lr{XeLaTeX} نیز می‌باشد.

\section{سبک‌های مختلف}
\textbf{متن ضخیم}، \textit{متن کج}، \underline{متن زیرخط‌دار}

\section{فهرست نمونه}
\begin{itemize}
\item مورد اول
\item مورد دوم  
\item مورد سوم
\end{itemize}

\end{document}