\documentclass{article}
\usepackage{fontspec}
\usepackage{xcolor}        % BEFORE xepersian (following best practices)
\usepackage{geometry}
\usepackage{graphicx}
\usepackage{hyperref}
\usepackage{listings}
\usepackage{fancyhdr}
\usepackage{tcolorbox}
\usepackage{enumitem}
\usepackage{amsmath}

% Page setup
\geometry{a4paper, margin=1in}
\pagestyle{fancy}
\fancyhf{}
\fancyhead[L]{Oh My Zsh Installation Guide}
\fancyhead[R]{\thepage}
\fancyfoot[C]{From Bash to Zsh with Oh My Zsh}

% Colors
\definecolor{codebackground}{RGB}{248,248,248}
\definecolor{codecolor}{RGB}{51,51,51}
\definecolor{commentcolor}{RGB}{0,128,0}
\definecolor{keywordcolor}{RGB}{0,0,255}

% Code listing setup
\lstset{
    backgroundcolor=\color{codebackground},
    basicstyle=\ttfamily\small\color{codecolor},
    commentstyle=\color{commentcolor},
    keywordstyle=\color{keywordcolor},
    breaklines=true,
    frame=single,
    frameround=tttt,
    numbers=left,
    numberstyle=\tiny\color{gray},
    showstringspaces=false,
    tabsize=2,
    captionpos=b
}

% Custom boxes
\newtcolorbox{warningbox}{
    colback=yellow!10,
    colframe=orange!75,
    title=Warning,
    fonttitle=\bfseries
}

\newtcolorbox{infobox}{
    colback=blue!5,
    colframe=blue!75,
    title=Information,
    fonttitle=\bfseries
}

\newtcolorbox{successbox}{
    colback=green!5,
    colframe=green!75,
    title=Success,
    fonttitle=\bfseries
}

\newtcolorbox{codebox}{
    colback=gray!10,
    colframe=gray!50,
    fontupper=\ttfamily\small
}

\title{Complete Guide: From Bash to Oh My Zsh\\
\large Including Codespaces Configuration}
\author{Terminal Enhancement Guide}
\date{\today}

\begin{document}

\maketitle

\tableofcontents
\newpage

\section{Introduction}

This guide provides comprehensive instructions for switching from the default bash shell to Zsh with Oh My Zsh framework. We'll cover installation, configuration, and specific instructions for GitHub Codespaces environments.

\begin{infobox}
\textbf{What is Oh My Zsh?}\\
Oh My Zsh is an open-source framework for managing Zsh configuration. It provides:
\begin{itemize}
    \item Beautiful themes and prompts
    \item Hundreds of helpful plugins
    \item Auto-completion enhancements
    \item Git integration
    \item Customizable aliases and functions
\end{itemize}
\end{infobox}

\section{Prerequisites}

Before starting, ensure you have:

\begin{enumerate}
    \item Terminal access (bash, zsh, or similar)
    \item Internet connection for downloading packages
    \item Basic command-line knowledge
    \item Administrative privileges (sudo access)
\end{enumerate}

\begin{warningbox}
\textbf{Backup Your Current Configuration}\\
Before making changes, backup your existing shell configuration:
\begin{verbatim}
cp ~/.bashrc ~/.bashrc.backup
cp ~/.bash_profile ~/.bash_profile.backup
\end{verbatim}
\end{warningbox}

\section{Step 1: Install Zsh}

\subsection{Ubuntu/Debian Systems}

\begin{codebox}
sudo apt update
sudo apt install zsh
\end{codebox}

\subsection{CentOS/RHEL/Fedora Systems}

\begin{codebox}
\# CentOS/RHEL
sudo yum install zsh

\# Fedora
sudo dnf install zsh
\end{codebox}

\subsection{macOS Systems}

\begin{codebox}
\# Using Homebrew
brew install zsh

\# Or using MacPorts
sudo port install zsh
\end{codebox}

\subsection{Verify Zsh Installation}

\begin{codebox}
zsh --version
which zsh
\end{codebox}

Expected output should show Zsh version 5.0 or higher.

\section{Step 2: Install Oh My Zsh}

\subsection{Method 1: Using curl}

\begin{verbatim}
sh -c "$(curl -fsSL https://raw.githubusercontent.com/ohmyzsh/ohmyzsh/master/tools/install.sh)"
\end{verbatim}

\subsection{Method 2: Using wget}

\begin{verbatim}
sh -c "$(wget https://raw.githubusercontent.com/ohmyzsh/ohmyzsh/master/tools/install.sh -O -)"
\end{verbatim}

\subsection{Method 3: Manual Installation}

\begin{codebox}
git clone https://github.com/ohmyzsh/ohmyzsh.git ~/.oh-my-zsh
cp ~/.oh-my-zsh/templates/zshrc.zsh-template ~/.zshrc
\end{codebox}

\section{Step 3: Change Default Shell to Zsh}

\subsection{Standard Linux/Unix Systems}

\begin{codebox}
chsh -s \$(which zsh)
\end{codebox}

\subsection{GitHub Codespaces - Method 1}

For GitHub Codespaces environments, use this command:

\begin{codebox}
sudo chsh -s \\$(which zsh) \\$(whoami)
\end{codebox}

\subsection{GitHub Codespaces - Method 2 (Alternative)}

If the first method doesn't work, try:

\begin{codebox}
sudo chsh "\$(id -un)" --shell "\$(which zsh)"
\end{codebox}

\begin{infobox}
\textbf{Why Different Commands for Codespaces?}\\
GitHub Codespaces runs in a containerized environment where:
\begin{itemize}
    \item Standard \texttt{chsh} might not have sufficient permissions
    \item \texttt{sudo} is required for shell changes
    \item User identification needs to be explicit
    \item Container restrictions may require alternative syntax
\end{itemize}
\end{infobox}

\subsection{Verify Shell Change}

\begin{codebox}
echo $SHELL
grep \$(whoami) /etc/passwd
\end{codebox}

The output should show the path to zsh (e.g., \texttt{/usr/bin/zsh}).

\section{Step 4: Restart Terminal}

After changing the shell, you need to restart your terminal session:

\begin{enumerate}
    \item Close your current terminal
    \item Open a new terminal session
    \item Or logout and login again
    \item For Codespaces: Restart the container or reload the window
\end{enumerate}

\begin{successbox}
If successful, you should see the Oh My Zsh welcome message and a new colorful prompt!
\end{successbox}

\section{Step 5: Basic Configuration}

\subsection{Edit Zsh Configuration}

\begin{codebox}
nano ~/.zshrc
\# or
vim ~/.zshrc
\end{codebox}

\subsection{Key Configuration Options}

\begin{lstlisting}[language=bash, caption=Basic .zshrc configuration]
\# Oh My Zsh installation path
export ZSH="$HOME/.oh-my-zsh"

\# Theme selection
ZSH_THEME="robbyrussell"  \# Default theme
\# ZSH_THEME="agnoster"    \# Popular alternative
\# ZSH_THEME="powerlevel10k/powerlevel10k"  \# Advanced theme

\# Plugins
plugins=(
    git
    zsh-autosuggestions
    zsh-syntax-highlighting
    docker
    kubectl
    node
    npm
)

\# Source Oh My Zsh
source $ZSH/oh-my-zsh.sh

\# Custom aliases
alias ll="ls -la"
alias la="ls -A"
alias l="ls -CF"
alias ..="cd .."
alias ...="cd ../.."

\# Custom functions
mkcd() {
    mkdir -p "$1" && cd "$1"
}
\end{lstlisting}

\section{Step 6: Install Popular Plugins}

\subsection{Zsh Autosuggestions}

\begin{codebox}
git clone https://github.com/zsh-users/zsh-autosuggestions ${ZSH_CUSTOM:-~/.oh-my-zsh/custom}/plugins/zsh-autosuggestions
\end{codebox}

\subsection{Zsh Syntax Highlighting}

\begin{codebox}
git clone https://github.com/zsh-users/zsh-syntax-highlighting.git ${ZSH_CUSTOM:-~/.oh-my-zsh/custom}/plugins/zsh-syntax-highlighting
\end{codebox}

\subsection{Powerlevel10k Theme (Optional)}

\begin{codebox}
git clone --depth=1 https://github.com/romkatv/powerlevel10k.git ${ZSH_CUSTOM:-$HOME/.oh-my-zsh/custom}/themes/powerlevel10k
\end{codebox}

Then set \texttt{ZSH\_THEME="powerlevel10k/powerlevel10k"} in your \texttt{.zshrc}.

\section{Step 7: Apply Changes}

After making configuration changes:

\begin{codebox}
source ~/.zshrc
\end{codebox}

\section{Troubleshooting}

\subsection{Common Issues and Solutions}

\begin{enumerate}
    \item \textbf{Permission Denied}
    \begin{codebox}
sudo chsh -s \$(which zsh) $USER
    \end{codebox}
    
    \item \textbf{Zsh Not Found}
    \begin{codebox}
which zsh
\# If empty, reinstall zsh
sudo apt install zsh  \# Ubuntu/Debian
    \end{codebox}
    
    \item \textbf{Oh My Zsh Installation Failed}
    \begin{codebox}
\# Check internet connection and try manual installation
curl -fsSL https://raw.githubusercontent.com/ohmyzsh/ohmyzsh/master/tools/install.sh
    \end{codebox}
    
    \item \textbf{Plugins Not Working}
    \begin{codebox}
\# Ensure plugins are in the correct directory
ls ~/.oh-my-zsh/custom/plugins/
\# Restart terminal after adding plugins
    \end{codebox}
\end{enumerate}

\subsection{Codespaces-Specific Issues}

\begin{warningbox}
\textbf{Container Persistence}\\
In GitHub Codespaces, remember that:
\begin{itemize}
    \item Shell changes persist across container restarts
    \item Custom configurations in \texttt{~/.zshrc} are preserved
    \item Installed plugins remain available
    \item Theme preferences are maintained
\end{itemize}
\end{warningbox}

\section{Advanced Configuration}

\subsection{Custom Prompt Configuration}

\begin{lstlisting}[language=bash, caption=Custom prompt example]
\# Custom prompt with git status
PROMPT='%F{cyan}%n@%m%f:%F{yellow}%~%f\$(git_prompt_info) %# '
ZSH_THEME_GIT_PROMPT_PREFIX=" %F{red}("
ZSH_THEME_GIT_PROMPT_SUFFIX=")%f"
ZSH_THEME_GIT_PROMPT_DIRTY="%F{yellow}*%f"
ZSH_THEME_GIT_PROMPT_CLEAN=""
\end{lstlisting}

\subsection{Environment Variables}

\begin{lstlisting}[language=bash, caption=Environment setup]
\# Development environment
export EDITOR="code"
export BROWSER="google-chrome"
export TERM="xterm-256color"

\# Path additions
export PATH="$HOME/.local/bin:$PATH"
export PATH="$HOME/bin:$PATH"

\# Node.js
export NVM_DIR="$HOME/.nvm"
[ -s "$NVM_DIR/nvm.sh" ] && \. "$NVM_DIR/nvm.sh"
\end{lstlisting}

\section{Useful Aliases and Functions}

\begin{lstlisting}[language=bash, caption=Productivity aliases]
\# Git aliases
alias gs="git status"
alias ga="git add"
alias gc="git commit"
alias gp="git push"
alias gl="git log --oneline"

\# Docker aliases
alias dps="docker ps"
alias dpa="docker ps -a"
alias di="docker images"
alias drm="docker rm"
alias drmi="docker rmi"

\# System aliases
alias h="history"
alias j="jobs"
alias c="clear"
alias e="exit"

\# Directory navigation
alias home="cd ~"
alias root="cd /"
alias dtop="cd ~/Desktop"
alias docs="cd ~/Documents"

\# File operations
alias cp="cp -i"
alias mv="mv -i"
alias rm="rm -i"
alias mkdir="mkdir -p"

\# Network
alias ping="ping -c 5"
alias ports="netstat -tulanp"
\end{lstlisting}

\section{Verification and Testing}

\subsection{Test Your Setup}

\begin{enumerate}
    \item \textbf{Check Shell}
    \begin{codebox}
echo $SHELL
ps -p $$
    \end{codebox}
    
    \item \textbf{Test Oh My Zsh}
    \begin{codebox}
omz version
omz plugin list
    \end{codebox}
    
    \item \textbf{Test Autocompletion}
    \begin{codebox}
git <TAB><TAB>
docker <TAB><TAB>
    \end{codebox}
    
    \item \textbf{Test Plugins}
    \begin{codebox}
\# Type a command and see suggestions
ls
\# Use arrow keys to navigate history
    \end{codebox}
\end{enumerate}

\section{Conclusion}

You have successfully:

\begin{successbox}
\begin{itemize}
    \item ✅ Installed Zsh shell
    \item ✅ Installed Oh My Zsh framework
    \item ✅ Changed default shell (including Codespaces methods)
    \item ✅ Configured themes and plugins
    \item ✅ Set up useful aliases and functions
    \item ✅ Verified the installation
\end{itemize}
\end{successbox}

\subsection{Next Steps}

\begin{enumerate}
    \item Explore more themes: \url{https://github.com/ohmyzsh/ohmyzsh/wiki/Themes}
    \item Discover plugins: \url{https://github.com/ohmyzsh/ohmyzsh/wiki/Plugins}
    \item Customize your prompt further
    \item Share your configuration with team members
    \item Consider using dotfiles for configuration management
\end{enumerate}

\subsection{Resources}

\begin{itemize}
    \item Oh My Zsh GitHub: \url{https://github.com/ohmyzsh/ohmyzsh}
    \item Zsh Documentation: \url{https://zsh.sourceforge.io/Doc/}
    \item Awesome Zsh Plugins: \url{https://github.com/unixorn/awesome-zsh-plugins}
    \item Powerlevel10k: \url{https://github.com/romkatv/powerlevel10k}
\end{itemize}

\begin{infobox}
\textbf{Remember}\\
Your terminal is now more powerful and user-friendly. Take time to explore the features and customize it to your workflow. Happy coding!
\end{infobox}

\end{document}