% Test Vazirmatn fonts with LaTeX
% Make sure fonts are installed on your system first
% Compile with: xelatex test_vazirmatn_fonts.tex

\documentclass[12pt,a4paper]{article}
\usepackage{fontspec}
\usepackage{xcolor}
\usepackage{geometry}
\usepackage{xepersian}

\geometry{margin=2cm}

% Test Vazirmatn font - change this line to test different weights
\settextfont{Vazirmatn}              % Main font
% \settextfont{Vazirmatn-Regular}    % Explicit regular
% \settextfont{Vazirmatn-Light}      % Light weight
% \settextfont{Vazirmatn-Medium}     % Medium weight
% \settextfont{Vazirmatn-Bold}       % Bold weight

\setdigitfont{Vazirmatn}
\setlatintextfont{Times New Roman}

\title{تست فونت وزیرمتن (Vazirmatn)}
\author{تست فونت}
\date{\today}

\begin{document}

\maketitle

\section{معرفی فونت}
\textcolor{blue}{فونت فعلی: وزیرمتن (Vazirmatn)}

اگر این متن به درستی نمایش داده شود، فونت وزیرمتن با موفقیت نصب و تنظیم شده است.

\section{تست وزن‌های مختلف فونت}

\subsection{متن عادی}
این متن با وزن عادی فونت وزیرمتن نوشته شده است.

\subsection{متن ضخیم}
\textbf{این متن با وزن ضخیم فونت وزیرمتن نوشته شده است.}

\subsection{متن کج}
\textit{این متن با حالت کج فونت وزیرمتن نوشته شده است.}

\subsection{ترکیب سبک‌ها}
\textbf{\textit{این متن هم ضخیم و هم کج است.}}

\section{تست حروف الفبا}
الف، ب، پ، ت، ث، ج، چ، ح، خ، د، ذ، ر، ز، ژ، س، ش، ص، ض، ط، ظ، ع، غ، ف، ق، ک، گ، ل، م، ن، و، ه، ی

\section{تست اعداد}

\subsection{اعداد فارسی}
۰، ۱، ۲، ۳، ۴، ۵، ۶، ۷، ۸، ۹

تاریخ امروز: \today

\subsection{اعداد انگلیسی}
\lr{0, 1, 2, 3, 4, 5, 6, 7, 8, 9}

\section{علائم نگارشی}
نقطه (.)، ویرگول (،)، نقطه‌ویرگول (؛)، دونقطه (:)، علامت سؤال (؟)، علامت تعجب (!)

نقل‌قول: «این یک نقل‌قول نمونه است.»

\section{متن ترکیبی فارسی-انگلیسی}
این متن شامل کلمات انگلیسی مثل \lr{LaTeX}، \lr{XeLaTeX}، \lr{Vazirmatn}، \lr{GitHub} و \lr{Persian Typography} نیز می‌باشد.

\section{فهرست‌ها}

\subsection{فهرست نقطه‌ای}
\begin{itemize}
\item مورد اول
\item مورد دوم
\item مورد سوم
    \begin{itemize}
    \item زیرمورد اول
    \item زیرمورد دوم
    \end{itemize}
\item مورد چهارم
\end{itemize}

\subsection{فهرست شماره‌دار}
\begin{enumerate}
\item گام اول: دانلود فونت
\item گام دوم: نصب فونت
\item گام سوم: تنظیم در \lr{LaTeX}
\item گام چهارم: تست و بررسی
\end{enumerate}

\section{نمونه متن طولانی}
لورم ایپسوم متن ساختگی با تولید سادگی نامفهوم از صنعت چاپ و با استفاده از طراحان گرافیک است. چاپگرها و متون بلکه روزنامه و مجله در ستون و سطرآنچنان که لازم است و برای شرایط فعلی تکنولوژی مورد نیاز و کاربردهای متنوع با هدف بهبود ابزارهای کاربردی می‌باشد.

\begin{quote}
«این یک نقل‌قول بلند است که برای نمایش کیفیت فونت در متن‌های طولانی‌تر و بررسی فاصله‌گذاری بین کلمات و خطوط استفاده می‌شود. فونت وزیرمتن باید این متن را به صورت زیبا و خوانا نمایش دهد.»
\end{quote}

\section{اندازه‌های مختلف فونت}

{\tiny متن بسیار کوچک}

{\scriptsize متن کوچک}

{\footnotesize متن یادداشت}

{\small متن کوچک}

{\normalsize متن عادی}

{\large متن بزرگ}

{\Large متن بزرگ‌تر}

{\LARGE متن خیلی بزرگ}

{\huge متن عظیم}

{\Huge متن بسیار عظیم}

\section{نتیجه‌گیری}
\textcolor{green}{\textbf{اگر تمام بخش‌های بالا به درستی نمایش داده شدند، فونت وزیرمتن با موفقیت نصب شده و آماده استفاده در پروژه‌های \lr{LaTeX} شماست!}}

\vspace{1cm}

\textcolor{red}{نکته مهم: اگر فونت به درستی نمایش داده نشد، مطمئن شوید که:}
\begin{enumerate}
\item فونت‌های \lr{.ttf} در سیستم نصب شده‌اند
\item از \lr{XeLaTeX} برای کامپایل استفاده می‌کنید (نه \lr{pdfLaTeX})
\item نام فونت در کد درست نوشته شده است
\end{enumerate}

\end{document}